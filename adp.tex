%======================================================
% Technische Universitaet Darmstadt
% Fachbereich Informatik
% Fachgebiet Simulation, Systemoptimierung und Robotik (SIM)
% Prof. Dr. Oskar von Stryk
%======================================================
% Template for Theses
% VERSION 1.0a (Mai 2010)
% Use pdfLaTeX (other possible, but not supported)
% Contact at SIM: Thomas Hemker (hemker@sim...)
%======================================================
% Official TUD-LaTeX-files have to be installed:
% http://exp1.fkp.physik.tu-darmstadt.de/tuddesign/
% Refer to the manuals and forum for details
%======================================================
\documentclass[longdoc,accentcolor=tud1b,11pt,paper=a4]{tudreport}
\definecolor{tudaccent}{RGB}{255,255,255}
\definecolor{tudidentbar}{RGB}{83,83,83}
%======================================================
% colorback = Bereich unter Titel mit Hintergrundfarbe
% colorbacktitle = Titel mit Hintergrundfarbe (Akzent)
% SIM-Blau = accentcolor=tud1b		
% Grau = accentcolor=tud0a
% blackrule fuer schwarze Leiste
% nochapterpage = do not start chapters on new page
% oneside = print only on one side of the page
%======================================================

%======================================================
% General package loading and definitions
%======================================================
\usepackage{inputenc}
\usepackage{textcomp}
\usepackage{ngerman}
\usepackage{xspace}
\usepackage{lineno}
%remove the following two comment mark for line number
%\pagewiselinenumbers
%\modulolinenumbers[1]
\usepackage{setspace}
\usepackage{todonotes}%Dies f\"{u}r Anmerkungen am Rand und im Text benutzen
%
\usepackage[fleqn]{amsmath} % math environments and more by the AMS
\newcounter{dummy} % necessary for correct hyperlinks (to index, bib, etc.)
\newcommand{\myfloatalign}{\centering} % how all the floats will be aligned

\usepackage{listings}

%======================================================
% SIM-modifications of the TUD-layout
%======================================================
% reduce font size of page footers and headers (fancyhdr)
\renewcommand{\footerfont}{\fontfamily{\sfdefault}\fontseries{m}\fontshape{n}\footnotesize\selectfont}
% remove space between items
\usepackage{enumitem}
	\setenumerate{noitemsep}
	\setitemize{noitemsep}
	\setdescription{noitemsep}
%\setlist{nolistsep}

%======================================================
% Package loading for example contents (content.tex)
%======================================================
\usepackage{tabularx} % better tables
\setlength{\extrarowheight}{3pt} % increase table row height
\usepackage{booktabs}
\usepackage{caption}
\captionsetup{format=hang,font=small}
\usepackage[square,numbers]{natbib}
\usepackage{subfig}
\usepackage[stable,bottom]{footmisc}
\usepackage{float}
\usepackage{url}

%======================================================
% Important information: to be set here and only here
%======================================================
\newcommand{\fzdTitle}{Entwicklung eines Konzepts und Konstruktion einer Versuchseinrichtung zur zeitlich synchronisierten Generierung von Partikel-Testsignalen
\xspace}
\newcommand{\fzdThesisType}{Advanced Design Project\xspace} 
\newcommand{\fzdID}{Nr. 123456\xspace}
\newcommand{\fzdName}{Alexander Sonnleitner, Dinh-Van Vo, Kim-Khanh Vo, Gia Thi Ngo, Felix Sternkopf\xspace}
\newcommand{\fzdSubmissionDate}{13. Februar 2018\xspace}
\newcommand{\fzdGutachter}{Gutachter: Prof. Dr. rer. nat. Hermann Winner\xspace}
\newcommand{\fzdBetreuer}{Betreuer: M.Sc. Hartmut Niemann\xspace}
\newcommand{\fzdExternerBetreuer}{\xspace}

%======================================================
% Setup for hyperref
%======================================================
\usepackage[pdftex,hyperfootnotes=true,pdfpagelabels]{hyperref}
	\pdfcompresslevel=9
	\pdfadjustspacing=1
\hypersetup{%
    colorlinks=false, linktocpage=false, pdfstartpage=1, pdfstartview=FitV,%
    breaklinks=true, pdfpagemode=UseNone, pageanchor=true, pdfpagemode=UseOutlines,%
    plainpages=false, bookmarksnumbered, bookmarksopen=true, bookmarksopenlevel=1,%
    hypertexnames=true, pdfhighlight=/O, %nesting=true,%frenchlinks,%
    %urlcolor=tud1b, linkcolor=tud1b, citecolor=tudtud1bccent,
    pdftitle={\fzdTitle, \fzdThesisType, \fzdID},%
    pdfauthor={\fzdName, FZD, TU Darmstadt},%
    pdfsubject={},%
    pdfkeywords={},%
    pdfcreator={},%
    pdfproducer={}%
}

%============================================
% Setup of the title page (do not change)
%============================================
\title{\fzdTitle}
\subtitle{\fzdThesisType \fzdID}
\subsubtitle{\fzdName \\ \fzdBetreuer}
\setinstitutionlogo[height]{gfx/fzd_logo.png}
\institution{\raggedleft Fachbereich Maschinenbau\\[\baselineskip]%
	Fachgebiet Fahrzeugtechnik und Dynamik \\
	Prof. Dr. rer. nat. Hermann Winner%
}

%============================================
% Setup of the title backside (do not change)
%============================================
\lowertitleback{%
	Technische Universit\"{a}t Darmstadt \\%
	Fachbereich Maschinenbau\\[\baselineskip]%
	Fachgebiet Fahrzeugtechnik und Dynamik\\%
	Prof. Dr. rer. nat. Hermann Winner%
}

\uppertitleback{%
	\fzdTitle \\%
	\fzdThesisType \\%
	\fzdID \\[\baselineskip]%
	Eingereicht von \fzdName \\%
	Tag der Einreichung: \fzdSubmissionDate \\[\baselineskip]%
	\fzdGutachter \\%
	\fzdBetreuer \\%
	\fzdExternerBetreuer%
}
	
%======================================================
% MAIN DOCUMENT STARTS HERE
%======================================================
\begin{document}
%======================================================
	% The front matter
	%======================================================
	\pagenumbering{roman}
	\frenchspacing
	\raggedbottom
	\selectlanguage{ngerman} % american ngerman
	\maketitle
	
	\chapter*{Ehrenw\"{o}rtliche Erkl\"{a}rung}
	
	\newpage	
	
	\section*{Kurzzusammenfassung}
Das Schützen der Umwelt und der Gesundheit der 		Mitmenschen spielt in der Automobilentwicklung eine immer größere Rolle. Doch nicht nur die Verbrennungsstoffe der Motoren oder die Abriebpartikel der Reifen führen zu einer Umweltbelastung. Immer interessanter werden temporäre Partikelquellen wie unter anderem die Abriebemissionen der Bremsen eines Fahrzeugs. Da die Eigenschaften und Zusammensetzung dieser Emissionen jedoch noch weitgehend unerforscht und unbekannt sind, müssen diese erst erforscht werden. Zur Messung und Analyse von Emissionspartikeln verwendet man Messgeräte, welche Partikelanteile in der  vorhandenen Luft messen. Eine Materialanalyse der aufgefangenen Partikel gibt Aufschluss über die chemische Zusammensetzung der Partikel. So kann man aufgrund der Kenntnisse über die Emissionen geeignete Maßnahmen zu Umweltschutz und Gesundheitsschutz treffen.
	\\
	\\
Innerhalb des Advanced Design Projects (ADP) wurde ein System zur Aufbereitung und Weiterleitung einer Partikelströmung entwickelt und eine theoretische Konstruktion ausgearbeitet um die Qualität von Partikelmessgeräten im Anwendungsgebiet Bremsen zu überprüfen. Dafür wurden in der ersten Phase der Projektes ein Zeitplan erstellt, der Stand der Technik und der Forschung im Bereich Strömungslehre und Aerosolerzeugung analysiert und eine Anforderungsliste erarbeitet. Da die Problemstellung mehrere Faktoren betraf, wurden mehrere morphologische Kästen erstellt um das Problem in kleinere Arbeitsschritte zu unterteilen und Lösungen für diese zu finden. Dadurch konnte in den letzten Schritten des Projektes die theoretische Konstruktion mehrere Konzepte entwickelt werden und eine Konzeptentscheidung anhand der Anforderungsliste getroffen werden.
	\\
	\\
Anschließend wurde das ausgewählte Konzept mathematisch und analytisch ausgewertet und evaluiert, um die Garantie des Systems zu gewährleisten.
	%======================================================
	% The main matter (insert your contents here)
	%======================================================
	\selectlanguage{american} % american ngerman
	\tableofcontents
	%\listoffigures
	%\listoftables
	\cleardoublepage
	\pagenumbering{arabic}
    %\begin{spacing}{2} %more space for annotations
    %*****************************************
\chapter{Einf\"{u}hrung}
%*****************************************
\section{Motivation}
\section{Voraussetzungen}
%\input{themes/intro}

%*****************************************
\chapter{Technische Grundlagen}\label{ch:foundations}
%*****************************************
\section{Str\"{o}mungsmechanik}
\subsection{Rohrstr\"{o}mungen}
\subsection{Laminare/Turbulente Str\"{o}mungen}
\subsection{Reynoldszahl}
\subsection{Prandtlzahl}

\section{Feinraumtechnik}
\subsection{Eigenschaften von Partikeln}
\subsection{Partikelmessverfahren}
\subsection{Aerosole}

\section{Mechanische Grundlagen}
\subsection{Ventile}
\subsection{Luftfiltersysteme}

%\input{themes/foundations}

%************************************************
\chapter{Versuchsplattform}\label{ch:platform}
%************************************************
\section{Partikelmessger\"{a}te}
\subsection{APS-3321}
\subsection{FMPS-3091}
\subsection{OPC-N2}

\section{Simulation}
\subsection{SpaceClaim}
\subsection{Fluent}

\section{Partikelgeneratoren}
\subsection{Dinhs Megazerst\"{a}uber 2000}
\subsection{Alexs Partikelhack 500M}
\subsection{Der Gia 6000}


%\input{themes/platform}

%************************************************
\chapter{Anforderungen an unsere Arbeit}\label{ch:platform}
%************************************************
%\input{themes/platform}

%************************************************
\chapter{Analyse verschiedener Industrie-Aerosole}\label{ch:work}
%************************************************
\section{Eigenschaften von Aerosolen}
\section{Di-Ethyl-Hexyl-Sebacat (DEHS)}
\section{Di-N-Octylphtalat (DOP)}
\section{Emery 3004 (PAO-4)}
\section{Poly Styrene Latex Spheres (PSL)}
\section{Auswertung der Analyse}
\subsection{Anforderungsvergleich der Aerosole}
\subsection{Auswahl eines Aerosols}

%\section{Stand der Technik}
W\"{a}hrend der Einarbeitungszeit sind wir durch Literaturrecherche auf einige Dissertationen, Abschlussarbeiten und Richtlinien aufmerksam geworden, welche vergleichbare Versuchsaufbauten beinhalten, weche haupts\"{a}chlich zur Kalibrierung von Partikelmessger\"{a}ten dienen.
\\
\\
Die Masterarbeit \ref{xyz} hatte das Ziel das Ansprechverhalten einer Constant Volume Sampling (CVS) Anlage zu optimieren. Der daf\"{u}r verwendete Versuchsaufbau ist in Abbildung \ref{aufbau_graz} abgebildet. Das hierf\"{u}r verwendete Aerosol wurde aus den Abgasen eines Versuchsfahzeugs entnommen und zusammen mit Verd\"{u}nnungsluft in einen CVS Tunnel geleitet, in welchem sich ein Partikelmesssystem befindet. Die Partikeltestsignale wurden hierbei mit Hilfe eines Umschalters erzeugt, bei dem zwischen einem Aerosolstrom und Luftstrom geschaltet werden kann (Abbildung \ref{umschalter_graz}). Diese M\"{o}glichkeit der Generierung eines Partikeltestsignals wurde in die Erarbeitung von unseren Konzepten miteinbezogen. Zudem k\"{o}nnte der CVS Tunnel f\"{u}r nachfolgende Arbeiten, welche sich ausschlie{\ss}lich mit den Str\"{o}mungen der Aerosole besch\"{a}ftigen w\"{u}rde, von Interesse sein.
\\
\begin{figure}
	\centering
	\caption{Pr\"{u}fstand der TU Graz}
	\label{aufbau_graz}
\end{figure}

\begin{figure}
	\centering
	\caption{Umschaltvorgang des Pr\"{u}fstands der TU Graz}
	\label{umschalter_graz}
\end{figure}
Auf direkte Nachfrage hin beim Hersteller der Partikelmessger\"{a}te, wie die angegebenen Werte in den technischen Dokumentationen verifiziert wurden, wiesen diese lediglich darauf hin, dass die Partikelmessger\"{a}te mit Hilfe von Aerosolgeneratoren kalibriert werden. Dies entspricht der Standardvorgehensweise zur Kalibrierung solcher Messger\"{a}te, wie sie in den VDI Richtlinien zu finden sind. Aus diesen liesen sich Kenngr\"{o}{\ss}en und Richtlinien an die zu verwendeten Aerosolgeneratoren ableiten, welche in der Reinraumtechnik Verwendung finden.
\\
\\
Uns sind letzendlich keine Versuchsaufbauten bekannt, welche ausschlie{\ss}lich zur Identifizierung der Zeitkonstante eines Partikelmessger\"{a}ts dienen. Zudem lies sich die Zeitkonstante aus keinem der uns vorliegenden Handb\"{u}cher entnehmen die darauf hindeuten w\"{u}rden, dass es ein existierendes Verfahren gibt diese zu messen. 

%************************************************
\chapter{Konzepte f\"{u}r den Versuchsaufbau}\label{ch:implementation}
\section{Konzept 1}
\subsection{Aufbau}
\subsection{•}
\section{Konzept 2}
\section{Konzept 3}
\section{Konzept 4}
\section{Konzept 5}
%************************************************
%\input{themes/implementation}

%************************************************
\chapter{Simulationsergebnisse}\label{ch:results}
%************************************************
\section{•}
%\input{themes/results}

%************************************************
\chapter{Auswertung der Konzepte}\label{ch:conclusion}
%************************************************
%%TODO

%************************************************
\addcontentsline{toc}{chapter}{\listfigurename}
%\listoffigures

    %\end{spacing} 


	
	%======================================================
	% The back matter
	%======================================================
	%\cleardoublepage
	\refstepcounter{dummy}
	%\addcontentsline{toc}{chapter}{\bibname}
	%\bibliographystyle{plainnat} % <--- layout of the bib

\end{document}
%======================================================
%====================================================== 