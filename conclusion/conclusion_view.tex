\section{Fazit und Ausblick}
F\"{u}r zuk\"{u}nftige Arbeiten gilt es auf jeden Fall, den Versuchsaufbau mit Hilfe des Konstruktionsplans baulich umzusetzen.
\\\\
Unsere Annahme \"{u}ber das laminare Verhalten des Aerosolstroms beruht auf unserer Wahl, technisch glatte Materialien f\"{u}r die Aerosolleitungen zu verwenden und die Verbindungen und Leitungen m\"{o}glichst str\"{o}mungsg\"{u}nstig auszulegen. Eine Str\"{o}mungssimulation kann hierbei mehr Aufschl\"{u}sse \"{u}ber das Str\"{o}mungsverhalten geben.
\\\\
Zus\"{a}tzlich lie{\ss}e sich die Schaltzeit durch den Einsatz einer Federauslegung statt eines Elektromotors verbessern, da so sehr viel h\"{o}here Drehmomente erzeugt werden k\"{o}nnen als mit einem Elektromotor. Dies konnte von uns aufgrund der technischen Komplexit\"{a}t nicht umgesetzt werden.
\\\\
Als Fazit l\"{a}sst sich sagen, dass eine robuste Versuchseinrichtung entworfen wurde um das zeitliche dynamische Verhalten von Partikelmessger\"{a}ten zu bestimmen. Daf\"{u}r wurden mit Hilfe der VDI Richtlinien mehrere Konzepte entwickelt und an Hand von bestimmten Anforderungen ausgewertet. Daraufhin wurde das beste Konzept ausgew\"{a}hlt und eine detaillierte technische Konstruktion entworfen. Es wurden Probleme erl\"{a}utert, welche f\"{u}r einen Versuchsaufbau dieser Art gel\"{o}st werden mussten und es wurde ein Ausblick gegeben, wie die vorgestellte Einrichtung noch verbessert und erweitert werden k\"{o}nnte.