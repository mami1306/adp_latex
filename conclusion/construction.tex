\section{Konstruktion der Konzeptauswahl}

\subsection{Konstruktion der Bauteile}
Ausgehend von der Konzeptauswertung werden für die Aerosolquelle drei handelsübliche Kerzen verwendet. Nach der Studie von Pagels et al.\cite{candle} liegt der vollständige Partikelgrößenbereich von Kerzen zwischen \(16nm - 1000nm\). Partikelgrößen von über \(100nm\) Durchmesser entstehen bei stabil brennenden Kerzen über Partikelwachstum. Ursache ist eine Koagulation über die Zeit. Flackernde Kerzen geben zusätzlich Rußpartikel ab, sodass die Partikelgröße auf über \(270nm\) steigt.
\\\\
Da das Aerosol in der Versuchseinrichtung direkt nach der Entstehung entnommen wird und der Partikelgrößenbereich zur Abdeckung aller Messgeräte über \(300nm\) liegen muss, wird eine flackernde Kerzenflamme angestrebt. Um einen Luftzug nachzuahmen, wird ein handelsüblicher Ventilator auf der niedrigsten Stufe hinter den Kerzen aufgestellt. Die Kerzen befinden sich in einem hohen offenen Behälter, in dem sich auf der Höhe des Aerosols zwei Durchlässe befinden. Ein Durchlass wird vom Eingangsschlauch des Rotationsverdünners besetzt, während der zweite Durchlass für den erzeugten Luftstrom des Ventilators vorhanden ist.
\\\\
Um die zu hohe Partikelanzahlkonzentration des Aerosols zu regulieren, wird ein Rotationsverdünner benötigt. Für die Versuchseinrichtung wird der Rotationsverdünner und Thermokonditionierer 379020A-30 von TSI verwendet. Das Verdünnungsverhältnis ist einstellbar und somit auch kompatibel für alle Messgeräte. Da der Rotationsverdünner nur einen maximalen Volumenstrom von \(5L/min\) ausgibt, erhöht der zugeschaltete Thermokonditionierer den abgegebenen Volumenstrom. Weiterhin reguliert dieser die Temperatur des Trägergases, da bei fortgeschrittener Versuchszeit eine kontinuierliche Temperatursteigerung aufgrund der Kerze entsteht.
\\\\
Um während der Totzeit den Aerosolstrom zu konditionieren, muss dieser in Bewegung versetzt werden, indem er mit Hilfe eines externen Verdichters mit dem gleichen Volumenstrom wie die des verwendeten Messgerätes angesaugt wird. Da der Versuchsaufbau mehrere Partikelmessgeräte abdecken soll, muss der Volumenstrom des Verdichters einstellbar sein, sodass er den Aerosolstrom für jedes Messgerät konditionieren kann. Aus der Anforderungsliste lässt sich entnehmen, welcher Bereich dafür abgedeckt werden muss. Ein Verdichter, der unsere Anforderungen erfüllt, ist die \textit{Capex V2 Pumpe} der Firma Charles Austen, welche mit Gleichstrom betrieben wird. Für den Betrieb wird der Verdichter an ein stufenloses Labornetzgerät angeschlossen, was es einem ermöglicht durch Variation der Spannung auch den Volumenstrom so anzupassen, dass er dem des verwendeten Messgerätes entspricht. Der maximale Volumenstrom den der Verdichter leisten kann, liegt bei \(17 L/min\). Die Ein- und Auslässe haben jeweils einen Außendurchmesser von \(8mm\), was in der Größenordnung der von uns verwendeten Schläuche liegt.
\\\\
Die Luft, welche das Messgerät während der Nullzeit ansaugt, muss frei von Partikeln sein, um beim Wechsel zum Aerosolstrom einen Partikelsprung zu ermöglichen. Dafür muss die Luft mit Hilfe eines HEPA Filters gereinigt werden, bevor sie vom Messgerät angesaugt wird. Für diesen Zweck eignet sich der \textit{HS-Mikroseal JG-S Patronenfilter} der Firma HS-Luftfilterbau, da dieser zum Einbau in Rohrleitungssysteme geeignet ist, welche über kleine Baugrößen mit moderatem Volumenstrom verfügen. Der Nennvolumenstrom liegt bei bis zu \(22m^3 / h\) und verursacht einen maximalen Druckunterschied von 

\subsection{Kosten und Evaluation}