\section{Partikelgeneratoren}
F\"{u}r die Kalibrierung und Pr\"{u}fung von Partikelmessger\"{a}ten werden oft industrielle Partikelgeneratoren verwendet. Diese Generatoren arbeiten auf Basis von Vernebelung oder Verdampfung von Fl\"{u}ssigkeiten. Die Fl\"{u}ssigkeiten werden unter hohem Druck in einer D\"{u}se vernebelt und verspr\"{u}ht oder sie werden durch hohe Temperaturen verdampft. Dabei entsteht ein Aerosol, welches Partikel der verwendeten Fl\"{u}ssigkeit enth\"{a}lt. Um die Messger\"{a}te gezielt kalibrieren zu k\"{o}nnen, kann bei den Generatoren die gew\"{u}nschte Dichte sowie die Gr\"{o}{\ss}e der Partikel innerhalb des Aerosols reguliert werden. So hat man zu jedem Zeitpunkt Kenntnis \"{u}ber den Aufbau des Partikelstroms. Da sich in dieser Arbeit auf die Verwendung von DEHS Aerosol beschr\"{a}nkt wird, werden nur die beiden Generatoren in Erw\"{a}gung gezogen, welche DEHS unterst\"{u}tzen. Dies sind auch die beiden Ger\"{a}te, welche in der Industrie am meisten vorkommen und verwendet werden.

\subsection{Topas ATM 220}
Die Ger\"{a}tereihe ATM von Topas wird schon seit langer Zeit speziell im Bereich der Reinraum- und Filtertestanwendung verwendet. Das Kernst\"{u}ck des Generators ist ein Edelstahlatomizer, eine nach dem Injektorprinzip arbeitende Zweistoffd\"{u}se. Dieser sorgt f\"{u}r die R\"{u}ckf\"{u}hrung der bei der Verd\"{u}sung entstehenden gro{\ss}en Tropfen, welches die Partikelkonzentration nur unwesentlich beeinflusst. Die f\"{u}r die Verd\"{u}sung ben\"{o}tigte Druckluft wird durch einen HEPA-Filter gereinigt. Die D\"{u}se ist direkt in die Fl\"{u}ssigkeit getaucht, um auch sehr geringe Massestr\"{o}me reproduzierbar einstellen zu k\"{o}nnen.\\
Der Topas ATM hat eine Partikelmengenspannbreite  von \(10^5\) bis \(10^8\) Partikel pro \(cm^3\), bei einer Gr\"{o}{\ss}enverteilung von \(0,1 - 0,5 \mu m\).\cite{topas}

\subsection{Palas 2000H}
Der Palas 2000H wird nicht nur f\"{u}r die Kalibrierung von Partikelmessger\"{a}ten sondern auch zum Testen von Filtersystemen verwendet. Im Gegensatz zum ATM arbeitet der Palas mit Hilfe von Kondensation und Verdampfung. So kann durch Druck- und Temperatur\"{a}nderungen die Partikelgr\"{o}{\ss}enverteilung reguliert werden. Der gro{\ss}e Vorteil des Palas 2000H ist, dass er \"{u}ber Ethernet von einem Computer aus gesteuert werden kann und somit direkt in ein Evaluationsprogramm f\"{u}r den Versuchsaufbau eingebunden werden kann. Er ist wesentlich nutzerfreundlicher als der f\"{u}r den Fachmann entwickelte ATM, erzeugt aber wesentlich gr\"{o}{\ss}ere Partikel, die f\"{u}r den Versuchsaufbau gar nicht ben\"{o}tigt werden. Au{\ss}erdem arbeitet der Generator mit einer Temperaturregelung und erhitzt das erzeugte Aerosol auf diesem Weg, welches sich nachteilig auf das Verhalten der Str\"{o}mung auswirken kann. Die Partikelmenge kann nicht eingestellt werden und liegt bei \(10^6\) Partikeln\(/cm^3\) bei einer regulierbaren Gr\"{o}{\ss}enverteilung von \(0,2 \mu m - 100\mu m\).\cite{palas}