\section{Di-Ethyl-Hexyl-Sebacat (DEHS)}
Di-Ethyl-Hexyl-Sebacat (DEHS) ist ein Gemisch aus drei isomeren chemischen Verbindungen aus der Gruppe der Carbons\"{a}ureester. Erzeugt wird die Fl\"{u}ssigkeit durch einen S\"{a}urekatalysator oder unter hohem Druck. DEHS ist brennbar, aber schwer entz\"{u}ndbar, farb- und geruchlos und praktisch unl\"{o}slich in Wasser. Verwendung findet es als Weichmacher, Schmierstoff und zur Erzeugung stabiler Pr\"{u}faerosole. DEHS ist aktuell das g\"{a}ngstige Partikelmaterial in Industrie und Forschung, da es gesundheitlich unbedenklich ist und in einem gro{\ss}en Temperaturbereich stabil ist. Je nach Generator ist eine gro{\ss}e Spannbreit an Partikelgr\"{o}{\ss}en m\"{o}glich.
\begin{itemize}
\item Aggregatzustand: fl\"{u}ssig
\item Dichte: \(0,91 g/cm^3\)
\item Schmelzpunkt: \(-55^\circ\text{C}\)
\item Siedepunkt: \(256^\circ\text{C}\)
\item Partikelgr\"{o}{\ss}e: Abh\"{a}ngig vom Generator
\end{itemize}