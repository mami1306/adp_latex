\section{Auswahl eines Aerosols}
Alle vier vorgestellten Fl\"{u}ssigkeiten eignen sich zur Herstellung von Pr\"{u}faerosolen. Di-N-Octylphtalat (DOP) entf\"{a}llt in dieser Arbeit als Pr\"{u}faerosol , da es gesundheitsgef\"{a}hrdend ist eine vollst\"{a}ndige Dichtheit der Versuchseinrichtung nicht gew\"{a}hrleistet werden kann.\\
Der Versuchsaufbau soll bei Raumtemperatur (\(15^\circ\text{C} - 25^\circ\text{C}\)) verwendet werden. In diesem Temperaturbereich verhalten sich alle Aerosole stabil. Die Partikel bleiben in der erzeugten Gr\"{o}{\ss}e erhalten und verbinden sich nicht bei der oberen Temperaturgrenze. Auch zerfallen sie nicht im unteren Bereich der Betriebstemperatur.\\
Die Dichte ist ein weiterer Entscheidungsfaktor, da sie sich direkt auf die Reynoldszahl und somit das Str\"{o}mungsverhalten auswirkt. Angestrebt wird eine m\"{o}glichst geringe Reynoldszahl, weshalb auch die Dichte gering bleiben sollte. Somit kann man Poly Styrene Latex auch ausschlie{\ss}en, da eine Dichte gr\"{o}{\ss}er als 1 sich stark auf die Reynoldszahl auswirkt. Weiterhin beschr\"{a}nkt die Verwendung von PSL die Partikelgr\"{o}{\ss}e. Das Gr\"{o}{\ss}enintervall ist zwar sehr gro{\ss}, allerdings haben die anderen Aerosole keine Beschr\"{a}nkungen, weshalb sie besser geeignet sind.\\
Letztendlich entscheidend sind die Anschaffungskosten der beiden Pr\"{u}faerosole. DEHS liegt preislich unter Emery 3004, weshalb DEHS f\"{u}r die Partikelgeneratoren in den weiteren Konzepten verwendet wird.