\section{Di-N-Octylphtalat (DOP)}
Di-N-Octylphtalat (DOP) ist eine organische Verbindung aus der Gruppe Phtalate. Es ist eine farblose, geruchlose und \"{o}lige Fl\"{u}ssigkeit. DOP wird durch die Reaktion von Phtalats\"{a}ureanhydrid und Octanol in Gegenwart eines Katalysators gewonnen. Wie DEHS ist DOP in Wasser kaum l\"{o}slich. Auch in der Anwendung deckt DOP dasselbe Gebiet ab, dar\"{u}ber hinaus findet es allerdings noch Anwendung im medizinischen Bereich und in der Sprengstoffindustrie. DOP z\"{a}hlt als stark krebserregend, weshalb es nicht mehr als Weichmacher f\"{u}r Verbrauchsgegenst\"{a}nde verwendet werden darf. Auch hier wird eine gro{\ss}e Spannbreite an Partikelgr\"{o}{\ss}en abgedeckt, je nachdem welcher Partikelgenerator verwendet wird.
\begin{itemize}
\item Aggregatzustand: fl\"{u}ssig
\item Dichte: \(0,98 g/cm^3\)
\item Schmelzpunkt: \(-49^\circ\text{C}\)
\item Siedepunkt: \(385^\circ\text{C}\)
\item Partikelgr\"{o}{\ss}e: Abh\"{a}ngig vom Generator
\end{itemize}