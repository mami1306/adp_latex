\section{Unkonventionelle Aerosolquellen}
Neben industriellen Partikelgeneratoren, wurden f\"{u}r die Arbeit noch andere, unkonventionelle Aerosolquellen in Betracht gezogen und f\"{u}r die Konzeptentwicklung verwendet. Grund daf\"{u}r ist einmal die Einfachheit solcher Quellen und die geringen Anschaffungskosten. Weiterhin ist es f\"{u}r die reine Erzeugung eines Partikelsprungs gar nicht n\"{o}tig eine hoch aufgel\"{o}ste und genaue Partikeldichte zu erreichen, sondern lediglich einen klaren Partikelanstieg zu generieren. In diesem Bereich k\"{o}nnten solche unkonventionellen Mittel bereits ausreichen, weshalb sie in die Konzeptentwicklung mit eingeschlossen wurden.\\
In Betracht gezogen wurden Kerzen, kleine Liquidverdampfer aus dem Modellbau oder einer elektronischen Zigarette und partikelbelastete Gro{\ss}r\"{a}ume. Eine Kerze erzeugt bei Windstille einen konstanten Ru{\ss}partikelstrom \"{u}ber einen langen Zeitraum und eignet sich so f\"{u}r einen Versuch mit mehreren Wiederholungen. Ein Nachteil ist, dass die Kerze nicht gut steuerbar ist, anders als bei einem elektronischen Modellverdampfer. Bei einem elektronischen Verdampfer muss der Strom allerdings jedes mal neu aufgebaut werden, w\"{a}hrend die Kerze direkt einen Konstanten Strom liefert und haben somit eine h\"{u}here Totzeit. Ein anderes Konzept war den Versuchsaufbau in einem konstant mit Partikeln belasteten Gro{\ss}raum einzusetzen, wie zum Beispiel in Werkst\"{a}tten oder in Sporthallen. Bei solch einem Einsatz muss darauf geachtet werden, dass der Frischluftstrom in der Nullphase des Systems mit guten Filtern erzeugt wird, da sonst der Partikelsprung nicht eindeutig genug zu erkennen ist.\\
Alles in allem ist es m\"{o}glich auch mit unkonventionellen Mitteln einen Partikelsprung zu generieren, wenn man den Versuchsaufbau entsprechend anpasst. In den Konzeptideen sind die m\"{o}glichen Anwendungen dargestellt.