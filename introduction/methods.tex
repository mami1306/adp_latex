\section{Methodik}
Zu Beginn muss gesagt werden, dass die Aufgaben nie in der Gruppe bearbeitet wurden, sondern in regelm\"{a}{\ss}igen Treffen der Gruppe besprochen und aufgeteilt wurden. Dabei wurden Milestones gesetzt, welche es termingerecht zu erreichen galt. Die erledigten Aufgaben wurden dann diskutiert und verfeinert. Anschlie{\ss}end wurden die neuen Aufgaben festgelegt, aufgeteilt und neue Milestones zu den Aufgaben gesetzt. Im Folgenden werden wir nicht weiter auf die Aufteilung der Aufgaben eingehen, da es f\"{u}r das Projekt keine Rolle spielt wer die Aufgaben erledigt hat, sondern es von gr\"{o}{\ss}erer Relevanz ist, wie die einzelnen Aufgaben angegangen und bearbeitet wurden.
\begin{enumerate}
\item \textbf{Einarbeitung in die verschiedenen Thematiken}:\\
F\"{u}r die Einarbeitung wurde haupts\"{a}chlich Literaturrecherche betrieben. Dabei wurde Literatur aus der Bibliothek der TU Darmstadt verwendet, welche von verschiedenen Professoren und wissenschaftlichen Mitarbeitern empfohlen wurde. Eine weitere Quelle waren die wissenschaftlichen Ver\"{o}ffentlichen der Arbeiten am Windkanal in Darmstadt und der Fachgebietes Fahrzeugtechnik an der TU Darmstadt. Auch das Internet und digitale Archiv der ULB Darmstadt wurden als Quellen f\"{u}r die Einarbeitung genutzt.
\item \textbf{Ausarbeitung einer Anforderungsliste}:\\
F\"{u}r die Erstellung einer Anforderungsliste wurde die gesamte geplante Versuchseinrichtung in verschiedene Teile unterteilt. Dabei gab es drei wichtige Aspekte der Einrichtung zu beachten. Welche Anforderungen werden aufgrund der verwendeten Messger\"{a}te vorgegeben, welche Anforderungen ergeben sich durch die Auswahl eines passenden Partikelsubstrates und welche Anforderungen entstehen f\"{u}r die eigentlichen Bauteile der Versuchseinrichtung. Dabei wurden einmal physikalische, wie auch chemische Anforderungen an Material und Aufbau gestellt, sowie allgemeine Anforderungen an das gesamte Versuchsergebnis.
\item \textbf{Entwicklung von Versuchseinrichtungskonzepten}:\\
F\"{u}r die Entwicklung von Konzepten f\"{u}r verschiedene Versuchseinrichtungen wurden f\"{u}r alle ben\"{o}tigten Elemente der Einrichtung morphologische K\"{a}sten erstellt. Dabei wurde sich bei der L\"{o}sungsfindung an die VDI-Richtlinien \textit{2222} und \textit{2225}.\\
VDI-Richtlinie \textit{2222} beschreibt das methodische Vorgehen bei der Konstruktion von Konzepten, bei dem Probleml\"{o}sungen und die genaue Aufgabenstellung f\"{u}r dritte nachvollziehbar hergeleitet werden. Dadurch ergeben sich Vorteile in der Wiederverwendung von Teill\"{o}sungen. Die VDI-Richtlinie \textit{2225} beschreibt das Entwickeln von vollst\"{a}ndig technischen Produkten, welche in technischer Funktionalit\"{a}t und Wirtschaftlichkeit lange konkurrenzf\"{a}hig bleiben.\\
Die ben\"{o}tigten Elemente ergaben sich aus der Aufgabenstellung und der Anforderungsliste. Durch Brainstorming wurden die morphologischen K\"{a}sten mit verschiedenen Ideen zu einzelnen Bauteilen gef\"{u}llt. Aus den Einzelteilen entstanden in mehreren Durchl\"{a}ufen verschiedene Konzeptideen, welche skizziert und kurz beschrieben wurden. Dabei unterschieden sich die Konzepte in einem wichtigen Punkt, n\"{a}mlich der Art, wie der Partikelsprung erzeugt wird. Die Konzepte wurden nach ihrer Entstehung mit wissenschaftlichen Mitarbeitern besprochen und verfeinert, wobei manche Konzepte in diesem Schritt bereits von der Liste gestrichen wurden.
\item \textbf{Vergleich der Konzepte}:\\
Mit der Anforderungsliste als Referenz wurden die \"{u}brig gebliebenen Konzepte verglichen. Dabei haben die Konzepte Punkte in jeder Anforderung bekommen. Je nach Wichtigkeit der Anforderung wogen hier die Punktzahlen mehr oder weniger. Am Ende wurde das Konzept ausgew\"{a}hlt welches, den Anforderungen am besten standhalten konnte.
\item \textbf{Auswahl und Konstruktion eines Konzeptes}:\\
Das ausgew\"{a}hlte Konzept wurde dann als detaillierte Skizze angefertigt. Nun wurden auch erste konkrete \"{U}berlegungen zu Ma{\ss}en und Materialien gemacht um eine endg\"{u}ltige Konstruktion zu entwerfen. Die Versuchseinrichtung wurde in Einzelteile zerlegt und es wurden konkrete Bauteile gefunden oder entwickelt um das bisher theoretische Konzept zu verwirklichen. Ma{\ss}e, Materialien, Herkunft und Preis der Teile wurden ermittelt und festgelegt, sodass ein fertiger Bauplan des Konzeptes vorlag.
\end{enumerate}