\section{Konkretisierung der Aufgabenstellung}
Am Fachgebiet Fahrzeugtechnik der TU Darmstadt wird in Zusammenarbeit mit der Industrie am Thema Partikelemissionen von Pkw-Scheibenbremsen geforscht. Zum Vergleich partikelemissionsarmer Bremsstrategien wird auf Grundlage von Messungen ein Partikelemissionsmodell erarbeitet. Im Rahmen der Systemidentifikation sind Informationen \"{u}ber die zeitliche Aufl\"{o}sung der verwendeten Partikelmessger\"{a}te notwendig.
\\\\
Ziel des Advanced Design Projects ist es, ein Konzept f\"{u}r eine Versuchseinrichtung zu entwickeln und dieses konstruktiv umzusetzen. Dabei ist es das Ziel die Generierung eines Partikeltestsignals in Form einer Sprungfunktion zu erm\"{o}glichen und so ein Versuchswerkzeug zur Identifizierung der zeitlichen Aufl\"{o}sung der Partikelmessger\"{a}te zu entwickeln.
\\\\
Im Einzelnen teilt sich das folgende ADP in f\"{u}nf Arbeitsschritte auf. Es beginnt mit der Einarbeitung in die Thematiken Bremspartikelemissionen, Partikelmesstechnik sowie Systemidentifikation mit Hilfe von Testsignalen. Nach der Einarbeitung in die f\"{u}r die Arbeit wichtigen Themen soll eine Ausarbeitung einer Anforderungsliste f\"{u}r die Versuchseinrichtung aufgestellt werden. Im dritten Schritt werden Konzepte f\"{u}r die Versuchseinrichtung erarbeitet um diese dann anschlie{\ss}end mit Blick auf die Anforderungsliste zu vergleichen. Am Ende wird das Konzept ausgew\"{a}hlt, welches am besten den Anforderungen entspricht. Der letzte Schritt ist eine detaillierte theoretische Konstruktion der Versuchseinrichtung.