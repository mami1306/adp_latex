\section{Motivation}
Im Bereich der Fahrzeugentwicklung spielen seit Jahren Leistung und Energieverbrauch die gr\"{o}{\ss}te Rolle als Fortschrittskriterium. Durch steigende Belastung der Umwelt und der Gesundheit von Mensch und Tier, sind Umweltschutzfaktoren bei der Entwicklung von Kraftfahrzeugen immer weiter in den Vordergrund ger\"{u}ckt. Immer mehr gesetzliche Richtlinien schr\"{a}nken die Automobilhersteller in ihren Freiheiten ein und zwingen die Entwicklung in eine umweltfreundlichere Richtung. Anfangs zielte man nur auf Verbrennungsprodukte ab, die bei der Verwendung von Otto- und Dieselmotoren entstehen. Sp\"{a}ter kamen Altschrottentsorgung und Reifenabrieb hinzu. 
\\\\
Neben all diesen Faktoren ist auch die Bremse ein Bauteil im Kraftfahrzeug, welches durch Benutzung Emissionen abgibt. Da auch in diesem Teilgebiet gesetzliche Richtlinien zu erwarten sind, konzentriert sich die Forschung aktuell auf die Entstehung solcher Emissionen bei Kraftfahrzeugbremsen. Da Emissionsart und Partikelzusammensetzung der Bremsemissionen aktuell nicht gut erforscht und deshalb unbekannt sind, kann man nicht direkt Ma{\ss}nahmen gegen die Belastung durch Bremsemissionen ergreifen. Um Aufschluss \"{u}ber die Art der entstehenden Partikel zu bekommen, werden Partikelmesssysteme eingesetzt. An diesem Punkt steigt diese Arbeit ein um das zeitliche Messverhalten solcher Systeme zu \"{u}berpr\"{u}fen, damit dieser sp\"{a}ter an der Kraftfahrzeugbremse eingesetzt werden k\"{o}nnen.