\section{Anforderungen}
Um sicherstllen zu k\"{o}nnen, dass das Projektziel bestm\"{o}glicht erreicht wird, wurde eine Anforderungsliste erstellt, mit deren Hilfe die G\"{u}te der Umsetzung des Versuchskonzepts gemessen werden kann. Diese Liste enth\"{a}lt die Anforderungen des Auftraggebers und unseren eigenen, welche sich aus den zu Grunde liegenden technischen Dokumenationen der verwendeten Messger\"{a}te und den VDI Richtlinien f\"{u}r Reinraumtechnik ergeben haben.

Bei den festgelegten Anforderungen war uns eine genaue Quantifizierung der Werte wichtig, wann immer es m\"{o}glich war. Bei den Anforderungen wird zwischen den folgenden Arten unterschieden:

\begin{itemize}
	\item \textbf{Festforderung} (FF): Es muss ein bestimmter Wert aufweisbar sein und Abweichungen davon sind nicht zul\"{a}ssig.
	
	\item \textbf{Bereichsforderung} (BF): Die Anforderung muss einen Wert aufweisen und dieser darf nur innerhalb eines bestimmten Bereich liegen.
	
	\item \textbf{Mindestforderung} (MF): Bei einer Mindestforderung muss die Anforderung mindestens einen angegeben Wert aufweisen.
	
	\item \textbf{Zielforderung} (ZF): Der Wert einer Anforderung muss bei einer ZF m\"{o}glichst nah an einem Zielwert liegen, um gut bewertet werden zu k\"{o}nnen.
	
	\item \textbf{Wunsch} (W): Da es sich um einen Wunsch handelt, ist die Wichtigkeit des Merkmals vergleichsweise gering. Die Anforderung sollte dann allerdings einen bestimmten Wert aufweisen. 
\end{itemize} 

Die Anforderungen lassen sich in die Kategorien \textbf{Forderung} und \textbf{Wunsch} einteilen, besonders wichtig ist hierbei eine genaue Quantifizierung der einzelnen Anforderungen wann immer es m\"{o}glich ist. Forderungen m\"{u}ssen auf jeden Fall von allen Versuchsaufbaukonzepten eingehalten werden, um das Ziel erreichen zu k\"{o}nnen. W\"{u}nsche hingegen k\"{o}nnen, m\"{u}ssen aber nicht von den Versuchsaufbauten erf\"{u}llt werden.

F\"{u}r die Generierung eines Partikeltestsignals in Form einer Sprungfunktion lassen sich die folgenden Anforderungen festlegen, die sich in drei Kategorien unterteilen lassen:
\begin{enumerate}
	\item Aerosol: Anforderungen an das Pr\"{u}faerosol, welches dem zu testenden Messger\"{a}t zugef\"{u}hrt wird
	\item Aerosolquelle-Messger\"{a}t-Schnittstelle: Anforderungen an die Schnittstelle, welche die Aerosolquelle mit dem Messger\"{a}t verbindet
	\item Versuchsaufbau: Anforderungen an den gesamten Versuchsaufbau inklusive der verwendeten Ger\"{a}te
\end{enumerate}

Die Anforderungsliste ist in Tabelle \ref{xyz} zu finden. In den nachfolgenden Abschnitten werden die einzelnen Anforderungen genauer erkl\"{a}rt.

\subsection{Messger\"{a}te}
Bei der Konzeption der Versuchsaufbauten sollte anfangs darauf geachtet werden, dass mit diesen verschiedene Messger\"{a}te unterschiedlicher Hersteller getestet werden k\"{o}nnen. Im Verlauf des Projekts wurde zur Vereinfachung des Konzeptentwurfs nach Absprache mit dem Auftraggeber sich darauf geeinigt, sich auf die Messger\"{a}te \textbf{Fast Mobility Particle Sizer Spectrometer Model 3091} (kurz FMPS 3091) und \textbf{Optical Particle Sizer Model 3330} (kurz OPS 3330) der Firma \textbf{TSI} zu konzentrieren. Einige der quantifizierten Werte sind daher aus den Handb\"{u}chern f\"{u}r die Anforderungen entnommen worden.

Beide Messger\"{a}te saugen sich ihre ben\"{o}tigten Aerosol-Luft-Gemisch Mengen \"{u}ber interne Pumpen an. Beim FMPS 3091 betr\"{a}gt dies insgesamt $50 L/min$, wovon $10 L/min$ auf das Aerosol und $40 L/min$ auf Luft entfallen, w\"{a}hrend der OPS 3330 jeweils $1 L/min$ Luft und Aerosol ben\"{o}tigt. Hauptunterschied zwischen den beiden Ger\"{a}ten sind ihre Mechanismen, mit denen Partikel detektiert werden. Der FMPS 3091 misst die elektrostatische Ladung einzelner Partikel, w\"{a}hrend der OPS 3330 ein optisches Verfahren anwendet. Weitere Unterschiede liegen in der Gr\"{o}{\ss}enverteilung der Partikel, bei denen die jeweiligen Messger\"{a}te noch korrekt funktionieren und der Partikelkonzentration. 

Aus diesen und weiteren Werten ergeben sich die nachfolgenden Anforderungen, die die Versuchsaufbauten zu erf\"{u}llen haben. Hierbei wurde immer versucht eine Schnittmenge zwischen den Wertebereichen der Messger\"{a}te zu finden nach der sich die Anforderungen richten um so garantieren zu k\"{o}nnen, dass die Pr\"{u}fst\"{a}nde auf mehrere Messger\"{a}te anwendbar sind.   

\subsection{Aerosole}
%TODO

\subsection{Aerosolquelle-Messger\"{a}t-Schnittstelle}
%TODO

\subsection{Versuchsaufbau}
%TODO