\section{Anforderungen}
Um sicherstllen zu k\"{o}nnen, dass das Projektziel bestm\"{o}glicht erreicht wird, wurde eine Anforderungsliste erstellt, mit deren Hilfe die G\"{u}te der Umsetzung des Versuchskonzepts gemessen werden kann. Diese Liste enth\"{a}lt die Anforderungen des Auftraggebers und unseren eigenen, welche sich aus den zu Grunde liegenden technischen Dokumenationen der verwendeten Messger\"{a}te und den VDI Richtlinien f\"{u}r Reinraumtechnik ergeben haben.

Bei den festgelegten Anforderungen war uns eine genaue Quantifizierung der Werte wichtig, wann immer es m\"{o}glich war. Bei den Anforderungen wird zwischen den folgenden Arten unterschieden:

\begin{itemize}
	\item \textbf{Festforderung} (FF): Es muss ein bestimmter Wert aufweisbar sein und Abweichungen davon sind nicht zul\"{a}ssig.
	
	\item \textbf{Bereichsforderung} (BF): Die Anforderung muss einen Wert aufweisen und dieser darf nur innerhalb eines bestimmten Bereich liegen.
	
	\item \textbf{Mindestforderung} (MF): Bei einer Mindestforderung muss die Anforderung mindestens einen angegeben Wert aufweisen.
	
	\item \textbf{Zielforderung} (ZF): Der Wert einer Anforderung muss bei einer ZF m\"{o}glichst nah an einem Zielwert liegen, um gut bewertet werden zu k\"{o}nnen.
	
	\item \textbf{Wunsch} (W): Da es sich um einen Wunsch handelt, ist die Wichtigkeit des Merkmals vergleichsweise gering. Die Anforderung sollte dann allerdings einen bestimmten Wert aufweisen und wird bei der Bewertung positiv ber\"{u}cksichtigt. 
\end{itemize} 

Die Anforderungen lassen sich in die Kategorien \textbf{Forderung} und \textbf{Wunsch} einteilen, besonders wichtig ist hierbei eine genaue Quantifizierung der einzelnen Anforderungen wann immer es m\"{o}glich ist. Forderungen m\"{u}ssen auf jeden Fall von allen Versuchsaufbaukonzepten eingehalten werden, um das Ziel erreichen zu k\"{o}nnen. W\"{u}nsche hingegen k\"{o}nnen, m\"{u}ssen aber nicht von den Versuchsaufbauten erf\"{u}llt werden.

F\"{u}r die Generierung eines Partikeltestsignals in Form einer Sprungfunktion lassen sich die folgenden Anforderungen festlegen, die sich in drei Kategorien unterteilen lassen:
\begin{enumerate}
	\item Aerosol: Anforderungen an das Pr\"{u}faerosol, welches dem zu testenden Messger\"{a}t zugef\"{u}hrt wird
	\item Aerosolquelle-Messger\"{a}t-Schnittstelle: Anforderungen an die Schnittstelle, welche die Aerosolquelle mit dem Messger\"{a}t verbindet
	\item Versuchsaufbau: Anforderungen an den gesamten Versuchsaufbau inklusive der verwendeten Ger\"{a}te
\end{enumerate}

In den nachfolgenden Abschnitten werden die einzelnen Anforderungen ausf\"{u}hrlich erkl\"{a}rt. Die dazugeh\"{o}rigen Werte sind aus der Anforderungsliste (Tabelle \ref{anforderungen}) zu entnehmen.

\subsection{Messger\"{a}te}
Bei der Konzeption der Versuchsaufbauten sollte anfangs darauf geachtet werden, dass mit diesen verschiedene Messger\"{a}te unterschiedlicher Hersteller getestet werden k\"{o}nnen. Im Verlauf des Projekts wurde zur Vereinfachung des Konzeptentwurfs nach Absprache mit dem Auftraggeber sich darauf geeinigt, sich auf die Messger\"{a}te \textbf{Fast Mobility Particle Sizer Spectrometer Model 3091} (kurz FMPS 3091) und \textbf{Optical Particle Sizer Model 3330} (kurz OPS 3330) der Firma \textbf{TSI} zu konzentrieren. 

Es bestand dennoch weiterhin die Schwierigkeit geeignete Anforderungen zu identifizieren, die die Versuchsaufbauten erf\"{u}llen m\"{u}ssen, um sie zumindest auf die Ger\"{a}te FMPS 3091 und OPS 3330 anwenden zu k\"{o}nnen. Die Gr\"{u}nde hierf\"{u}r liegen in ihren unterschiedlichen Leistungen und Funktionsrinzipien. Abbildung \ref{verfahren} zeigt einen \"{U}berblick \"{u}ber die heutigen verf\"{u}gbaren Messverfahren. Die Wahl eines Verfahrens h\"{a}ngt unter anderem davon ab, welche Partikelgr\"{o}{\ss}en gemssen werden sollen. 

Optische Partikelz\"{a}hler arbeiten nach dem physikalischen Prinzip, dass beleuchtete Partikel das Licht aus seiner urspr\"{u}nglichen Ausbreitungsrichtung ablenken (Abbildung \ref{optisch}). Bei diesen passiert jeder einzelne Partikel ein Messvolumne, in dem er von Licht bestrahlt wird und dann mit Hilfe eines Detektors die Intensit\"{a}t des gestreuten Lichts und die Anzahl der gestreuten Lichtimpulse gemessen. Die Intensit\"{a}t l\"{a}sst Aufschl\"{u}sse \"{u}ber die Partikelgr\"{o}{\ss}e ziehen, bei einem bekannten Volumenstrom und vordefinierter Messdauer l\"{a}sst sich \"{u}ber die Anzahl der Lichtimpulse die Partikelkonzentration bestimmen. Bei moderneren Ger\"{a}ten wie dem OPS 3330 geschieht die optische Messung mit Hilfe eines Laser (Abbildung \ref{laser}). Hierbei kommt zus\"{a}tzlich ein Parabolspiegel hinzu, mit dem das gestreute Licht von einem Detektor gesammelt wird. Bei diesem Messprinzip ist es wichtig, dass jeder Partikel das Messvolumen einzeln durchlaufen muss. Falls zwei oder mehr Partikel das Messvolumen gleichzeitig durchlaufen, werden sie als ein Gro{\ss}es gewertet. Dieser Effekt wird als Koinzidenz bezeichnet. Es existieren zwei Methoden, um diesen Effekt zu vermeiden. Die erste Methode arbeitet aerodynamisch, bei welcher der Aerosolstrahl von einem Mantel aus reiner Luft umgeben ist und mittels einer Ansaugd\"{u}se als d\"{u}nner Strahl austritt (Abbildung \ref{aerodynamisch}).\\
Die zweite Methode arbeitet rein optisch mit Hilfe von zwei Blenden, die den Bereich des zu empfangenden Lichts einschr\"{a}nken (Abbildung \ref{optisch_trennung}).

Elektrische Pertikelz\"{a}hler wie der FMPS 3091 hingegen ben\"{o}tigen keine speziellen Techniken, um Partikel einzeln zu erfassen. Beim FMPS 3091 werden die Partikel positiv aufgeladen und dann mit einem reinen Luftstrom in das Messvolumen gef\"{u}hrt, in welchem sich eine Elektrode befindet und ein elektrisches Feld erzeugt. Elektrometer, welche um die Eletrode angebracht sind und verschieden emfindlich sind, messen die geladenen Partikel (Abbildung \ref{elektrisch}).

Ein wesentlicher Leistungsunterschied zwischen dem FMPS 3091 und OPS 3330 ist das Volumen des Aerosol-Luftgemisches, welches von beiden im Verh\"{a}ltnis 1:4 angesaugt wird. Der OPS arbeitet mit $5 L/min$, w\"{a}rend der FMPS mit $50 L/min$ arbeitet. Ein Teil des Luftgemisches wird dabei vom Ger\"{a}t gefiltert und dazu genutzt, den Aerosolstrom zum Messvolumen zu f\"{u}hren.     

\subsection{Aerosolquelle}
Die Aerosolquelle liefert mit dem Aerosol, den entscheidenden Faktor zum Partikelanzahlkonzentrationssprungs. Aus den Arbeitsbereichsbegrenzungen der Messger\"{a}te sowie allgemeiner Richtlinien zur Aerosolgenerierung ergeben sich folgende Anforderungen:

\begin{itemize}
\item 1. Da die internen Pumpen, weche den Volumenstrom der Messger\"{a}te regulieren, nur innerhal eines bestimmten Druckbereiches korrekt arbeiten, darf der Aerosoldruck an den Einl\"{a}ssen der Messger\"{a}te diesen nicht verlassen. Dies kann durch die Aerosolquelle, Aerosolleitungen und Schaltvorrichtungen beeinflusst werden. 

\item 2. Die Einstellzeit einer Leistungskenngr\"{o}{\ss}e beschreibt das dynamische Verhalten der Aerosolquelle bez\"{u}glich dieser Gr\"{o}{\ss}e bei sprungf\"{o}rmigen \"{A}nderungen eines Einstellparameters. Dabei ist vor allem die Einstellzeit der Partikelanzahlkonzentration für den Versuchsaufbau von Interesse.(Quelle: VDI 3491)

\item 3. Um die Funktionen der Messger\"{a}te und die Messergebnisse nicht zu beeintr\"{a}chtigen, wird als Tr\"{a}gergas gereinigte Luft verwendet, was ebenfalls der Gebrauchsanweisung der verwendeten Messger\"{a}te entspricht. 

\item 4. Die Messger\"{a}te arbeiten mit einem fest vorgegebenen Volumenstrom des Tr\"{a}gergases. Daher muss sichergestellt werden, dass die Aerosolquelle diesen auch bereitstellen kann, da es ansonsten zu einer Fehlfunktion der Messger\"{a}te kommt. 

\item 5. Da die Temperatur des Aerosols den Messprozess und die Eigenschaften des Aerosolstromes beeinflusst, sind durch die Hersteller der Messger\"{a}te feste Temperaturbereiche vorgegeben, in denen sich das Tr\"{a}gergas und das Aerosol befinden d\"{u}rfen.  
\end{itemize}

\subsection{Aerosol}
Folgende Anforderung an das Aerosol ergeben sich aus den Arbeitsbereichsbegrenzungen und aus der statistischen Verwertbarkeit der Messwerte zur Berechnung des zeitlichen Verhaltens der Messger\"{a}te:

\begin{itemize}
\item 6. Die Partikelkonzentration, welche die Aerosolquelle liefert, muss sich innerhalb des Bereiches befinden, der von den Messger\"{a}ten angegeben ist. Eine zu hohe Partikelkonzentration kann bei optischen Messger\"{a}ten zu Koinzidenz f\"{u}hren, bei elektrischen Messger\"{a}ten k\"{o}nnen Partikel durch eine zu geringe Konzentration im Grundrauschen untergehen.

\item 7. Da einige Partikelmaterialien f\"{u}r bestimmte Messverfahren nicht geeignet sind und zus\"{a}tzlich geseundheits- oder umweltgef\"{a}hrdend sein k\"{o}nnen, ist es wichtig ein geeignetes Material hierf\"{u}r zu finden. Da die hier benutzten Messgeräte vor allem für die Feinstaubmessung benutzt werden, bieten sich zum Beispiel feste Materialien für die Aerosole an.

\item 8. Auf die Gr\"{o}{\ss}e der Partikel ist zu achten, da zu kleine von einigen Messger\"{a}ten nicht erkannt werden k\"{o}nnen und zu gro{\ss}e die Funktion beeintr\"{a}chtigen k\"{o}nnen.
\end{itemize}

\subsection{Schaltvorrichtung}
Die Schaltvorrichtung ist das entscheidende Bauteil beim Erzeugen der Sprungfunktion. Dabei sind ein schneller Schaltprozess aber auch ein str\"{o}mungsg\"{u}nstiges Design für die G\"{u}te der Schaltvorrichtung die ausschlaggebenden Kriterien. Auch das Material kann hierbei eine Rolle spielen. Folgende Anforderungen lassen sich definieren:

\begin{itemize}
	\item 9. DIe Zeit, die die Vorrichtung ben\"{o}tigt um von einem Schaltzustand in den anderen zu wechseln, sollte so kurz wie m\"{o}glich sein. Diese Schaltzeit tr\"{a}gt zur Totzeit des Versuchsaufbaus bei. 
	
	\item 10. Hochwinklige Umlenkungen und abrupte Querschnitts\"{a}nderungen k\"{o}nnen zu Partikel- und Druckverlusten f\"{u}hren. Somit k\"{o}nnen sie die G\"{u}te der Partikelfront beeinflussen und sollten deswegen vermieden werden. Au{\ss}erdem kann es bei einem ung\"{u}nstigen Design der Schaltvorrichtung zu Turbulenzen in ihrem Inneren kommen, was die Berechenbarkeit der Totzeit des Aufbaus erheblich erschwert. Die vorhergehenden \"{U}berlegungen gelten auch für die Auslegung der Aerosolleitung.
\end{itemize}

\subsection{Aerosolleitungen}
Auch bei der Leitung des Aerosols, ist auf eine str\"{o}mungsg\"{u}nstige Auslegung zu achten. Dies betrifft die Ausma{\ss}e, das Material, den Zustand der Innenfl\"{a}chen und die r\"{a}umliche F\"{u}hrung der einzelnen Verbindungen. Die folgenden Anforderungen k\"{o}nnen also auch als Designkriterien angesehen werden:

\begin{itemize}
\item 11. Der Weg zwischen der Umschaltvorrichtung und dem Messger\"{a}teeinlass f\"{u}hrt bei gegebenen Volumenstr\"{o}men, abh\"{a}ngig von dem benutzten Messger\"{a}t, zu einer Verz\"{o}gerungszeit, die sich in der Totzeit des Versuchsaufbaus niederschl\"{a}gt. Da das Str\"{o}mungsverhalten in den Verbindungen nur unter Annahme von Vereinfachungen zutrifft, gilt: Umso l\"{a}nger der Weg ist, umso gr\"{o}{\ss}er ist der Rechenfehler bei der Berechnung der Totzeit. Au{\ss}erdem f\"{u}hren Druck- und Partikelverluste auf diesem Weg zu einer Verzerrung der Partikelfront. (VDI 3491)

\item 12. Da eine turbulente Str\"{o}mung zu einem analytisch kaum berechenbaren Verhalten der Str\"{o}mung f\"{u}hrt, ist der Versuchsaufbau so zu gestalten, dass sich die Str\"{o}mung m\"{o}glichst laminar verh\"{a}lt.(VDI 3491) Vor allem am Einlass der Messger\"{a}te, wo es zu einem Querschnitts\"{u}bergang kommen kann, sollten entsprechend gestaltete Adapter daf\"{u}r Sorge tragen.

\item 13. Die Messger\"{a}te besitzen einen Aerosoleinlass in Form eines R\"{o}hrchens mit unterschiedlichen Durchmessern, welches aus ihren Geh\"{a}usen ragt. Je nach Versuchsaufbau k\"{o}nnen Adapter von N\"{o}ten sein, um eine direkte Verbindung zwischen der Schaltvorrichtung und dem Messger\"{a}t zu gew\"{a}hrleisten. Bei der Gestaltung dieser Adapter muss ein Kompromiss zwischen Anforderung 12 und 13 eingegangen werden.
\end{itemize}

\subsection{Versuchsaufbau}
Folgende Anforderungen betreffen den Versuchsaufbau als ganzes:

\begin{itemize}
\item 14. Die minimale Versuchszeit ist wichtig in der Versuchsplanung, da durch die Messaufl\"{o}sung der Messger\"{a}te von einer Sekunde die Anzahl der Messwerte je nach Versuchsaufbau begrenzt ist. Eine Versuchsdurchf\"{u}hrung kann mehrere Abl\"{a}ufe (abwechselnd: gereinigte Luft (zb. 5 s) -- Aerosol(zb. 30s)) haben und innerhalb eines Ablaufs wiederum mehrere Messungen (der Partikelanzahl) stattfinden. Je nach Varianz der Partikelanzahl sind eine bestimmte Anzahl von Messungen und Messabl\"{a}ufen von N\"{o}ten, um ein valides Ergebnis für die Zeitkonstante zu erhalten. So sollte der Aufbau ca. 100 Abl\"{a}ufe mit je 30 Partikelanzahlmessungen (+ 5 Sekunden gereinigte Luft pro Ablauf) mit reproduzierbaren Ergebnissen liefern k\"{o}nnen.

\item 15. Aus wirtschaftlicher Sicht sind die Kosten ein ausschlaggebender Faktor für die G\"{u}te des Aufbaus. Sie setzen sich aus den Materialkosten selbstgestalteter und eventuell selbstgebauter Teile, den Kosten für Zukaufteile (z.B. Filter, Verdichter, Aerosolgeneratoren etc.) und den Kosten f\"{u}r Ressourcen, wie Strom, Aerosol etc., zusammen. Die maximalen Kosten d\"{u}rfen die vorgegebenen Mittel des Fachbereiches nicht \"{u}berschreiten.

\item 16. Ziel des Aufbaus ist die Evaluation der Dynamik von Messger\"{a}ten. Von den f\"{u}nf f\"{u}r diesen Zweck ausgesuchten Messger\"{a}ten, wurden vor allem der FMPS 3091 und der OPS 3330 f\"{u}r geeignet befunden(Fachbereich Fahrzeugdynamik).
Der Versuchsaufbau soll mindestens f\"{u}r diese beiden Messger\"{a}te valide Ergebnisse liefern. Jedes weitere, welches durch den Aufbau erfolgreich getestet werden kann, erh\"{o}ht dessen wirtschaftlichen und wissenschaftlichen Wert. 

\item 17. Der Versuchsaufbau soll m\"{o}glichst unabh\"{a}ngig von Umgebungsbedingungen sein, um eine vielseitige Einsetzbarkeit zu gew\"{a}hrleisten. Zus\"{a}tzlich sollte er ohne gro{\ss}en Mehraufwand den Einsatzort wechseln und dort in Betrieb genommen werden k\"{o}nnen. Da aber Konzepte mit sation\"{a}ren Bauteilen vorstellbar und die verwendeten Messger\"{a}te transportabel sind, ist dieser Punkt der Funktionalit\"{a}t der Einrichtung untergeordnet.

\item 18. Die Pr\"{u}fung des dynamischen Verhaltens der Messger\"{a}te soll durch ein gezieltes Ver\"{a}ndern der Eingangsparameters (Eigenschaften des Eingangsstroms) erfolgen. Dies erfolgt bei den hiesigen Versuchseinrichtungen prim\"{a}r durch die Variation der Partikelkonzentration. Da allerdings die ausgegebenen Daten vieler Messger\"{a}te nur ab einer bestimmten Partikelkonzentration aussagekr\"{a}ftig sind, muss die Anzahlkonzentratinsdifferenz mindestens der spezifischen Partikelkonzentration der Messger\"{a}te entsprechen. 
\end{itemize}

\begin{longtable}{| l | l | l | l | l | l |}
	\caption{Anforderungsliste}\label{anforderungen}\\
	\hline
	Anforderung& Nr. & Art & Wert & Einheit & Quelle\\
	\hline
	Druck am  & 1 & BF & Eingang - Ausgang & $kPa$ & Datenbl\"{a}tter\\
	Messger\"{a}teinlass& & & $70 - 103$ (FMPS) & &\\
	& & & $40 - 103$ (APS) & &\\
	& & & $<0.746$ (OPS) & &\\
	& & & $75-105$ (UCPC) & &\\
	\hline
	Einstellzeit der & 2 & ZF & 0 & $s$ &selbstgew"{a}hlte\\
	Partikelanzahlkonzentration & & & & &Last\\
	der Aerosolquelle & & & & &\\
	\hline
	Tr\"{a}gergas& 3 & FF & & & Datenbl\"{a}tter\\
	- gefilterte Luft & & & & &\\
	\hline
	Volumenstrom & 4 & FF & 10 (FMPS) & $l/min$ & Datenbl\"{a}tter\\
	(Tr\"{a}gergas) & & & $5 \pm 0.2$ (APS)& &\\
	& & & $1 \pm 0.05$ (OPS)& &\\
	& & & $1.5 \pm 0.05$ (UCPC)& &\\
	\hline
	Tr\"{a}gergastemperatur & 5 & BF & $283.15-325.15$(FMPS) & $K$ & Datenbl\"{a}tter\\
	& & & $283.15-313.15$(APS) & &\\
	& & & $273.15-318.15$(OPS) & &\\
	& & & $283.15-308.15$(UCPC) & &\\
	\hline
	Partikelanzahlkonzentration & 6 & BF & $100-10^{7}$ (FMPS)& $n/cm^{3}$ & Datenbl\"{a}tter\\
	& & & $0-1000$ (APS) & &\\
	& & & $0-3000$ (OPS) & &\\
	& & & $0-3\cdot10^{5}$ (UCPC) & &\\
	\hline
	Partikelmaterial (Messger\"{a}t) & 7 & W &  &  & Datenblatt APS\\
	- feste Schwebeteilchen in Luft & & & & &\\
	- nicht fl\"{u}chtige, fl\"{u}ssige Teilchen & & & & &\\
	\hline
	Partikelgr\"{o}{\ss}e & 8 & BF & $5.6-560$ (FMPS) & $nm$ & Datenbl\"{a}tter\\
	& & & $500-20000$ (APS) & &\\
	& & & $300-10000$ (OPS) & &\\
	& & & $2.5-3000$ (UCPC) & &\\
	\hline		
	Schaltzeit & 9 & ZF & 0 & s & selbstgew\"{a}hlte\\
	& & & & & Last\\
	\hline
	Abs\"{a}tze, abrupte Quer- & 10 & ZF & 0 & n & VDI 3491\\
	schnitts\"{a}nderungen, & & & & &\\
	Str\"{o}mungsumlenkungen & & & & &\\
	\hline			
	Weg zwischen Umschaltvorrichtung & 11 & ZF & 0 & m & selbstgew\"{a}hlte\\ 
	und Messger\"{a}teinlass & & & & & Last\\
	\hline
	laminarer Strom & 12 & MF & $< 2300$ & dimensionslos & VDI 3491\\
	& & ZF & $1500$ & &\\
	\hline
	Scnittstelle zum & 13 & FF & 9.5 (FMPS) & $mm$ & Datenbl\"{a}tter\\
	Messger\"{a}t (Inlet) & & & 19 (APS) & &\\
	& & & 6.35 (OPS) & &\\
	& & & 6.4 (UCPC) & &\\
	\hline			
	minimale Versuchszeit& 14 & ZF & 3300 & $s$ & selbstgew\"{a}hlte\\
	& & & & & Last\\
	\hline
	Kosten & 15 & ZF & 0 & Euro & Festlegung \\
	& & BF & $0-1000$ & & durch FZD\\
	\hline
	Anzahl Messger\"{a}te& 16 & MF & $\geq2$ & n & Festlegung\\
	& & & & & durch FZD\\
	\hline
	Transportabel& 17 & W &  &  & Festlegung\\
	& & & & & durch FZD\\
	\hline
	Partikelanzahldifferenz zwischen& 18 & MF & $>100$ (FMPS) & $n/cm^{3}$ & Datenbl\"{a}tter/\\
	gefiltertem und konditioniertem & & & $> 1$ (APS) & & Aufgabenstellung\\
	Luftstrom & & & $> 1$ (OPS)& &\\
	& & & $> 1 (UCPC)$ & &\\
	\hline
\end{longtable}