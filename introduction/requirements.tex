\section{Anforderungen}
Um sicherstllen zu k\"{o}nnen, dass das Projektziel bestm\"{o}glicht erreicht wird, wurde eine Anforderungsliste erstellt, mit deren Hilfe die G\"{u}te der Umsetzung des Versuchskonzepts gemessen werden kann. Diese Liste enth\"{a}lt die Anforderungen des Auftraggebers und unseren eigenen, welche sich aus den zu Grunde liegenden technischen Dokumenationen der verwendeten Messger\"{a}te und den VDI Richtlinien f\"{u}r Reinraumtechnik ergeben haben.

Bei den festgelegten Anforderungen war uns eine genaue Quantifizierung der Werte wichtig, wann immer es m\"{o}glich war. Bei den Anforderungen wird zwischen den folgenden Arten unterschieden:

\begin{itemize}
	\item \textbf{Festforderung} (FF): Es muss ein bestimmter Wert aufweisbar sein und Abweichungen davon sind nicht zul\"{a}ssig.
	
	\item \textbf{Bereichsforderung} (BF): Die Anforderung muss einen Wert aufweisen und dieser darf nur innerhalb eines bestimmten Bereich liegen.
	
	\item \textbf{Mindestforderung} (MF): Bei einer Mindestforderung muss die Anforderung mindestens einen angegeben Wert aufweisen.
	
	\item \textbf{Zielforderung} (ZF): Der Wert einer Anforderung muss bei einer ZF m\"{o}glichst nah an einem Zielwert liegen, um gut bewertet werden zu k\"{o}nnen.
	
	\item \textbf{Wunsch} (W): Da es sich um einen Wunsch handelt, ist die Wichtigkeit des Merkmals vergleichsweise gering. Die Anforderung sollte dann allerdings einen bestimmten Wert aufweisen und wird bei der Bewertung positiv ber\"{u}cksichtigt. 
\end{itemize} 

Die Anforderungen lassen sich in die Kategorien \textbf{Forderung} und \textbf{Wunsch} einteilen, besonders wichtig ist hierbei eine genaue Quantifizierung der einzelnen Anforderungen wann immer es m\"{o}glich ist. Forderungen m\"{u}ssen auf jeden Fall von allen Versuchsaufbaukonzepten eingehalten werden, um das Ziel erreichen zu k\"{o}nnen. W\"{u}nsche hingegen k\"{o}nnen, m\"{u}ssen aber nicht von den Versuchsaufbauten erf\"{u}llt werden.

F\"{u}r die Generierung eines Partikeltestsignals in Form einer Sprungfunktion lassen sich die folgenden Anforderungen festlegen, die sich in drei Kategorien unterteilen lassen:
\begin{enumerate}
	\item Aerosol: Anforderungen an das Pr\"{u}faerosol, welches dem zu testenden Messger\"{a}t zugef\"{u}hrt wird
	\item Aerosolquelle-Messger\"{a}t-Schnittstelle: Anforderungen an die Schnittstelle, welche die Aerosolquelle mit dem Messger\"{a}t verbindet
	\item Versuchsaufbau: Anforderungen an den gesamten Versuchsaufbau inklusive der verwendeten Ger\"{a}te
\end{enumerate}

Die Anforderungsliste ist in Tabelle \ref{anforderungsliste} zu finden. In den nachfolgenden Abschnitten werden die einzelnen Anforderungen genauer erkl\"{a}rt.

\begin{longtable}{| l | l | l | l | l | l |}
	\caption{Anforderungsliste}\label{anforderungen}\\
	\hline
	Anforderung& Nr. & Art & Wert & Einheit & Quelle\\
	\hline
	Druck am  & 1 & BF & Eingang - Ausgang & $kPa$ & Datenbl\"{a}tter\\
	Messger\"{a}tilnet& & & $70 - 103$ (FMPS) & &\\
	& & & $40 - 103$ (APS) & &\\
	& & & $<0.746$ (OPS) & &\\
	& & & $75-105$ (UCPC) & &\\
	\hline
	Einstellzeit der & 2 & ZF & 0 & $s$ &selbstgew"{a}hlte\\
	Partikelanzahlkonzentration & & & & &Last\\
	der Aerosolquelle & & & & &\\
	\hline
	Tr\"{a}gergas& 3 & FF & & & Datenbl\"{a}tter\\
	- gefilterte Luft & & & & &\\
	- Inertgas bei UCPC & & & & &\\
	\hline
	Volumenstrom & 4 & FF & 10 (FMPS) & $l/min$ & Datenbl\"{a}tter\\
	(Tr\"{a}gergas) & & & $5 \pm 0.2$ (APS)& &\\
	& & & $1 \pm 0.05$ (OPS)& &\\
	& & & $1.5 \pm 0.05$ (UCPC)& &\\
	\hline
	Verianzaufkl\"{a}rung& 5 & MF & $>90$ & \% & selbstgew\"{a}hlte\\
	& & & & & Last\\
	\hline
	Tr\"{a}gergastemperatur & 6 & BF & $283.15-325.15$(FMPS) & $K$ & Datenbl\"{a}tter\\
	& & & $283.15-313.15$(APS) & &\\
	& & & $273.15-318.15$(OPS) & &\\
	& & & $283.15-308.15$(UCPC) & &\\
	\hline
	Partikelanzahlkonzentration & 7 & BF & $100-10^{7}$ (FMPS)& $n/cm^{3}$ & Datenbl\"{a}tter\\
	& & & $0-1000$ (APS) & &\\
	& & & $0-3000$ (OPS) & &\\
	& & & $0-3\cdot10^{5}$ (UCPC) & &\\
	\hline
	Partikelmaterial (Messger\"{a}t) & 8 & W &  &  & Datenblatt APS\\
	\hline
	Partikelgr\"{o}{\ss}e & 9 & BF & $5.6-560$ (FMPS) & $nm$ & Datenbl\"{a}tter\\
	& & & $500-20000$ (APS) & &\\
	& & & $300-10000$ (OPS) & &\\
	& & & $2.5-3000$ (UCPC) & &\\
	\hline		
	Schaltzeit & 10 & ZF & 0 & s & selbstgew\"{a}hlte\\
	& & & & & Last\\
	\hline
	Abs\"{a}tze, abrupte Quer- & 11 & ZF & 0 & n & VDI 3491\\
	schnitts\"{a}nderungen, & & & & &\\
	Str\"{o}mungsumlenkungen & & & & &\\
	\hline			
	Weg zwischen Umschaltvorrichtung & 12 & ZF & 0 & m & selbstgew\"{a}hlte\\ 
	und Messger\"{a}teinlass & & & & & Last\\
	\hline
	laminarer Strom & 13 & MF & $< 2300$ & dimensionslos & VDI 3491\\
	& & ZF & $1500$ & &\\
	\hline
	Scnittstelle zum & 14 & FF & 9.5 (FMPS) & $mm$ & Datenbl\"{a}tter\\
	Messger\"{a}t (Inlet) & & & 19 (APS) & &\\
	& & & 6.35 (OPS) & &\\
	& & & 6.4 (UCPC) & &\\
	\hline			
	minimale Versuchszeit& 15 & ZF & 3300 & $s$ & selbstgew\"{a}hlte\\
	& & & & & Last\\
	\hline
	Kosten & 16 & ZF & 0 & Euro & Festlegung \\
	& & BF & $0-1000$ & & durch FZD\\
	\hline
	Anzahl Messger\"{a}te& 17 & MF & $\geq2$ & n & Festlegung\\
	& & & & & durch FZD\\
	\hline
\end{longtable}

\subsection{Messger\"{a}te}
Bei der Konzeption der Versuchsaufbauten sollte anfangs darauf geachtet werden, dass mit diesen verschiedene Messger\"{a}te unterschiedlicher Hersteller getestet werden k\"{o}nnen. Im Verlauf des Projekts wurde zur Vereinfachung des Konzeptentwurfs nach Absprache mit dem Auftraggeber sich darauf geeinigt, sich auf die Messger\"{a}te \textbf{Fast Mobility Particle Sizer Spectrometer Model 3091} (kurz FMPS 3091) und \textbf{Optical Particle Sizer Model 3330} (kurz OPS 3330) der Firma \textbf{TSI} zu konzentrieren. Einige der quantifizierten Werte sind daher aus den Handb\"{u}chern f\"{u}r die Anforderungen entnommen worden.

Beide Messger\"{a}te saugen sich ihre ben\"{o}tigten Aerosol-Luft-Gemisch Mengen \"{u}ber interne Pumpen an. Beim FMPS 3091 betr\"{a}gt dies insgesamt $50 L/min$, wovon $10 L/min$ auf das Aerosol und $40 L/min$ auf Luft entfallen, w\"{a}hrend der OPS 3330 jeweils $1 L/min$ Luft und Aerosol ben\"{o}tigt. Hauptunterschied zwischen den beiden Ger\"{a}ten sind ihre Mechanismen, mit denen Partikel detektiert werden. Der FMPS 3091 misst die elektrostatische Ladung einzelner Partikel, w\"{a}hrend der OPS 3330 ein optisches Verfahren anwendet. Weitere Unterschiede liegen in der Gr\"{o}{\ss}enverteilung der Partikel, bei denen die jeweiligen Messger\"{a}te noch korrekt funktionieren und der Partikelkonzentration. 

Aus diesen und weiteren Werten ergeben sich die nachfolgenden Anforderungen, die die Versuchsaufbauten zu erf\"{u}llen haben. Hierbei wurde immer versucht eine Schnittmenge zwischen den Wertebereichen der Messger\"{a}te zu finden nach der sich die Anforderungen richten um so garantieren zu k\"{o}nnen, dass die Pr\"{u}fst\"{a}nde auf mehrere Messger\"{a}te anwendbar sind.   

\subsection{Aerosolquelle}
Die Aerosolquelle liefert mit dem Aerosol, den entscheidenden Faktor zum Partikelanzahlkonzentrationssprungs. Aus den Arbeitsbereichsbegrenzungen der Messger\"{a}te sowie allgemeiner Richtlinien zur Aerosolgenerierung ergeben sich folgende Anforderungen:

\begin{itemize}
\item 1. Die internen Pumpen, welche den Volumenstrom der Messger\"{a}te regulieren, arbeiten nur in einem bestimmten Druckbereich korrekt (Quelle: Datenbl\"{a}tter). Dieser Druckbereich begrenzt wiederum den Aerosoldruck, welcher am Messger\"{a}t-Einlass herrschen darf. Es ist darauf zu achten, dass die Aerosolquelle diesen Druckbereich einh\"{a}lt. 
Auch das Design der Aerosolleitungen und der Schaltvorrichtung beeinflussen den Druck am Messgerät-Einlass (Quelle: TSL-Skript).

\item 2. Die Einstellzeit einer Leistungskenngr\"{o}{\ss}e beschreibt das dynamische Verhalten der Aerosolquelle bez\"{u}glich dieser Gr\"{o}{\ss}e bei sprungf\"{o}rmigen \"{A}nderungen eines Einstellparameters. Dabei ist vor allem die Einstellzeit der Partikelanzahlkonzentration für den Versuchsaufbau interessant.(Quelle: VDI 3491)

\item 3. Das Tr\"{a}gergas der Aerosole hat einen gro{\ss}en Einfluss auf das Messverhalten der Messger\"{a}te.  Für den Versuchsaufbau kommt vor allem gereinigte Luft als Tr\"{a}gergas in Frage, da diese auch bei der Messung der Bremspartikel als Tr\"{a}gergas fungiert und die ausgew\"{a}hlten Messger\"{a}te mit Luft oder Inertgas (UCPC) arbeiten. (Quelle: Datenbl\"{a}tter)

\item 4. Die Messger\"{a}te arbeiten mit einem fest vorgegebenen Volumenstrom des Tr\"{a}gergases.  Es muss gew\"{a}hrleistet sein, dass die Aerosolquelle einen entsprechenden Strom bereitstellen kann. Sollte der Volumenstrom den Anforderungen nicht entsprechen, generieren die Messger\"{a}te eine Fehlermeldung und liefern keine Messwerte mehr. (Handb\"{u}cher)

\item 5. Varianzaufkl\"{a}rung ist ein Maß Für die G\"{u}te eines Mathematischen Modells. Sie sagt aus inwieweit dieses Modell die Streuung von Messdaten erkl\"{a}ren kann. In dem Versuchsaufbau trifft sie eine Aussage über die Abh\"{a}ngigkeit der ben\"{o}tigten Anzahl der Messwerte und der Varianz der Partikelanzahkonzentratioslwerte des Stromes.

\item 6. Die Tr\"{a}gergastemperatur kann je nach Aerosolquelle verschiedene Werte annehmen. Da die Temperatur des Aerosols den Messprozess und die Eigenschaften des Aerosolstromes beeinflusst, geben die Messger\"{a}tehersteller feste Temperaturgrenzen vor zwischen welchen sich diese Temperatur bewegen darf.(Datenbl\"{a}tter)
\end{itemize}

\subsection{Aerosol}
Folgende Anforderung an das Aerosol ergeben sich aus den Arbeitsbereichsbegrenzungen der Messger\"{a}te und aus der statistischen Verwertbarkeit der Messwerte zur Berechnung des zeitlichen Verhaltens der Messger\"{a}te:

\begin{itemize}
\item 7. Eine zu hohe Partikelanzahlkonzentration führt zu einer H\"{a}ufung von falschen Messergebnissen zum Beispiel durch Koinzidenz bei optischen Sensoren. Messergebnisse bei einer zu niedrigen Partikelanzahlkonzentration k\"{o}nnen bei Messger\"{a}ten, welche mit elektrischen Feldern arbeiten, in dem herrschenden, systematischen Grundrauschen untergehen(Quelle: Handb\"{u}cher). Somit muss f\"{u}r verwertbare Ergebnisse die Partikelanzahlkonzentration des Aerosols mit dem benutzten Messgerät abgeglichen werden.

\item 8. Partikelmaterialien k\"{o}nnen Gesundheits- oder Umweltgef\"{a}hrdend sein oder aufgrund von ihren Eigenschaften f\"{u}r bestimmte Messverfahren nicht geeignet sein.(VDI 3491) Deswegen ist es wichtig das benutzte Partikelmaterial zu kennen. Da die hier benutzten Messgeräte vor allem für die Feinstaubmessung benutzt werden, bieten sich zum Beispiel feste Materialien für die Aerosole an.

\item 9. Zu kleine Partikel k\"{o}nnen von den Messsystemen der Partikelz\"{a}hler nicht erfasst werden, w\"{a}hrend zu gro{\ss}e Partikel durch Verstopfen der Filter die Funktion beeintr\"{a}chtigen k\"{o}nnen (Datenbl\"{a}tter). Manche Hersteller geben an, dass zu große Partikel zwar gez\"{a}hlt werden, aber nicht mehr in die Gr\"{o}{\ss}enverteilung mit eingehen.(Datenblatt FMPS) Trotzdem sollte zur Sicherheit ein Zyklonabscheider dem System zugeschaltet werden, wenn nicht garantiert werden kann, dass nicht zu viele zu gro{\ss}e Partikel im Aerosol sind.
\end{itemize}

\subsection{Schaltvorrichtung}
Die Schaltvorrichtung ist das entscheidende Bauteil beim Erzeugen der Sprungfunktion. Dabei sind ein schneller Schaltprozess aber auch ein str\"{o}mungsg\"{u}nstiges Design für die G\"{u}te der Schaltvorrichtung die ausschlaggebenden Kriterien. Auch das Material kann hierbei eine Rolle spielen. Folgende Anforderungen lassen sich definieren:

\begin{itemize}
	\item 10. DIe Zeit, die die Vorrichtung ben\"{o}tigt um von einem Schaltzustand in den anderen zu wechseln, sollte so kurz wie m\"{o}glich sein. Diese Schaltzeit tr\"{a}gt zur Totzeit des Versuchsaufbaus bei. 
	
	\item 11. Hochwinklige Umlenkungen und abrupte Querschnitts\"{a}nderungen k\"{o}nnen zu Partikel- und Druckverlusten f\"{u}hren. Somit k\"{o}nnen sie die G\"{u}te der Partikelfront beeinflussen und sollten deswegen vermieden werden. Außerdem kann es bei ung\"{u}nstigem Design der Schaltvorrichtung zu Turbulenzen in ihrem Inneren kommen, was die Berechenbarkeit der Totzeit des Aufbaus erheblich erschwert. (HIwis vom Windkanal) Die vorhergehenden \"{U}berlegungen gelten auch für die Auslegung der Aerosolleitung.
\end{itemize}

\subsection{Aerosolleitungen}
Auch bei der Leitung des Aerosols, ist auf eine str\"{o}mungsg\"{u}nstige Auslegung zu achten. Dies betrifft die Ausma{\ss}e, das Material, den Zustand der Innenfl\"{a}chen und die r\"{a}umliche F\"{u}hrung der einzelnen Verbindungen. Die folgenden Anforderungen k\"{o}nnen also auch als Designkriterien angesehen werden:

\begin{itemize}
\item 12. Der Weg zwischen der Umschaltvorrichtung und dem Messger\"{a}teeinlass f\"{u}hrt bei gegebenen Volumenstr\"{o}men, abh\"{a}ngig von dem benutzten Messger\"{a}t, zu einer Verz\"{o}gerungszeit, die sich in der Totzeit des Versuchsaufbaus niederschl\"{a}gt. Da das Str\"{o}mungsverhalten in den Verbindungen nur unter Annahme von Vereinfachungen zutrifft, gilt: Umso l\"{a}nger der Weg ist, umso gr\"{o}{\ss}er ist der Rechenfehler bei der Berechnung der Totzeit. Außerdem f\"{u}hren Druck- und Partikelverluste auf diesem Weg zu einer Verzerrung der Partikelfront. (VDI 3491)

\item 13. Da eine turbulente Str\"{o}mung zu einem analytisch kaum berechenbaren Verhalten der Str\"{o}mung f\"{u}hrt, ist der Versuchsaufbau so zu gestalten, dass sich die Str\"{o}mung m\"{o}glichst \"{u}berall laminar verh\"{a}lt.(VDI 3491) Vor allem am Messger\"{a}te- Einlass, wo es zu einem Querschnitts\"{u}bergang kommen kann, sollten entsprechend gestaltete Adapter daf\"{u}r Sorge tragen.

\item 14. Die Messger\"{a}te besitzen einen Aerosoleinlass in Form eines R\"{o}hrchens mit unterschiedlichen Durchmessern, welches in die Umgebung ragt (Datenbl\"{a}tter/Handb\"{u}cher). Je nach Versuchsaufbau k\"{o}nnen Adapter von N\"{o}ten sein, um eine direkte Verbindung zwischen Schaltvorrichtung und Messger\"{a}t zu gew\"{a}hrleisten. Bei der Gestaltung dieser Adapter muss ein Kompromiss zwischen Anforderung 12 und 13 eingegangen werden.
\end{itemize}

\subsection{Versuchsaufbau}
Folgende Anforderungen betreffen den Versuchsaufbau als ganzes:

\begin{itemize}
\item 15. Die minimale Versuchszeit ist wichtig in der Versuchsplanung, da durch die Messaufl\"{o}sung der Messger\"{a}te von einer Sekunde, die Anzahl der Messwerte je nach Versuchsaufbau begrenzt ist. Nun kann eine Versuchsdurchf\"{u}hrung aus mehreren Abl\"{a}ufen (abwechselnd: gereinigte Luft (zb. 5 s) -- Aerosol(zb. 30s)) und die Abl\"{a}ufe wiederum aus mehreren Messungen (der Partikelanzahl) bestehen. Je nach Varianz der Partikelanzahl, sind eine bestimmte Anzahl von Messungen und Messabl\"{a}ufen von N\"{o}ten um ein valides Ergebnis für die Zeitkonstante zu erhalten. So sollte der Aufbau ca. 100 Abl\"{a}ufe mit je 30 Partikelanzahlmessungen (+ 5 Sekunden gereinigte Luft pro Ablauf) mit reproduzierbaren Ergebnissen liefern k\"{o}nnen.

\item 16. Aus wirtschaftlicher Sicht, sind die Kosten ein ausschlaggebender Faktor für die G\"{u}te des Aufbaus. Sie setzen sich aus den Materialkosten selbstgestalteter und eventuell selbstgebauter Teile, den Kosten für Zukaufteile, wie Filter, Verdichter, Aerosolgeneratoren etc. und den Kosten f\"{u}r Ressourcen, wie Strom, Aerosol etc., zusammen. Die maximalen Kosten d\"{u}rfen die vorgegebenen Mittel des Fachbereiches nicht \"{u}berschreiten. (Fachbereich Fahrzeugdynamik)

\item 17. Ziel des Aufbaus ist die Evaluation der Dynamik von Messger\"{a}ten, welche f\"{u}r Messungen von Bremspartikeln am Schwungmassenpr\"{u}fstand ben\"{o}tigt werden. Von den f\"{u}nf f\"{u}r diesen Zweck ausgesuchten Messger\"{a}ten, wurden vor allem der FMPS 3091 und der OPS 3330 f\"{u}r geeignet befunden(Fachbereich Fahrzeugdynamik). Nun sollen die, durch den hier konstruierten Versuchsaufbau, ermittelten Werte zumindest f\"{u}r diese beiden Ger\"{a}te ein valides Ergebnis darstellen. Jedes weitere Messger\"{a}t, das durch den Aufbau erfolgreich getestet werden kann, erh\"{o}ht dessen wirtschaftlichen und wissenschaftlichen Wert.
\end{itemize}