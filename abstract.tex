\section*{Kurzzusammenfassung}
Das Sch\"{u}tzen der Umwelt und der Gesundheit der Mitmenschen spielt in der Automobilentwicklung eine immer gr\"{o}{\ss}ere Rolle. Doch nicht nur die Verbrennungsstoffe der Motoren oder die Abriebpartikel der Reifen f\"{u}hren zu einer Umweltbelastung. Immer interessanter werden tempor\"{a}re Partikelquellen wie unter anderem die Abriebemissionen der Bremsen eines Fahrzeugs. Da die Eigenschaften und Zusammensetzung dieser Emissionen jedoch noch weitgehend unerforscht und unbekannt sind, m\"{u}ssen diese erst erforscht werden. 
\\\\
Zur Messung und Analyse von Emissionspartikeln verwendet man Messger\"{a}te, welche Partikelanteile in der vorhandenen Luft messen. Eine Materialanalyse der aufgefangenen Partikel gibt Aufschluss \"{u}ber die chemische Zusammensetzung der Partikel. So kann man aufgrund der Kenntnisse \"{u}ber die Emissionen geeignete Ma{\ss}nahmen zu Umweltschutz und Gesundheitsschutz treffen.
\\\\
Innerhalb des Advanced Design Projects (ADP) wurde ein System zur Aufbereitung und Weiterleitung einer Partikelstr\"{o}mung entwickelt und eine theoretische Konstruktion ausgearbeitet um die Qualit\"{a}t von Partikelmessger\"{a}ten im Anwendungsgebiet Bremsen zu \"{u}berpr\"{u}fen. Hauptaugenmerk lag dabei auf der Ermittlung und Quantifizierung der Zeitkonstanten von Messger\"{a}ten.
\\\\
Daf\"{u}r wurden in der ersten Phase des Projektes ein Zeitplan erstellt, der Stand der Technik und der Forschung im Bereich Str\"{o}mungslehre und Aerosolerzeugung analysiert und eine Anforderungsliste erarbeitet. Da die Problemstellung mehrere Faktoren betraf, wurden mehrere morphologische K\"{a}sten erstellt um das Problem in kleinere Arbeitsschritte zu unterteilen und L\"{o}sungen f\"{u}r diese zu finden. Dadurch konnte in den letzten Schritten des Projektes die theoretische Konstruktion mehrerer Konzepte entwickelt werden und eine Konzeptentscheidung anhand der Anforderungsliste getroffen werden.
\\\\
Anschlie{\ss}end wurde das ausgew\"{a}hlte Konzept theoretisch konstruiert. Dabei wurde ein Bauplan mit den konkreten Bauteilen der Konstruktion erstellt und die besten L\"{o}sungen f\"{u}r m\"{o}gliche Bauteile gesucht. Dieser letzte Schritt wurde immer mit Beachtung der Anforderungsliste getan. Die Evaluation unserer Konstruktion beinhaltet die analytische Auswertung in Hinblick auf Funktionalit\"{a}t und das Einhalten der Anforderungsliste.