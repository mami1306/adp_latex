%*****************************************
\chapter{Einf\"{u}hrung}\label{ch:introduction}
%*****************************************
Im Folgenden wird der Inhalt dieses Advanced Design Projects vorgestellt. Beginnend mit der Motivation, warum dieses Projekt ein interessantes Thema und eine nicht unwichtige Problematik darstellt und im Weiteren eine konkrete Darlegung der Aufgabenstellung der Arbeit. Im letzten Punkt werden die Methodiken der einzelnen Herangehensweisen an die verschiedenen Aufgaben und Themengebiete erl\"{a}utert. 

\section{Motivation}
Im Bereich der Fahrzeugentwicklung spielen seit Jahren Leistung und Energieverbrauch die gr\"{o}{\ss}te Rolle als Fortschrittskriterium. Durch steigende Belastung der Umwelt und der Gesundheit von Mensch und Tier, sind Umweltschutzfaktoren bei der Entwicklung von Kraftfahrzeugen immer weiter in den Vordergrund ger\"{u}ckt. Immer mehr gesetzliche Richtlinien schr\"{a}nken die Automobilhersteller in ihren Freiheiten ein und zwingen die Entwicklung in eine umweltfreundlichere Richtung. Anfangs zielte man nur auf Verbrennungsprodukte ab, die bei der Verwendung von Otto- und Dieselmotoren entstehen. Sp\"{a}ter kamen Altschrottentsorgung und Reifenabrieb hinzu. 
\\\\
Neben all diesen Faktoren ist auch die Bremse ein Bauteil im Kraftfahrzeug, welches durch Benutzung Emissionen abgibt. Da auch in diesem Teilgebiet gesetzliche Richtlinien zu erwarten sind, konzentriert sich die Forschung aktuell auf die Entstehung solcher Emissionen bei Kraftfahrzeugbremsen. Da Emissionsart und Partikelzusammensetzung der Bremsemissionen aktuell nicht gut erforscht und deshalb unbekannt sind, kann man nicht direkt Ma{\ss}nahmen gegen die Belastung durch Bremsemissionen ergreifen. Um Aufschluss \"{u}ber die Art der entstehenden Partikel zu bekommen, werden Partikelmesssysteme eingesetzt. An diesem Punkt steigt diese Arbeit ein um das zeitliche Messverhalten solcher Systeme zu \"{u}berpr\"{u}fen, damit dieser sp\"{a}ter an der Kraftfahrzeugbremse eingesetzt werden k\"{o}nnen.
\section{Konkretisierung der Aufgabenstellung}
Am Fachgebiet Fahrzeugtechnik der TU Darmstadt wird in Zusammenarbeit mit der Industrie am Thema Partikelemissionen von Pkw-Scheibenbremsen geforscht. Zum Vergleich partikelemissionsarmer Bremsstrategien wird auf Grundlage von Messungen ein Partikelemissionsmodell erarbeitet. Im Rahmen der Systemidentifikation sind Informationen \"{u}ber die zeitliche Aufl\"{o}sung der verwendeten Partikelmessger\"{a}te notwendig.
\\\\
Ziel des Advanced Design Projects ist es, ein Konzept f\"{u}r eine Versuchseinrichtung zu entwickeln und dieses konstruktiv umzusetzen. Dabei ist es das Ziel die Generierung eines Partikeltestsignals in Form einer Sprungfunktion zu erm\"{o}glichen und so ein Versuchswerkzeug zur Identifizierung der zeitlichen Aufl\"{o}sung der Partikelmessger\"{a}te zu entwickeln.
\\\\
Im Einzelnen teilt sich das folgende ADP in f\"{u}nf Arbeitsschritte auf. Es beginnt mit der Einarbeitung in die Thematiken Bremspartikelemissionen, Partikelmesstechnik sowie Systemidentifikation mit Hilfe von Testsignalen. Nach der Einarbeitung in die f\"{u}r die Arbeit wichtigen Themen soll eine Ausarbeitung einer Anforderungsliste f\"{u}r die Versuchseinrichtung aufgestellt werden. Im dritten Schritt werden Konzepte f\"{u}r die Versuchseinrichtung erarbeitet um diese dann anschlie{\ss}end mit Blick auf die Anforderungsliste zu vergleichen. Am Ende wird das Konzept ausgew\"{a}hlt, welches am besten den Anforderungen entspricht. Der letzte Schritt ist eine detaillierte theoretische Konstruktion der Versuchseinrichtung.
\section{Methodik}
Zu Beginn muss erw\"{a}hnt werden, dass die Aufgaben nicht in der Gruppe bearbeitet wurden, sondern in regelm\"{a}{\ss}igen Treffen der Gruppe besprochen und aufgeteilt wurden. Dabei wurden Milestones gesetzt, welche es termingerecht zu erreichen galt. Die erledigten Aufgaben wurden dann diskutiert und verfeinert. Anschlie{\ss}end wurden die neuen Aufgaben festgelegt, aufgeteilt und neue Milestones zu den Aufgaben gesetzt. Im Folgenden werden wir nicht weiter auf die Aufteilung der Aufgaben eingehen, da es f\"{u}r das Projekt keine Rolle spielt, wer die Aufgaben erledigt hat, sondern es von gr\"{o}{\ss}erer Relevanz ist, wie die einzelnen Aufgaben angegangen und bearbeitet wurden.
\begin{enumerate}
\item \textbf{Einarbeitung in die verschiedenen Thematiken}:\\
F\"{u}r die Einarbeitung wurde haupts\"{a}chlich Literaturrecherche betrieben. Dabei wurde Literatur aus der Bibliothek der TU Darmstadt verwendet, welche von verschiedenen Professoren und wissenschaftlichen Mitarbeitern empfohlen wurde. Eine weitere Quelle waren die wissenschaftlichen Ver\"{o}ffentlichen der Arbeiten am Windkanal in Darmstadt und des Fachgebietes Fahrzeugtechnik an der TU Darmstadt. Auch das Internet und digitale Archiv der ULB Darmstadt wurden als Quellen f\"{u}r die Einarbeitung genutzt.
\item \textbf{Ausarbeitung einer Anforderungsliste}:\\
F\"{u}r die Erstellung einer Anforderungsliste wurde die gesamte geplante Versuchseinrichtung in verschiedene Teile unterteilt. Dabei gab es drei wichtige Aspekte der Einrichtung zu beachten. Welche Anforderungen werden aufgrund der verwendeten Messger\"{a}te vorgegeben, welche Anforderungen werden durch die Auswahl eines passenden Partikelsubstrates vorgegeben und welche Anforderungen entstehen f\"{u}r die eigentlichen Bauteile der Versuchseinrichtung. Dabei wurden einmal physikalische, sowie allgemeine Anforderungen an die gesamte Versuchsdurchf\"{u}hrung gestellt.
\item \textbf{Entwicklung von Versuchseinrichtungskonzepten}:\\
F\"{u}r die Entwicklung von Konzepten f\"{u}r verschiedene Versuchseinrichtungen wurden alle ben\"{o}tigten Elemente der Einrichtung in einem morphologischen Kasten erstellt. Dabei wurde sich bei der L\"{o}sungsfindung an die VDI-Richtlinien \textit{2222} und \textit{2225} gehalten.\\
VDI-Richtlinie \textit{2222} beschreibt das methodische Vorgehen bei der Konstruktion von Konzepten, bei dem Probleml\"{o}sungen und die genaue Aufgabenstellung f\"{u}r Dritte nachvollziehbar hergeleitet werden. Dadurch ergeben sich Vorteile in der Wiederverwendung von Teill\"{o}sungen. Die VDI-Richtlinie \textit{2225} beschreibt das Entwickeln von vollst\"{a}ndig technischen Produkten, welche in technischer Funktionalit\"{a}t und Wirtschaftlichkeit lange konkurrenzf\"{a}hig bleiben.\\
Die ben\"{o}tigten Elemente ergaben sich aus der Aufgabenstellung und der Anforderungsliste. Durch Brainstorming wurde der morphologische Kasten mit verschiedenen Ideen zu den wichtigsten Teilelementen gef\"{u}llt. Aus den Einzelteilen entstanden in mehreren Durchl\"{a}ufen verschiedene Konzeptideen, welche skizziert und kurz beschrieben wurden. Dabei unterschieden sich die Konzepte in einem wichtigen Punkt, n\"{a}mlich der Art, wie der Partikelsprung erzeugt wird. Die Konzepte wurden nach ihrer Entstehung mit wissenschaftlichen Mitarbeitern besprochen und verfeinert, wobei manche Konzepte in diesem Schritt bereits von der Liste gestrichen wurden.
\item \textbf{Vergleich der Konzepte}:\\
Mit der Anforderungsliste als Referenz wurden die \"{u}brig gebliebenen Konzepte verglichen. Dabei haben die Konzepte Punkte in jeder Anforderung bekommen. Je nach Wichtigkeit der Anforderung wogen hier die Punktzahlen mehr oder weniger. Am Ende wurde das Konzept ausgew\"{a}hlt welches, den Anforderungen am besten standhalten konnte.
\item \textbf{Auswahl und Konstruktion eines Konzeptes}:\\
Das ausgew\"{a}hlte Konzept wurde dann als detaillierte Skizze angefertigt. Nun wurden auch erste konkrete \"{U}berlegungen zu Ma{\ss}en und Materialien gemacht, um eine endg\"{u}ltige Konstruktion zu entwerfen. Die Versuchseinrichtung wurde in Einzelteile zerlegt und es wurden konkrete Bauteile gefunden oder entwickelt, um das bisher theoretische Konzept zu verwirklichen. Ma{\ss}e, Materialien, Herkunft und Preis der Teile wurden ermittelt und festgelegt, sodass ein vollst\"{a}ndiger Bauplan des Konzeptes vorlag.
\end{enumerate}

%*****************************************
\chapter{Projektdefinition und Zeitmanagement}\label{ch:project}
%*****************************************
\section{Projektziele}
Ziel des Projektes ist es ein Konzept f\"{u}r einen Versuchsaufbau zu erarbeiten und evaluieren, mit dessen Hilfe ein Partikeltestsignal in Form einer Sprungfunktion generiert werden kann, um damit die zeitliche Aufl\"{o}sung von Partikelmessger\"{a}ten zu identifizieren. Die zeitliche Aufl\"{o}sung der verwendeten Partikelmessger\"{a}te ist f\"{u}r die Systemidentifikationen eines Partikelemissionsmodell auf der Grundlage von Messungen notwendig, mit deren Hilfe partikelemissionsarme Bremsstrategien erarbeitet werden sollen.  
\section{Aufgaben}
Zur Entwicklung eines Konzepts f\"{u}r die Generierung eines Partikeltestsignals sind folgende Aufgaben zu erf\"{u}llen:
\begin{enumerate}
	\item Einarbeitung in die Themen Bremspartikelemission, Partikelmesstechnik und Systemidentifikation mit Hilfe von Testsignalen
	\item Ausarbeitung einer Anforderungsliste f\"{u}r den Versuchsaufbau
	\item Erarbeitung von Versuchseinrichtungskonzepten
	\item Vergleich der erarbeiteten Konzepte an Hand von Kriterien aus der Anforderungsliste
	\item Auswahl eines geeigneten Konzeptes
	\item Konstruktion des ausgew\"{a}hlten Konzeptes
	\item Dokumentation der Ergebnisse 
\end{enumerate}
\section{Anforderungen}
%TODO

\subsection{Aerosole}
%TODO

\subsection{Versuchsaufbau}
%TODO

\subsection{Messger\"{a}te}
%TODO

%*****************************************
\chapter{Technische Grundlagen}\label{ch:foundations}
%*****************************************
F\"{u}r die Arbeit wurden verschiedene technische Grundlagen verwendet, welche im Folgenden kurz erl\"{a}utert werden sollen. Hierf\"{u}r wird auf allgemeine Elemente der Str\"{o}mungslehre sowie auf Einwirken der Reynolds- und der Prandtlzahl auf aktive Str\"{o}mungen hingewiesen. Da die Arbeit sich um die Messung von Bremsemissionspartikeln dreht, werden auch die Eigenschaften von Partikeln und Aerosolen, die f\"{u}r die Arbeit verwendet wurden kurz erkl\"{a}rt. Letztendlich werden die mechanischen Grundlagen f\"{u}r verschiedene verwendete Bauteile erl\"{a}utert und der aktuelle Stand der Technik im Bereich Aerosolstr\"{o}mung und Messtechnik aufgegriffen.

\section{Str\"{o}mungslehre}
Da es Ziel der Arbeit ist, einen konstanten Strom von Partikeln durch ein bestimmtes Medium zu generieren, um letztendlich ein Partikeltestsignal zu bekommen, ist es wichtig grundlegende physikalische Eigenschaften von Str\"{o}mungen zu kennen. Deshalb ist es unumg\"{a}nglich sich im Vorfeld mit den Grundlagen der Str\"{o}mungslehre auseinanderzusetzen.
\\\\
Die Str\"{o}mungslehre ist die Wissenschaft vom physikalischen Verhalten von Fluiden und Gasen. Die in dieser Lehre gewonnenen Kenntnisse sind Gesetzm\"{a}{\ss}igkeiten in Str\"{o}mungsvorg\"{a}ngen und dienen der L\"{o}sung von Str\"{o}mungsproblemen. Dabei liegt der Fokus auf den Problemen bei umstr\"{o}mten oder durchstr\"{o}mten Bauteilen. Gegenstand der Str\"{o}mungslehre sind die Bewegungen von Fluiden, Gasen und ruhenden, flie{\ss}enden oder str\"{o}menden Substanzen. Die Str\"{o}mungslehre l\"{a}sst sich in verschiedene Teilgebiete unterteilen, von denen allerdings f\"{u}r diese Arbeit nur die Fluiddynamik relevant ist und die Auslegungen sich daher auf dieses Teilgebiet beschr\"{a}nken werden\cite{stream}.

\subsection{Str\"{o}mungseigenschaften}
Grob betrachtet gibt es zwei wichtige Eigenschaften von Str\"{o}mungen, die f\"{u}r diese Arbeit wichtig sind. Auf der einen Seite gibt es die laminaren Str\"{o}mungen, die man sich als ein gleichm\"{a}{\ss}iges Flie{\ss}en vorstellen kann, auf der anderen Seite als Gegensatz die turbulenten Str\"{o}mungen.
\\\\
In der Fluiddynamik ist die laminare Str\"{o}mung eine Bewegung des Fluids ohne sichtbare Turbulenzen. Dabei str\"{o}mt das Fluid in Schichten, die sich nicht miteinander vermischen. Verwirbelungen treten erst mit h\"{o}heren Str\"{o}mungsgeschwindigkeiten auf, ab diesem Zeitpunkt ist die Str\"{o}mung turbulent.
\begin{figure}[H]
        \myfloatalign
        {\includegraphics[width=.9\linewidth]{gfx/fundamentals/laminar.jpg}} \quad
        \caption[Laminare Str\"{o}mung]
        {Laminare Str\"{o}mung}
        \label{fig:laminar}
\end{figure}
\newpage
Im Vergleich dazu ist die turbulente Str\"{o}mung die Bewegung von Fluiden, bei der Verwirbelungen auftreten. Diese Str\"{o}mungsform ist gekennzeichnet durch eine scheinbar zuf\"{a}llig variierende Komponente. Turbulenzen f\"{u}hren zu einer verst\"{a}rkten Durchmischung des Fluids mit umgebenden Gasen. Dabei hat die turbulente Str\"{o}mung eine ungeordnete und schwer vorhersagbare Struktur
\begin{figure}[H]
        \myfloatalign
        {\includegraphics[width=.9\linewidth]{gfx/fundamentals/turbulent.jpg}} \quad
        \caption[Turbulente Str\"{o}mung]
        {Turbulente Str\"{o}mung}
        \label{fig:turbulent}
\end{figure}
F\"{u}r diese Arbeit wurden verschiedene Ideen mit verschiedenen Anforderungen an die Str\"{o}mung entwickelt. Gemeinsam haben alle Konzepte, dass ein laminarer Strom des Fluids am Eingang des Messger\"{a}tes verlangt war. Bei einigen kam noch hinzu, dass das Fluid vor dem Einstr\"{o}men in das Messger\"{a}t mit Luft vermischt werden sollte. Dort ist eine gezielte turbulente Str\"{o}mung gewollt, um eine Vermischung zu generieren, die im sp\"{a}teren Aufbau allerdings wieder eine laminare Form annehmen sollte\cite{stream}.
\subsection{Reynoldszahl}
Die Reynoldszahl ist eine dimensionslose Kennzahl. Sie wird in der Str\"{o}mungslehre als Beurteilungskriterium f\"{u}r laminare Str\"{o}mungen verwendet und ist deshalb relevant f\"{u}r diese Arbeit. Die Reynoldszahl ist definiert durch:
\begin{align*}
{Re} = \frac{\rho vd}{\eta} = \frac{vd}{\nu}
\end{align*}
Dabei ist \(\rho\) die Dichte des Fluids, \(v\) die Str\"{o}mungsgeschwindigkeit des Fluids gegen\"{u}ber dem durchstr\"{o}mten K\"{o}rper und \(d\) die L\"{a}nge des K\"{o}rpers. Die kinematische Viskosit\"{a}t \(\nu\) des Fluids unterscheidet sich von der dynamischen Viskosit\"{a}t \(\eta = \nu \rho\) durch den Faktor \(\rho\). \"{U}berschreitet die Reynoldszahl einen kritischen Wert \({Re}_{krit}\), wird eine bis dahin laminare Str\"{o}mung anf\"{a}llig gegen kleinste St\"{o}rungen und aus der laminaren Str\"{o}mung wird eine turbulente Str\"{o}mung. In Bezug auf die Arbeit war es also wichtig eine Reynoldszahl zu erreichen, die dem verlangten Str\"{o}mungsbild gen\"{u}gt\cite{reynolds}.
\section{Partikeleigenschaften}
Die in dieser Arbeit verwendeten Partikel sind mineralische und organische Feststoffe, die nicht in L\"{o}sung gehen und wegen ihrer geringen Gr\"{o}{\ss}e und damit geringen Gewichts in der Schwebe gehalten oder durch geringe Bewegungen des Mediums verteilt werden. Im speziellen sind das hier Aerosolpartikel, St\"{a}ube und auch Nebel- und Rauchpartikel.\\
Die Eigenschaft von Partikeln, \"{u}ber l\"{a}ngere Zeit in Gasen transportiert werden zu k\"{o}nnen, liegt darin, dass sie sich mit abnehmendem Durchmesser immer mehr wie Gas Molek\"{u}le verhalten.\\
Partikel entstehen durch verschiedene Vorg\"{a}nge, welche ausschlaggebend f\"{u}r die Gr\"{o}{\ss}e der Partikel sind. Die kleinsten Partikel entstehen durch Verbrennungen und sind selten gr\"{o}{\ss}er als \(1 nm\). Mineralstaubpartikel hingegen gehen auf Abrieb mineralischer Stoffe zur\"{u}ck und sind meist von gr\"{o}{\ss}erer Form im \(\mu m\) Bereich.\\
Partikel k\"{o}nnen vom Menschen eingeatmet werden, dabei bleiben bis zu 10\% der Partikel im Atemtrakt h\"{a}ngen und f\"{u}hren so zu einer gro{\ss}en Belastung in Lungen und Bronchien. Da es abh\"{a}ngig von der Partikelgr\"{o}{\ss}e ist, ob und wie sch\"{a}dlich die Partikel sind, ist es wichtig durch Messungen zu erfahren welche Eigenschaften und Aufbau die Partikel haben, die durch den Bremsvorgang entstehen\cite{reinraumtechnik}.
\newpage

\subsection{Bremsemissionspartikel}
Die Emissionspartikel von Bremsen  sind Bestandteil des Bremsstaubes, welcher beim Bremsvorgang entsteht. Sie stellen ein Gemisch aus Bremsbelagabrieb und Abrieb der Bremsscheiben dar, wobei der gr\"{o}{\ss}ere Anteil auf die Bremsbel\"{a}ge f\"{a}llt, da diese wesentlich weicher sind. Kritisch beim Bremsstaub sind die Partikel von Eisen, Kupfer und Mangan, die beim scharfen Bremsen freigesetzt werden. Die toxische Wirkung der einzelnen Partikel k\"{o}nnen Entz\"{u}ndungen in den Lungenzellen verursachen.\\
Da es auf das Bremsverhalten ankommt, welche Partikel in welchen Massen entstehen, ist es wichtig die entstehenden Partikel bei verschiedenen Bremsvorg\"{a}ngen zu messen und zu analysieren. Nur dann kann man gezielt Partikel bestimmter Art und Sorte reduzieren oder abfangen. So ist zum Beispiel das krebserregende Asbest auch aus der Entwicklung von Bremsscheiben herausgenommen worden\cite{envyl}.

\subsection{Partikelmessverfahren}
Unter dem Begriff Partikelmessung wird eine Gruppe von Messverfahren zur Qualifizierung und Quantifizierung von Partikeln unterschiedlicher Natur zusammengefasst. Dabei wird meistens das Prinzip des Streulichtpartikelmessung angewendet. Bei der Streulichtpartikelmessung wird eine definierte Menge einer Probe durch einen Laserstrahl gef\"{u}hrt. Das Licht des Laserstrahls bricht sich an den Partikeln oder wird von ihnen absorbiert. Photodioden k\"{o}nnen diese Effekte messtechnisch erfassen und in ein elektrisches Signal umwandeln. Dieses Signal wird von einem Computer mit einem zuvor aufgenommenen Signal verglichen, bei dem Latexkugeln definierter Gr\"{o}{\ss}e vermessen worden sind, um Referenzwerte zu schaffen. Mit diesen Daten ist man in der Lage aufgrund von statistischen Methoden die Anzahl der Partikel in einem Kubikmeter Luft oder Gas zu ermitteln. Dieses Verfahren funktioniert allerdings nur f\"{u}r Partikel mit einer Gr\"{o}{\ss}e von unter \(25 \mu m\)\cite{reinraumtechnik}.

\subsection{Aerosole}
Ein Aerosol ist ein heterogenes Gemisch aus festen oder fl\"{u}ssigen Partikeln in einem Gas. Ein Aerosol ist ein dynamisches System und unterliegt st\"{a}ndigen \"{A}nderungen durch Kondensation von D\"{a}mpfen an bereits vorhandenen Partikeln, Verdampfen fl\"{u}ssiger Bestandteile der Partikel oder Abscheidung von Teilchen an umgebenden Gegenst\"{a}nden.\\
Aerosole k\"{o}nnen durch mechanische Zerkleinerung von Material oder durch Kondensation von Material entstehen. Mechanische Prozesse umfassen Zerreiben, Zerstoßen oder andere Zerkleinerungsprozesse von Feststoffen. Kondensation hingegen ist ein Prozess, in dem sich festes oder fl\"{u}ssiges Material aus \"{u}bers\"{a}ttigten Gasen bildet.
\\\\
Der Durchmesser von Aerosolpartikeln liegt in der Gr\"{o}{\ss}enordnung zwischen \(0,1 nm\) und \(10 \mu m\) und sie haben unterschiedliche Zusammensetzungen. Kleinste Partikel sind einzelne Molek\"{u}le, die bei Verbrennungen entstehen.
\\\\
Angewendet werden Aerosole gro{\ss}fl\"{a}chig in der Industrie als Spraybasis, zum Auftragen von Lacken, medizinische Sprays und als Industriek\"{u}hlmittel. Um f\"{u}r die Zwecke dieser Arbeit einen Partikelsprung zu erzeugen hat sich ein Aerosol als Partikeltr\"{a}ger angeboten, da die Partikel in Aerosolen gleich verteilt sind und zu laminaren Str\"{o}mungen neigen\cite{aerosole}.
\section{Mechanische Grundlagen}
In den vorgestellten Konzepten werden verschiedene mechanische Bauteile verwendet. Dabei werden einfache Teile wie Rohre, Klemmen und Schläuche als trivial angenommen und es wird nicht weiter auf die Eigenschaften dieser Teile eingegangen. Kompliziertere Bauteile sollen hier allerdings kurz erläutert werden um das Verständnis für deren Funktion zu vereinfachen und die Anwendung zu verstehen.
\\\\
Da der aufgebaute Aerosolstrom durch ein System von mehreren Rohren und Schläuchen geleitet werden muss und gegebenenfalls auch umgeleitet werden muss, werden Ventile benötigt. Ein weiteres Bauteil sind Luftfilter um die zum Mischen verwendete Umgebungsluft zu reinigen. Letztendlich werden noch Ventilatoren für einige Konzepte benötigt um dem Strom eine konstante Geschwindigkeit zu geben.

\subsection{Ventile}
Ein Ventil ist ein Bauteil zur Absperrung oder Regelung des Durchflusses von Fluiden. Innerhalb des Bauteils wird ein Verschlussteil nahezu parallel zur Strömungsrichtung des Fluids bewegt. Die Strömung kann reduziert oder unterbrochen, indem das gesamte Verschlussteil an eine passend geformte Öffnung gepresst wird. Ventile haben die Eigenschaft über den gesamten Stellbereich ein gleichmäßiges Strömungsbild zu besitzen, weshalb sie sich gut für Regelaufgaben eignen.
\\\\
Ventile lassen sich in vier Kategorien einteilen. Durchgangsventile reduzieren Strömungen und der Eintritt liegt in der selben Richtung wie der Austritt. Eckventile leiten den Strom um, indem Ein- und Austritt im rechten Winkel zueinander liegen, während Schrägventile die Strömungsrichtung um 45° ändern. Die für diese Arbeit wichtigen Ventile sind Drei-Wege-Ventile, die für das kontrollierte Mischen von Fluidströmen verwendet werden. Bei allen Kategorien kann man noch Unterschiede in den Betätigungsarten machen, um jedoch ein Signal für den Messbereich zu haben, werden hier elektronische Betätigungen bevorzugt.

\subsection{Luftfiltersysteme}
Als Luftfilter werden alle Abscheider bezeichnet, die Aerosole oder andere unerwünschte Schwebstoffe aus der Luft herausfiltern. Es gibt viele verschiedene Bauarten von Luftfiltersystemen, doch es werden nur Trockenfilter, Zyklonabscheider und elektrostatische Luftfilter verwendet, weshalb die flüssigkeitsbasierten Filter hier nicht wieder erwähnt werden. \\
Trockenfilter bestehen aus ringförmig oder rechteckig angeordneten Gebilde, die zickzackförmiges Gewebe als Filterelement haben. Die Faltung vergrößert die Filterfläche und verringert den Strömungswiderstand. Die kontaminierte und ungefilterte Luft wird ausserhalb des Zylinders angesaugt und die gefilterte saubere Luft wird innerhalb der Lamellen weitergeleitet. Die Partikel und Schwebeteilchen bleiben in den Gewebefalten hängen.\\
Elektrofilter finden hauptsächlich Anwendung bei der Abscheidung von Aerosolen, sind also für die Arbeit sehr gut geeignet. Im Filter werden die Staub- oder Aerosolpartikel elektrostatisch aufgeladen und an den Elektrodenflächen abgeschieden. In getakteten Abfolgen werden die aufgefangenen Verschmutzungen maschinell entfernt und in einem Trichter gesammelt. Der Vorteil dieser Filter im Vergleich zu den Trockenfiltern ist, dass es keine Filtereinsätze gibt, die ausgetauscht werden müssen, allerdings sind die elektrostatischen Filter wesentlich teurer als die Wegwerffilter mit Gewebeeinsatz.\\
Zyklonabscheider werden dazu genutzt um Partikel einer bestimmten Größe aus dem Strom herauszufiltern. Diese finden ihre Anwendung um nur Partikelgrößen durchzulassen, die für die verwendeten Messgeräte erkennbar sind. Sie sind allerdings zu ineffizient um die Luft von allen Partikeln zu reinigen, weshalb sie nur zur Aufbereitung des Partikelstroms in Frage kommen.

\subsection{Ventilatoren}
Ventilatoren werden in den Konzepten verwendet um einen konstanten Strom zu erzeugen. Mit ihnen können Fluide und Gase gefördert werden. Dafür wird die Antriebsart des Ventilators in die Bewegungsenergie des Mediums umgewandelt ohne dass es in der Strömung zu einer Drucksteigerung kommt. Ein Ventilator ist eine fremd angetriebene Strömungsmaschine, die mittels eines in einem Gehäuse rotierenden Laufrads ein gasförmiges Medium fördert. 
\section{Stand der Technik}
W\"{a}hrend der Einarbeitungszeit sind wir durch Literaturrecherche auf einige Dissertationen, Abschlussarbeiten und Richtlinien aufmerksam geworden, welche vergleichbare Versuchsaufbauten beinhalten, weche haupts\"{a}chlich zur Kalibrierung von Partikelmessger\"{a}ten dienen.

Die Masterarbeit \cite{auswertemethodik} hatte das Ziel das Ansprechverhalten einer Constant Volume Sampling (CVS) Anlage zu optimieren. Der daf\"{u}r verwendete Versuchsaufbau ist in Abbildung \ref{fig:aufbau_graz} abgebildet. Das hierf\"{u}r verwendete Aerosol wurde aus den Abgasen eines Versuchsfahzeugs entnommen und zusammen mit Verd\"{u}nnungsluft in einen CVS Tunnel geleitet, in welchem sich ein Partikelmesssystem befindet. Die Partikeltestsignale wurden hierbei mit Hilfe eines Umschalters erzeugt, bei dem zwischen einem Aerosolstrom und Luftstrom geschaltet werden kann (Abbildung \ref{fig:umschalter_graz}). Diese M\"{o}glichkeit der Generierung eines Partikeltestsignals wurde in die Erarbeitung von unseren Konzepten miteinbezogen. Zudem k\"{o}nnte der CVS Tunnel f\"{u}r nachfolgende Arbeiten, welche sich ausschlie{\ss}lich mit den Str\"{o}mungen der Aerosole besch\"{a}ftigen w\"{u}rde, von Interesse sein.

\begin{figure}[H]
	\myfloatalign
	{\includegraphics[width=.7\linewidth]{gfx/related/graz_versuch.pdf}} \quad
	\caption[Versuchsaufbau der TU Graz (Quelle: \cite{auswertemethodik}, S.2)]
	{Versuchsaufbau der TU Graz (Quelle: \cite{auswertemethodik}, S.2)}
	\label{fig:aufbau_graz}
\end{figure}

In \cite{candle} wird die chemische Zusammensetzung von Kerzenrauch untersucht. Diese Arbeit beinhaltet zwei Punkte, welche f\"{u}r unser Projekt von Interess waren. Zum einen zeigt die Arbeit, dass sich Kerzenrauch unter bestimmten Bedinungen als Pr\"{u}faeorosol eignet. Zum anderen stellt der verwendete Versuchsaufbau (Abbildung \ref{fig:kerze_exp}) eine M\"{o}glichkeit dar, wie eine solche Aerosolprobe zu einem Partikelmessger\"{a}t gef\"{u}hrt werden kann. Im Verlauf des Projektes hat sich allerdings gezeigt, dass eine Kerze als Aerosolquelle f\"{u}r den von uns zu erarbeitenden Pr\"{u}fstand nicht geeignet ist.

\begin{figure}[H]
	\myfloatalign
	{\includegraphics[width=.9\linewidth]{gfx/related/kerze_versuch.pdf}} \quad
	\caption[Versuchsaufbau zur Ermittlung der chemischen Zusammensetzung von Kerzenrauch (Quelle: \cite{candle}, S.195)]
	{Versuchsaufbau zur Ermittlung der chemischen Zusammensetzung von Kerzenrauch (Quelle: \cite{candle}, S.195)}
	\label{fig:kerze_exp}
\end{figure} 

Auf direkte Nachfrage hin beim Hersteller der Partikelmessger\"{a}te, wie die angegebenen Werte in den technischen Dokumentationen verifiziert wurden, wiesen diese lediglich darauf hin, dass die Partikelmessger\"{a}te mit Hilfe von Aerosolgeneratoren kalibriert werden. Dies entspricht der Standardvorgehensweise zur Kalibrierung solcher Messger\"{a}te, wie sie in den VDI Richtlinien zu finden sind. Aus diesen liesen sich Kenngr\"{o}{\ss}en und Richtlinien an die zu verwendeten Aerosolgeneratoren ableiten, welche in der Reinraumtechnik Verwendung finden.
\\\\
Uns sind letzendlich keine Versuchsaufbauten bekannt, welche ausschlie{\ss}lich zur Identifizierung der Zeitkonstante eines Partikelmessger\"{a}ts dienen. Zudem lies sich die Zeitkonstante aus keinem der uns vorliegenden Handb\"{u}cher entnehmen die darauf hindeuten w\"{u}rden, dass es ein existierendes Verfahren gibt diese zu messen. 

%************************************************
\chapter{Vergleich von Pr\"{u}f-Aerosole}\label{ch:aerosol}
%************************************************
\section{Di-Ethyl-Hexyl-Sebacat (DEHS)}
%TODO
\section{Di-N-Octylphtalat (DOP)}
Di-N-Octylphtalat (DOP) ist eine organische Verbindung aus der Gruppe Phtalate. Es eine farblose, geruchlose und \"{o}lige Fl\"{u}ssigkeit. DOP wird durch die Reaktion von Phtalats\"{a}ureanhydrid und Octanol in Gegenwart eines Katalysators gewonnen. Wie DEHS ist DOP in Wasser kaum l\"{o}slich. Auch in der Anwendung deckt DOP das selbe Gebiet ab, dar\"{u}ber hinaus findet es allerdings noch Anwendung im medizinischen Bereich und in der Sprengstoffindustrie. DOP z\"{a}hlt als stark krebserregend, weshalb es nicht mehr als Weichmacher f\"{u}r Verbrauchsgegenst\"{a}nde verwendet werden darf. Auch hier wird eine gro{\ss}e Spannbreite an Partikelgr\"{o}{\ss}en abgedeckt, je nachdem welcher Partikelgenerator verwendet wird.
\begin{itemize}
\item Aggregatzustand: fl\"{u}ssig
\item Dichte: \(0,98 g/cm^3\)
\item Schmelzpunkt: \(-49^\circ\text{C}\)
\item Siedepunkt: \(385^\circ\text{C}\)
\item Partikelgr\"{o}{\ss}e: Abh\"{a}ngig vom Generator
\end{itemize}
\section{Emery 3004 (PAO-4)}
%TODO
\section{Poly Styrene Latex Spheres (PSL)}
%TODO
\section{Auswertung der Analyse}
%TODO

\subsection{Anforderungsvergleich der Aerosole}
%TODO

%************************************************
\chapter{Konzepte f\"{u}r den Versuchsaufbau}\label{ch:concepts}
%************************************************
\section{Partikelgeneratoren}
F\"{u}r die Kalibrierung und Pr\"{u}fung von Partikelmessger\"{a}ten, werden oft industrielle Partikelgeneratoren verwendet. Diese Generatoren arbeiten auf Basis von Vernebelung von Fl\"{u}ssigkeiten. Die Fl\"{u}ssigkeiten werden unter hohem Druck in einer D\"{u}se vernebelt und verspr\"{u}ht. Dabei entsteht ein Aerosol welches Partikel der verwendeten Fl\"{u}ssigkeit enth\"{a}lt. Um die Messger\"{a}te gezielt kalibrieren zu k\"{o}nnen, kann man bei den Generatoren die gew\"{u}nschte Dichte sowie die Gr\"{o}{\ss}e der Partikel innerhalb des Aerosols regulieren. So hat man zu jedem Zeitpunkt Kenntnis \"{u}ber den Aufbau des Partikelstroms. Da sich in dieser Arbeit auf die Verwendung von DEHS Aerosol beschr\"{a}nkt wird, werden nur die beiden Generatoren in Erw\"{a}gung gezogen, welche DEHS unterst\"{u}tzen. Dies sind auch die beiden Ger\"{a}te, welche in der Industrie am meisten vorkommen und verwendetet werden.

\subsection{Topas ATM 220}
Die Ger\"{a}tereihe ATM von Topas wird schon seit langer Zeit speziell im Bereich der Reinraum- und Filtertestanwendung verwendet. Das Kernst\"{u}ck des Generators ist ein Edelstahlatomizer, eine nach dem Injektorprinzip arbeitende Zweistoffd\"{u}se. Dieser sorgt f\"{u}r die R\"{u}ckf\"{u}hrung der bei der Verd\"{u}sung entstehenden gro{\ss}en Tropfen, welches die Partikelkonzentration nur unwesentlich beeinflusst. Die f\"{u}r die Verd\"{u}sung ben\"{o}tigte Druckluft wird durch einen HEPA-Filter gereinigt. Die D\"{u}se ist direkt in die Fl\"{u}ssigkeit getaucht, um auch sehr geringe Massestr\"{o}me reproduzierbar einstellen zu k\"{o}nnen.\\
Der Topas ATM hat eine Partikelmengenspannbreite  von \(10^5\) bis \(10^8\) Partikel pro \(cm^3\), bei einer Gr\"{o}{\ss}enverteilung von \(0,1 - 0,5 \mu m\).\cite{topas}

\subsection{Palas 2000H}
Der Palas 2000H wird nicht nur f\"{u}r die Kalibrierung von Partikelmessger\"{a}ten sondern auch zum Testen von Filtersystemen verwendet. Im Gegensatz zum ATM arbeitet der Palas mit Hilfe von Kondensation und Verdampfung. So kann durch Druck- und Temperatur\"{a}nderungen die Partikelgr\"{o}{\ss}enverteilung reguliert werden. Der gro{\ss}e Vorteil des Palas 2000H ist, dass er \"{u}ber Ethernet von einem Computer aus gesteuert werden kann und somit direkt in ein Evaluationsprogramm f\"{u}r den Versuchsaufbau eingebunden werden kann. Er ist wesentlich nutzerfreundlicher als der f\"{u}r den Fachmann entwickelte ATM, erzeugt aber wesentlich gr\"{u}{\ss}ere Partikel, die man f\"{u}r den Versuchsaufbau eventuell gar nicht ben\"{o}tigt. Au{\ss}erdem arbeitet der Generator mit einer Temperaturregelung und erhitzt das erzeugte Aerosol auf diesem Weg, welches sich nachteilig auf das Verhalten der Str\"{o}mung auswirken kann. Die Partikelmenge kann nicht eingestellt werden und liegt bei \(10^6\) Partikeln\(/cm^3\) bei einer regulierbaren Gr\"{o}{\ss}enverteilung von \(0,2 \mu m - 100\mu m\).\cite{palas}
\section{Konzept 1}
%TODO

\subsection{Aufbau}
%TODO

\subsection{Funktionsweise}
%TODO

\subsection{Vorteile}
%TODO

\subsection{Nachteile}
%TODO

\addtocontents{toc}{\protect\newpage}

\section{Konzept 2.1}
%TODO

\subsection{Aufbau}
%TODO

\subsection{Funktionsweise}
%TODO

\subsection{Vorteile}
%TODO

\subsection{Nachteile}
%TODO

\subsection{Variation: Konzept 2.2}
%TODO

\subsection{Vergleich mit Konzept 2.1}
%TODO

\subsection{Variation: Konzept 2.3}
%TODO

\subsection{Vergleich mit Konzept 2.1}
%TODO
\section{Konzept 3}
%TODO

\subsection{Aufbau}
%TODO

\subsection{Funktionsweise}
%TODO

\subsection{Vorteile}
%TODO

\subsection{Nachteile}
%TODO
\section{Konzept 4}
%TODO

\subsection{Aufbau}
%TODO

\subsection{Funktionsweise}
%TODO

\subsection{Vorteile}
%TODO

\subsection{Nachteile}
%TODO
\section{Konzept 5}
Konzept f\"{u}nf ist eine Art Variation vom vierten Konzept, da der Aufbau grundlegend analog ist. Unterschiede liegen im Partikelmaterial und der Durchf\"{u}hrung des Messungen. Die Partikel werden hier von einem elektronischen Verdampfer einer E-Zigarette erzeugt. Da auch hier hohe Temperaturen entstehen, ist es n\"{o}tig diese vor dem Messen zu regulieren.

\subsection{Aufbau}
Genau wie bei Konzept vier besteht die Versuchseinrichtung aus einem Verd\"{u}nner, einem Thermokonditionierer, einer Schaltvorrichtung mit Ventilator und dem Messger\"{a}t. Zur Schaltvorrichtung f\"{u}hren zum einen der Schlauch von der E-Zigarette und zum anderen ein Rohr zur Umgebungsluft. Ein Kanal am Schaltsystem f\"{u}hrt zu einem extern angeschlossenen Ventilator, ein weiterer f\"{u}hrt zum Messger\"{a}t. Zwischen dem Schaltsystem und der E-Zigarette ist ein Verd\"{u}nner kombiniert mit einem Thermokonditionierer geschaltet. Anders als zu Konzept 4 wird hier anstatt Kerzen, der Rauch der E-Zigarette verwendet. Die Versuchseinrichtung wird direkt am Mundst\"{u}ck der E-Zigarette angeschlossen.
\begin{figure}[H]
        \myfloatalign
        {\includegraphics[width=.9\linewidth]{gfx/concepts/Konzept_5.jpg}} \quad
        \caption[Skizze Konzept 5]
        {Skizze Konzept 5}
        \label{fig:concept_5}
\end{figure}

\subsection{Funktionsweise}
Wie bei Konzept vier saugt das Messger\"{a}t in der ersten Phase gereinigte Umgebungsluft. So kann sich das Messger\"{a}t zun\"{a}chst einstellen und eine Nullz\"{a}hlrate durchf\"{u}hren. Die Schaltvorrichtung befindet sich zu Beginn in \textit{Stellung 0}. W\"{a}hrend der Nullz\"{a}hlrate des Messger\"{a}tes wird der Aerosolstrom durch den am Schaltsystem angeschlossenen Ventilator aufgebaut. Somit ist der Strom bereits geschwindigkeitsbehaftet, wenn dieser zu Messger\"{a}t umgeleitet wird. Da die Partikelanzahlkonzentration zu gro{\ss} f\"{u}r die Messger\"{a}te ist, durchstr\"{o}mt der Volumenstrom zun\"{a}chst einen Rotationsverd\"{u}nner, wo dem Aerosol gereinigte Luft beigemischt wird. Die Kombination aus Verd\"{u}nner und Thermokonditionierer arbeitet analog zu der Anwendung in Konzept vier. Anschlie{\ss}end gelangt der Volumenstrom zum Schaltsystem der Versuchseinrichtung.
\\\\
Zu einer fest eingestellten Zeit wird der Schaltmechanismus in \textit{Stellung 1} geschaltet. Dadurch str\"{o}mt das Aerosol zum Einlass des Messger\"{a}tes und erzeugt dort den Partikelsprung. Die Luftzufuhr zum Messger\"{a}t ist nun unterbrochen. Das abwechselnde Beaufschlagen von gefilterter Umgebungsluft und Aerosol gew\"{a}hrleistet einen ausreichenden Konzentrationsunterschied, um das generierte Sprungsignal zu verst\"{a}rken\cite{auswertemethodik}. Um bei Konzept vier die Totzeit des Systems so gering wie m\"{o}glich zu halten, wird der Verbindungsschlauch von Schaltsystem zu Messger\"{a}teeingang sehr kurz gew\"{a}hlt.
\input{concepts/concept_6.tex}

%************************************************
\chapter{Evaluation}\label{ch:evaluation}
%************************************************
\section{Vergleich der Konzepte}
%TODO
\section{Ausschluss von Konzepten}
%TODO
\section{Konstruktion des Konzepts}
%TODO


%************************************************
\chapter{Fazit}\label{ch:conclusion}
%************************************************
\input{conclusion/conclusion.tex}

\addcontentsline{toc}{chapter}{\listfigurename}

