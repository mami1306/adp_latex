%*****************************************
\chapter{Einf\"{u}hrung}
%*****************************************
\section{Motivation}
\section{Voraussetzungen}


%*****************************************
\chapter{Projektdefinition und Zeitmanagement}\label{ch:project}
%*****************************************
\section{Projektziele}
Ziel des Projektes ist es ein Konzept f\"{u}r einen Versuchsaufbau zu erarbeiten und evaluieren, mit dessen Hilfe ein Partikeltestsignal in Form einer Sprungfunktion generiert werden kann, um damit die zeitliche Aufl\"{o}sung von Partikelmessger\"{a}ten zu identifizieren. Die zeitliche Aufl\"{o}sung der verwendeten Partikelmessger\"{a}te ist f\"{u}r die Systemidentifikationen eines Partikelemissionsmodell auf der Grundlage von Messungen notwendig, mit deren Hilfe partikelemissionsarme Bremsstrategien erarbeitet werden sollen.  
\section{Aufgaben}
Zur Entwicklung eines Konzepts f\"{u}r die Generierung eines Partikeltestsignals sind folgende Aufgaben zu erf\"{u}llen:
\begin{enumerate}
	\item Einarbeitung in die Themen Bremspartikelemission, Partikelmesstechnik und Systemidentifikation mit Hilfe von Testsignalen
	\item Ausarbeitung einer Anforderungsliste f\"{u}r den Versuchsaufbau
	\item Erarbeitung von Versuchseinrichtungskonzepten
	\item Vergleich der erarbeiteten Konzepte an Hand von Kriterien aus der Anforderungsliste
	\item Auswahl eines geeigneten Konzeptes
	\item Konstruktion des ausgew\"{a}hlten Konzeptes
	\item Dokumentation der Ergebnisse 
\end{enumerate}
\section{Anforderungen}
\subsection{Aerosole}
\subsection{Versuchsaufbau}

%*****************************************
\chapter{Technische Grundlagen}\label{ch:foundations}
%*****************************************
\section{Str\"{o}mungsmechanik}
\subsection{Str\"{o}mungseigenschaften}
\subsubsection{Reynolds- und Prandtlzahl}

\section{Reinraumtechnik}
\subsection{Eigenschaften von Partikeln}
\subsubsection{Bremsemissionspartikel}
\subsection{Partikelmessverfahren}
\subsection{Aerosole}

\section{Mechanische Grundlagen}
\subsection{Ventile (Noch nicht fest)}
\subsection{Luftfiltersysteme (Noch nicht fest)}


%************************************************
\chapter{Versuchsplattform}\label{ch:platform}
%************************************************
\section{Partikelmessger\"{a}te}
\subsection{OPS-3330}
\subsection{FMPS-3091}

\section{Simulation (Unsicher)}
\subsection{SpaceClaim (Unsicher)}
\subsection{Fluent (Unsicher)}

\section{Partikelgeneratoren}
\subsection{Topas ATM 220}
\subsection{Palas 2000H}


%************************************************
\chapter{Analyse von Pr\"{u}f-Aerosole}\label{ch:aerosol}
%************************************************
\section{Di-Ethyl-Hexyl-Sebacat (DEHS)}
\section{Di-N-Octylphtalat (DOP)}
\section{Emery 3004 (PAO-4)}
\section{Poly Styrene Latex Spheres (PSL)}
\section{Auswertung der Analyse}
\subsection{Anforderungsvergleich der Aerosole}

\addtocontents{toc}{\protect\newpage}

%************************************************
\chapter{Konzepte f\"{u}r den Versuchsaufbau}\label{ch:concepts}
%************************************************
\section{Konzept 1}
\subsection{Aufbau}
\section{Konzept 2}
\section{Konzept 3}
\section{Konzept 4}
\section{Konzept 5}


%************************************************
\chapter{Evaluation}\label{ch:evaluation}
%************************************************

\section{Analyse der Konzepte}
\section{Evaluation der Ergebnisse}

%************************************************
\chapter{Fazit}\label{ch:conclusion}
%************************************************

\addcontentsline{toc}{chapter}{\listfigurename}

