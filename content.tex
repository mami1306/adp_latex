%*****************************************
\chapter{Einf\"{u}hrung}\label{ch:introduction}
%*****************************************
Im Folgenden wird der Inhalt dieses Advanced Design Projects vorgestellt. Beginnend mit der Motivation, warum dieses Projekt ein interessantes Thema und eine nicht unwichtige Problematik darstellt und im Weiteren eine konkrete Darlegung der Aufgabenstellung der Arbeit. Im letzten Punkt werden die Methodiken der einzelnen Herangehensweisen an die verschiedenen Aufgaben und Themengebiete erl\"{a}utert. 

\section{Motivation}
Im Bereich der Fahrzeugentwicklung spielen seit Jahren Leistung und Energieverbrauch die gr\"{o}{\ss}te Rolle als Fortschrittskriterium. Durch steigende Belastung der Umwelt und der Gesundheit von Mensch und Tier, sind Umweltschutzfaktoren bei der Entwicklung von Kraftfahrzeugen immer weiter in den Vordergrund ger\"{u}ckt. Immer mehr gesetzliche Richtlinien schr\"{a}nken die Automobilhersteller in ihren Freiheiten ein und zwingen die Entwicklung in eine umweltfreundlichere Richtung. Anfangs zielte man nur auf Verbrennungsprodukte ab, die bei der Verwendung von Otto- und Dieselmotoren entstehen. Sp\"{a}ter kamen Altschrottentsorgung und Reifenabrieb hinzu. 
\\\\
Neben all diesen Faktoren ist auch die Bremse ein Bauteil im Kraftfahrzeug, welches durch Benutzung Emissionen abgibt. Da auch in diesem Teilgebiet gesetzliche Richtlinien zu erwarten sind, konzentriert sich die Forschung aktuell auf die Entstehung solcher Emissionen bei Kraftfahrzeugbremsen. Da Emissionsart und Partikelzusammensetzung der Bremsemissionen aktuell nicht gut erforscht und deshalb unbekannt sind, kann man nicht direkt Ma{\ss}nahmen gegen die Belastung durch Bremsemissionen ergreifen. Um Aufschluss \"{u}ber die Art der entstehenden Partikel zu bekommen, werden Partikelmesssysteme eingesetzt. An diesem Punkt steigt diese Arbeit ein um das zeitliche Messverhalten solcher Systeme zu \"{u}berpr\"{u}fen, damit dieser sp\"{a}ter an der Kraftfahrzeugbremse eingesetzt werden k\"{o}nnen.
\section{Konkretisierung der Aufgabenstellung}
Am Fachgebiet Fahrzeugtechnik der TU Darmstadt wird in Zusammenarbeit mit der Industrie am Thema Partikelemissionen von Pkw-Scheibenbremsen geforscht. Zum Vergleich partikelemissionsarmer Bremsstrategien wird auf Grundlage von Messungen ein Partikelemissionsmodell erarbeitet. Im Rahmen der Systemidentifikation sind Informationen \"{u}ber die zeitliche Aufl\"{o}sung der verwendeten Partikelmessger\"{a}te notwendig.
\\\\
Ziel des Advanced Design Projects ist es, ein Konzept f\"{u}r eine Versuchseinrichtung zu entwickeln und dieses konstruktiv umzusetzen. Dabei ist es das Ziel die Generierung eines Partikeltestsignals in Form einer Sprungfunktion zu erm\"{o}glichen und so ein Versuchswerkzeug zur Identifizierung der zeitlichen Aufl\"{o}sung der Partikelmessger\"{a}te zu entwickeln.
\\\\
Im Einzelnen teilt sich das folgende ADP in f\"{u}nf Arbeitsschritte auf. Es beginnt mit der Einarbeitung in die Thematiken Bremspartikelemissionen, Partikelmesstechnik sowie Systemidentifikation mit Hilfe von Testsignalen. Nach der Einarbeitung in die f\"{u}r die Arbeit wichtigen Themen soll eine Ausarbeitung einer Anforderungsliste f\"{u}r die Versuchseinrichtung aufgestellt werden. Im dritten Schritt werden Konzepte f\"{u}r die Versuchseinrichtung erarbeitet um diese dann anschlie{\ss}end mit Blick auf die Anforderungsliste zu vergleichen. Am Ende wird das Konzept ausgew\"{a}hlt, welches am besten den Anforderungen entspricht. Der letzte Schritt ist eine detaillierte theoretische Konstruktion der Versuchseinrichtung.
\section{Methodik}
Zu Beginn muss erw\"{a}hnt werden, dass die Aufgaben nicht in der Gruppe bearbeitet wurden, sondern in regelm\"{a}{\ss}igen Treffen der Gruppe besprochen und aufgeteilt wurden. Dabei wurden Milestones gesetzt, welche es termingerecht zu erreichen galt. Die erledigten Aufgaben wurden dann diskutiert und verfeinert. Anschlie{\ss}end wurden die neuen Aufgaben festgelegt, aufgeteilt und neue Milestones zu den Aufgaben gesetzt. Im Folgenden werden wir nicht weiter auf die Aufteilung der Aufgaben eingehen, da es f\"{u}r das Projekt keine Rolle spielt, wer die Aufgaben erledigt hat, sondern es von gr\"{o}{\ss}erer Relevanz ist, wie die einzelnen Aufgaben angegangen und bearbeitet wurden.
\begin{enumerate}
\item \textbf{Einarbeitung in die verschiedenen Thematiken}:\\
F\"{u}r die Einarbeitung wurde haupts\"{a}chlich Literaturrecherche betrieben. Dabei wurde Literatur aus der Bibliothek der TU Darmstadt verwendet, welche von verschiedenen Professoren und wissenschaftlichen Mitarbeitern empfohlen wurde. Eine weitere Quelle waren die wissenschaftlichen Ver\"{o}ffentlichen der Arbeiten am Windkanal in Darmstadt und des Fachgebietes Fahrzeugtechnik an der TU Darmstadt. Auch das Internet und digitale Archiv der ULB Darmstadt wurden als Quellen f\"{u}r die Einarbeitung genutzt.
\item \textbf{Ausarbeitung einer Anforderungsliste}:\\
F\"{u}r die Erstellung einer Anforderungsliste wurde die gesamte geplante Versuchseinrichtung in verschiedene Teile unterteilt. Dabei gab es drei wichtige Aspekte der Einrichtung zu beachten. Welche Anforderungen werden aufgrund der verwendeten Messger\"{a}te vorgegeben, welche Anforderungen werden durch die Auswahl eines passenden Partikelsubstrates vorgegeben und welche Anforderungen entstehen f\"{u}r die eigentlichen Bauteile der Versuchseinrichtung. Dabei wurden einmal physikalische, sowie allgemeine Anforderungen an die gesamte Versuchsdurchf\"{u}hrung gestellt.
\item \textbf{Entwicklung von Versuchseinrichtungskonzepten}:\\
F\"{u}r die Entwicklung von Konzepten f\"{u}r verschiedene Versuchseinrichtungen wurden alle ben\"{o}tigten Elemente der Einrichtung in einem morphologischen Kasten erstellt. Dabei wurde sich bei der L\"{o}sungsfindung an die VDI-Richtlinien \textit{2222} und \textit{2225} gehalten.\\
VDI-Richtlinie \textit{2222} beschreibt das methodische Vorgehen bei der Konstruktion von Konzepten, bei dem Probleml\"{o}sungen und die genaue Aufgabenstellung f\"{u}r Dritte nachvollziehbar hergeleitet werden. Dadurch ergeben sich Vorteile in der Wiederverwendung von Teill\"{o}sungen. Die VDI-Richtlinie \textit{2225} beschreibt das Entwickeln von vollst\"{a}ndig technischen Produkten, welche in technischer Funktionalit\"{a}t und Wirtschaftlichkeit lange konkurrenzf\"{a}hig bleiben.\\
Die ben\"{o}tigten Elemente ergaben sich aus der Aufgabenstellung und der Anforderungsliste. Durch Brainstorming wurde der morphologische Kasten mit verschiedenen Ideen zu den wichtigsten Teilelementen gef\"{u}llt. Aus den Einzelteilen entstanden in mehreren Durchl\"{a}ufen verschiedene Konzeptideen, welche skizziert und kurz beschrieben wurden. Dabei unterschieden sich die Konzepte in einem wichtigen Punkt, n\"{a}mlich der Art, wie der Partikelsprung erzeugt wird. Die Konzepte wurden nach ihrer Entstehung mit wissenschaftlichen Mitarbeitern besprochen und verfeinert, wobei manche Konzepte in diesem Schritt bereits von der Liste gestrichen wurden.
\item \textbf{Vergleich der Konzepte}:\\
Mit der Anforderungsliste als Referenz wurden die \"{u}brig gebliebenen Konzepte verglichen. Dabei haben die Konzepte Punkte in jeder Anforderung bekommen. Je nach Wichtigkeit der Anforderung wogen hier die Punktzahlen mehr oder weniger. Am Ende wurde das Konzept ausgew\"{a}hlt welches, den Anforderungen am besten standhalten konnte.
\item \textbf{Auswahl und Konstruktion eines Konzeptes}:\\
Das ausgew\"{a}hlte Konzept wurde dann als detaillierte Skizze angefertigt. Nun wurden auch erste konkrete \"{U}berlegungen zu Ma{\ss}en und Materialien gemacht, um eine endg\"{u}ltige Konstruktion zu entwerfen. Die Versuchseinrichtung wurde in Einzelteile zerlegt und es wurden konkrete Bauteile gefunden oder entwickelt, um das bisher theoretische Konzept zu verwirklichen. Ma{\ss}e, Materialien, Herkunft und Preis der Teile wurden ermittelt und festgelegt, sodass ein vollst\"{a}ndiger Bauplan des Konzeptes vorlag.
\end{enumerate}
\section{Messger\"{a}te}
Bei der Konzeption der Versuchsaufbauten sollte anfangs darauf geachtet werden, dass mit diesen verschiedene Messger\"{a}te unterschiedlicher Hersteller getestet werden k\"{o}nnen. Im Verlauf des Projekts wurde zur Vereinfachung des Konzeptentwurfs nach Absprache mit dem Auftraggeber sich darauf geeinigt, sich auf die Messger\"{a}te \textbf{Fast Mobility Particle Sizer Spectrometer Model 3091} (kurz FMPS 3091) und \textbf{Optical Particle Sizer Model 3330} (kurz OPS 3330) der Firma \textbf{TSI} zu konzentrieren. 
\\
Es bestand dennoch weiterhin die Schwierigkeit geeignete Anforderungen zu identifizieren, die die Versuchsaufbauten erf\"{u}llen m\"{u}ssen, um sie zumindest auf die Ger\"{a}te FMPS 3091 und OPS 3330 anwenden zu k\"{o}nnen. Die Gr\"{u}nde hierf\"{u}r liegen in ihren unterschiedlichen Leistungen und Funktionsrinzipien. Abbildung \ref{fig:verfahren} zeigt einen \"{U}berblick \"{u}ber die heutigen verf\"{u}gbaren Messverfahren. Die Wahl eines Verfahrens h\"{a}ngt unter anderem davon ab, welche Partikelgr\"{o}{\ss}en gemssen werden sollen. 
\begin{figure}[H]
	\myfloatalign
	{\includegraphics[width=.6\linewidth]{gfx/measuring_dev/partikelmessverfahren.pdf}} \quad
	\caption[\"{U}bersicht \"{u}ber gebr\"{a}uchliche Partikelmessverfahren\cite{reinraum}]
	{\"{U}bersicht \"{u}ber gebr\"{a}uchliche Partikelmessverfahren\cite{reinraum}}
	\label{fig:verfahren}
\end{figure}
\newpage
Optische Partikelz\"{a}hler arbeiten nach dem physikalischen Prinzip, dass beleuchtete Partikel das Licht aus seiner urspr\"{u}nglichen Ausbreitungsrichtung ablenken (Abbildung \ref{fig:optischer_messer}). Bei diesen passiert jeder einzelne Partikel ein Messvolumne, in dem er von Licht bestrahlt wird und dann mit Hilfe eines Detektors die Intensit\"{a}t des gestreuten Lichts und die Anzahl der gestreuten Lichtimpulse gemessen. Die Intensit\"{a}t l\"{a}sst Aufschl\"{u}sse \"{u}ber die Partikelgr\"{o}{\ss}e ziehen, bei einem bekannten Volumenstrom und vordefinierter Messdauer l\"{a}sst sich \"{u}ber die Anzahl der Lichtimpulse die Partikelkonzentration bestimmen. Bei moderneren Ger\"{a}ten wie dem OPS 3330 geschieht die optische Messung mit Hilfe eines Lasers (Abbildung \ref{fig:laser_schwert}). Hierbei kommt zus\"{a}tzlich ein Parabolspiegel hinzu, mit dem das gestreute Licht von einem Detektor gesammelt wird. Bei diesem Messprinzip ist es wichtig, dass jeder Partikel das Messvolumen einzeln durchl\"{a}uft. Falls zwei oder mehr Partikel das Messvolumen gleichzeitig durchlaufen, werden sie als ein Gro{\ss}es gewertet. Dieser Effekt wird als Koinzidenz bezeichnet. Es existieren zwei Methoden, um diesen Effekt zu vermeiden. Die erste Methode arbeitet aerodynamisch, bei welcher der Aerosolstrahl von einem Mantel aus reiner Luft umgeben ist und mittels einer Ansaugd\"{u}se als d\"{u}nner Strahl austritt (Abbildung \ref{fig:aerodynamisch_trennung}).\\
Die zweite Methode arbeitet rein optisch mit Hilfe von zwei Blenden, die den Bereich des zu empfangenden Lichts einschr\"{a}nken (Abbildung \ref{fig:optisch_trennung}).

\begin{figure}[H]
	\myfloatalign
	{\includegraphics[width=.5\linewidth]{gfx/measuring_dev/optischer_messer.pdf}} \quad
	\caption[Funktionsprinzip eines optischen Partikelz\"{a}hlers (Quelle: \cite{reinraum}, S.75)]
	{Funktionsprinzip eines optischen Partikelz\"{a}hlers\cite{reinraum}}
	\label{fig:optischer_messer}
\end{figure}

\begin{figure}[H]
	\myfloatalign
	\subfloat[Seitenansicht]
	{\includegraphics[width=.45\linewidth]{gfx/measuring_dev/aerodynamisch_seite.pdf}} \quad
	\subfloat[Von-Oben Ansicht]
	{\includegraphics[width=.45\linewidth]{gfx/measuring_dev/aerodynamisch_oben.pdf}}
	\caption[Messvolumentrennung durch aerodynamische Fokussierung\cite{reinraum}]
	{Messvolumentrennung durch aerodynamische Fokussierung\cite{reinraum}}
	\label{fig:aerodynamisch_trennung}
\end{figure}

\begin{figure}[H]
	\myfloatalign
	\subfloat[Seitenansicht]
	{\includegraphics[width=.45\linewidth]{gfx/measuring_dev/optisch_seite.pdf}} \quad
	\subfloat[Von-Oben Ansicht]
	{\includegraphics[width=.45\linewidth]{gfx/measuring_dev/optisch_oben.pdf}}
	\caption[Messvolumentrennung mit Hilfe von Blenden\cite{reinraum}]
	{Messvolumentrennung mit Hilfe von Blenden\cite{reinraum}}
	\label{fig:optisch_trennung}
\end{figure}

Elektrische Pertikelz\"{a}hler, wie der FMPS 3091 hingegen ben\"{o}tigen keine speziellen Techniken, um Partikel einzeln zu erfassen. Beim FMPS 3091 werden die Partikel positiv aufgeladen und anschlie{\ss}end mit einem reinen Luftstrom in das Messvolumen gef\"{u}hrt, in welchem sich eine Elektrode befindet und ein elektrisches Feld erzeugt. Elektrometer, welche um die Eletrode angebracht sind und verschieden empfindlich sind, messen die geladenen Partikel (Abbildung \ref{fig:laser_schwert}).

\begin{figure}[H]
	\myfloatalign
	\subfloat[Elektrische Messung beim FMPS 3091\cite{fmps_3091}]
	{\includegraphics[width=.45\linewidth]{gfx/measuring_dev/fmps_3091.pdf}} \quad
	\subfloat[Lasermessung beim OPS 3330\cite{ops_3330}]
	{\includegraphics[width=.45\linewidth]{gfx/measuring_dev/ops_3330.pdf}}
	\caption[Messverfahren des FMPS und OPS]
	{Messverfahren des FMPS und OPS}
	\label{fig:laser_schwert}
\end{figure}

Ein wesentlicher Leistungsunterschied zwischen dem FMPS 3091 und OPS 3330 ist das Volumen des Aerosol-Tr\"{a}gergasgemisch, welches von beiden im Verh\"{a}ltnis 1:4 angesaugt wird. Der OPS arbeitet mit $5 L/min$, w\"{a}rend der FMPS mit $50 L/min$ arbeitet. Ein Teil des Luftgemisches wird dabei vom Ger\"{a}t gefiltert und dazu genutzt, den Aerosolstrom zum Messvolumen zu f\"{u}hren.
\section{Anforderungen}
Um sicherstllen zu k\"{o}nnen, dass das Projektziel bestm\"{o}glicht erreicht wird, wurde eine Anforderungsliste erstellt, mit deren Hilfe die G\"{u}te der Umsetzung des Versuchskonzepts gemessen werden kann. Diese Liste enth\"{a}lt die Anforderungen des Auftraggebers und unseren eigenen, welche sich aus den zu Grunde liegenden technischen Dokumenationen der verwendeten Messger\"{a}te und den VDI Richtlinien f\"{u}r Reinraumtechnik ergeben haben.

Bei den festgelegten Anforderungen war uns eine genaue Quantifizierung der Werte wichtig, wann immer es m\"{o}glich war. Bei den Anforderungen wird zwischen den folgenden Arten unterschieden:

\begin{itemize}
	\item \textbf{Festforderung} (FF): Es muss ein bestimmter Wert aufweisbar sein und Abweichungen davon sind nicht zul\"{a}ssig.
	
	\item \textbf{Bereichsforderung} (BF): Die Anforderung muss einen Wert aufweisen und dieser darf nur innerhalb eines bestimmten Bereich liegen.
	
	\item \textbf{Mindestforderung} (MF): Bei einer Mindestforderung muss die Anforderung mindestens einen angegeben Wert aufweisen.
	
	\item \textbf{Zielforderung} (ZF): Der Wert einer Anforderung muss bei einer ZF m\"{o}glichst nah an einem Zielwert liegen, um gut bewertet werden zu k\"{o}nnen.
	
	\item \textbf{Wunsch} (W): Da es sich um einen Wunsch handelt, ist die Wichtigkeit des Merkmals vergleichsweise gering. Die Anforderung sollte dann allerdings einen bestimmten Wert aufweisen und wird bei der Bewertung positiv ber\"{u}cksichtigt. 
\end{itemize} 

Die Anforderungen lassen sich in die Kategorien \textbf{Forderung} und \textbf{Wunsch} einteilen, besonders wichtig ist hierbei eine genaue Quantifizierung der einzelnen Anforderungen wann immer es m\"{o}glich ist. Forderungen m\"{u}ssen auf jeden Fall von allen Versuchsaufbaukonzepten eingehalten werden, um das Ziel erreichen zu k\"{o}nnen. W\"{u}nsche hingegen k\"{o}nnen, m\"{u}ssen aber nicht von den Versuchsaufbauten erf\"{u}llt werden.

F\"{u}r die Generierung eines Partikeltestsignals in Form einer Sprungfunktion lassen sich die folgenden Anforderungen festlegen, die sich in drei Kategorien unterteilen lassen:
\begin{enumerate}
	\item Aerosol: Anforderungen an das Pr\"{u}faerosol, welches dem zu testenden Messger\"{a}t zugef\"{u}hrt wird
	\item Aerosolquelle-Messger\"{a}t-Schnittstelle: Anforderungen an die Schnittstelle, welche die Aerosolquelle mit dem Messger\"{a}t verbindet
	\item Versuchsaufbau: Anforderungen an den gesamten Versuchsaufbau inklusive der verwendeten Ger\"{a}te
\end{enumerate}

Die Anforderungsliste ist in Tabelle \ref{anforderungsliste} zu finden. In den nachfolgenden Abschnitten werden die einzelnen Anforderungen genauer erkl\"{a}rt.

\begin{table}
	\begin{center}
		\begin{tabular}{| l | l | l | l | l | l |}
			\hline
			Anforderung& Nr. & Art & Wert & Einheit & Quelle\\
			\hline
			Druck am  & 1 & BF & Eingang - Ausgang & $kPa$ & Datenbl\"{a}tter\\
			Messger\"{a}tilnet& & & $70 - 103$ (FMPS) & &\\
			& & & $40 - 103$ (APS) & &\\
			& & & $<0.746$ (OPS) & &\\
			& & & $75-105$ (UCPC) & &\\
			\hline
			Einstellzeit der & 2 & ZF & 0 & $s$ &selbstgew"{a}hlte\\
			Partikelanzahlkonzentration & & & & &Last\\
			der Aerosolquelle & & & & &\\
			\hline
			Tr\"{a}gergas& 3 & FF & & & Datenbl\"{a}tter\\
			- gefilterte Luft & & & & &\\
			- Inertgas bei UCPC & & & & &\\
			\hline
			Volumenstrom & 4 & FF & 10 (FMPS) & $l/min$ & Datenbl\"{a}tter\\
			(Tr\"{a}gergas) & & & $5 \pm 0.2$ (APS)& &\\
			& & & $1 \pm 0.05$ (OPS)& &\\
			& & & $1.5 \pm 0.05$ (UCPC)& &\\
			\hline
			Verianzaufkl\"{a}rung& 5 & MF & $>90$ & \% & selbstgew\"{a}hlte\\
			& & & & & Last\\
			\hline
			Tr\"{a}gergastemperatur & 6 & BF & $283.15-325.15$(FMPS) & $K$ & Datenbl\"{a}tter\\
			& & & $283.15-313.15$(APS) & &\\
			& & & $273.15-318.15$(OPS) & &\\
			& & & $283.15-308.15$(UCPC) & &\\
			\hline
		\end{tabular}
	\end{center}
	\caption{Anforderungsliste f\'{u}r Aerosolquellen}
	\label{anforderung_aerosolquelle}
\end{table}	


\begin{table}
	\begin{center}
		\begin{tabular}{| l | l | l | l | l | l |}
			\hline
			Partikelanzahlkonzentration & 7 & BF & $100-10^{7}$ (FMPS)& $n/cm^{3}$ & Datenbl\"{a}tter\\
			& & & $0-1000$ (APS) & &\\
			& & & $0-3000$ (OPS) & &\\
			& & & $0-3\cdot10^{5}$ (UCPC) & &\\
			\hline
			Partikelmaterial (Messger\"{a}t) & 8 & W &  &  & Datenblatt APS\\
			\hline
			Partikelgr\"{o}{\ss}e & 9 & BF & $5.6-560$ (FMPS) & $nm$ & Datenbl\"{a}tter\\
			& & & $500-20000$ (APS) & &\\
			& & & $300-10000$ (OPS) & &\\
			& & & $2.5-3000$ (UCPC) & &\\
			\hline
		\end{tabular}
	\end{center}
	\caption{Anforderungsliste f\'{u}r Aerosole}
	\label{anforderung_aerosol}
\end{table}			
			
\begin{table}
	\begin{center}
		\begin{tabular}{| l | l | l | l | l | l |}
			\hline			
			Schaltzeit & 10 & ZF & 0 & s & selbstgew\"{a}hlte\\
			& & & & & Last\\
			\hline
			Abs\"{a}tze, abrupte Quer- & 11 & ZF & 0 & n & VDI 3491\\
			schnitts\"{a}nderungen, & & & & &\\
			Str\"{o}mungsumlenkungen & & & & &\\
			\hline
		\end{tabular}
	\end{center}
	\caption{Anforderungsliste f\'{u}r Schaltvorrichtungen}
	\label{anforderung_schaltvorrichtung}
\end{table}			
			
			
\begin{table}
	\begin{center}
		\begin{tabular}{| l | l | l | l | l | l |}
			\hline			
			Weg zwischen Umschaltvorrichtung & 12 & ZF & 0 & m & selbstgew\"{a}hlte\\ 
			und Messger\"{a}teinlass & & & & & Last\\
			\hline
			laminarer Strom & 13 & MF & $< 2300$ & dimensionslos & VDI 3491\\
			& & ZF & $1500$ & &\\
			\hline
			Scnittstelle zum & 14 & FF & 9.5 (FMPS) & $mm$ & Datenbl\"{a}tter\\
			Messger\"{a}t (Inlet) & & & 19 (APS) & &\\
			& & & 6.35 (OPS) & &\\
			& & & 6.4 (UCPC) & &\\
			\hline
		\end{tabular}
	\end{center}
	\caption{Anforderungsliste f\'{u}r Aerosolleitungen}
	\label{anforderung_aerosollitung}
\end{table}			
			
			
\begin{table}
	\begin{center}
		\begin{tabular}{| l | l | l | l | l | l |}
			\hline			
			minimale Versuchszeit& 15 & ZF & 3300 & $s$ & selbstgew\"{a}hlte\\
			& & & & & Last\\
			\hline
			Kosten & 16 & ZF & 0 & Euro & Festlegung \\
			& & BF & $0-1000$ & & durch FZD\\
			\hline
			Anzahl Messger\"{a}te& 17 & MF & $\geq2$ & n & Festlegung\\
			& & & & & durch FZD\\
			\hline
		\end{tabular}
	\end{center}
	\caption{Anforderungsliste f\'{u}r Versuchsaufbauten}
	\label{anforderung_versuchsaufbau}
\end{table}

\subsection{Messger\"{a}te}
Bei der Konzeption der Versuchsaufbauten sollte anfangs darauf geachtet werden, dass mit diesen verschiedene Messger\"{a}te unterschiedlicher Hersteller getestet werden k\"{o}nnen. Im Verlauf des Projekts wurde zur Vereinfachung des Konzeptentwurfs nach Absprache mit dem Auftraggeber sich darauf geeinigt, sich auf die Messger\"{a}te \textbf{Fast Mobility Particle Sizer Spectrometer Model 3091} (kurz FMPS 3091) und \textbf{Optical Particle Sizer Model 3330} (kurz OPS 3330) der Firma \textbf{TSI} zu konzentrieren. Einige der quantifizierten Werte sind daher aus den Handb\"{u}chern f\"{u}r die Anforderungen entnommen worden.

Beide Messger\"{a}te saugen sich ihre ben\"{o}tigten Aerosol-Luft-Gemisch Mengen \"{u}ber interne Pumpen an. Beim FMPS 3091 betr\"{a}gt dies insgesamt $50 L/min$, wovon $10 L/min$ auf das Aerosol und $40 L/min$ auf Luft entfallen, w\"{a}hrend der OPS 3330 jeweils $1 L/min$ Luft und Aerosol ben\"{o}tigt. Hauptunterschied zwischen den beiden Ger\"{a}ten sind ihre Mechanismen, mit denen Partikel detektiert werden. Der FMPS 3091 misst die elektrostatische Ladung einzelner Partikel, w\"{a}hrend der OPS 3330 ein optisches Verfahren anwendet. Weitere Unterschiede liegen in der Gr\"{o}{\ss}enverteilung der Partikel, bei denen die jeweiligen Messger\"{a}te noch korrekt funktionieren und der Partikelkonzentration. 

Aus diesen und weiteren Werten ergeben sich die nachfolgenden Anforderungen, die die Versuchsaufbauten zu erf\"{u}llen haben. Hierbei wurde immer versucht eine Schnittmenge zwischen den Wertebereichen der Messger\"{a}te zu finden nach der sich die Anforderungen richten um so garantieren zu k\"{o}nnen, dass die Pr\"{u}fst\"{a}nde auf mehrere Messger\"{a}te anwendbar sind.   

\subsection{Aerosolquelle}
Die Aerosolquelle liefert mit dem Aerosol, den entscheidenden Faktor zum Partikelanzahlkonzentrationssprungs. Aus den Arbeitsbereichsbegrenzungen der Messger\"{a}te sowie allgemeiner Richtlinien zur Aerosolgenerierung ergeben sich folgende Anforderungen:

\begin{itemize}
\item 1. Die internen Pumpen, welche den Volumenstrom der Messger\"{a}te regulieren, arbeiten nur in einem bestimmten Druckbereich korrekt (Quelle: Datenbl\"{a}tter). Dieser Druckbereich begrenzt wiederum den Aerosoldruck, welcher am Messger\"{a}t-Einlass herrschen darf. Es ist darauf zu achten, dass die Aerosolquelle diesen Druckbereich einh\"{a}lt. 
Auch das Design der Aerosolleitungen und der Schaltvorrichtung beeinflussen den Druck am Messgerät-Einlass (Quelle: TSL-Skript).

\item 2. Die Einstellzeit einer Leistungskenngr\"{o}{\ss}e beschreibt das dynamische Verhalten der Aerosolquelle bez\"{u}glich dieser Gr\"{o}{\ss}e bei sprungf\"{o}rmigen \"{A}nderungen eines Einstellparameters. Dabei ist vor allem die Einstellzeit der Partikelanzahlkonzentration für den Versuchsaufbau interessant.(Quelle: VDI 3491)

\item 3. Das Tr\"{a}gergas der Aerosole hat einen gro{\ss}en Einfluss auf das Messverhalten der Messger\"{a}te.  Für den Versuchsaufbau kommt vor allem gereinigte Luft als Tr\"{a}gergas in Frage, da diese auch bei der Messung der Bremspartikel als Tr\"{a}gergas fungiert und die ausgew\"{a}hlten Messger\"{a}te mit Luft oder Inertgas (UCPC) arbeiten. (Quelle: Datenbl\"{a}tter)

\item 4. Die Messger\"{a}te arbeiten mit einem fest vorgegebenen Volumenstrom des Tr\"{a}gergases.  Es muss gew\"{a}hrleistet sein, dass die Aerosolquelle einen entsprechenden Strom bereitstellen kann. Sollte der Volumenstrom den Anforderungen nicht entsprechen, generieren die Messger\"{a}te eine Fehlermeldung und liefern keine Messwerte mehr. (Handb\"{u}cher)

\item 5. Varianzaufkl\"{a}rung ist ein Maß Für die G\"{u}te eines Mathematischen Modells. Sie sagt aus inwieweit dieses Modell die Streuung von Messdaten erkl\"{a}ren kann. In dem Versuchsaufbau trifft sie eine Aussage über die Abh\"{a}ngigkeit der ben\"{o}tigten Anzahl der Messwerte und der Varianz der Partikelanzahkonzentratioslwerte des Stromes.

\item 6. Die Tr\"{a}gergastemperatur kann je nach Aerosolquelle verschiedene Werte annehmen. Da die Temperatur des Aerosols den Messprozess und die Eigenschaften des Aerosolstromes beeinflusst, geben die Messger\"{a}tehersteller feste Temperaturgrenzen vor zwischen welchen sich diese Temperatur bewegen darf.(Datenbl\"{a}tter)
\end{itemize}

\subsection{Aerosol}
Folgende Anforderung an das Aerosol ergeben sich aus den Arbeitsbereichsbegrenzungen der Messger\"{a}te und aus der statistischen Verwertbarkeit der Messwerte zur Berechnung des zeitlichen Verhaltens der Messger\"{a}te:

\begin{itemize}
\item 7. Eine zu hohe Partikelanzahlkonzentration führt zu einer H\"{a}ufung von falschen Messergebnissen zum Beispiel durch Koinzidenz bei optischen Sensoren. Messergebnisse bei einer zu niedrigen Partikelanzahlkonzentration k\"{o}nnen bei Messger\"{a}ten, welche mit elektrischen Feldern arbeiten, in dem herrschenden, systematischen Grundrauschen untergehen(Quelle: Handb\"{u}cher). Somit muss f\"{u}r verwertbare Ergebnisse die Partikelanzahlkonzentration des Aerosols mit dem benutzten Messgerät abgeglichen werden.

\item 8. Partikelmaterialien k\"{o}nnen Gesundheits- oder Umweltgef\"{a}hrdend sein oder aufgrund von ihren Eigenschaften f\"{u}r bestimmte Messverfahren nicht geeignet sein.(VDI 3491) Deswegen ist es wichtig das benutzte Partikelmaterial zu kennen. Da die hier benutzten Messgeräte vor allem für die Feinstaubmessung benutzt werden, bieten sich zum Beispiel feste Materialien für die Aerosole an.

\item 9. Zu kleine Partikel k\"{o}nnen von den Messsystemen der Partikelz\"{a}hler nicht erfasst werden, w\"{a}hrend zu gro{\ss}e Partikel durch Verstopfen der Filter die Funktion beeintr\"{a}chtigen k\"{o}nnen (Datenbl\"{a}tter). Manche Hersteller geben an, dass zu große Partikel zwar gez\"{a}hlt werden, aber nicht mehr in die Gr\"{o}{\ss}enverteilung mit eingehen.(Datenblatt FMPS) Trotzdem sollte zur Sicherheit ein Zyklonabscheider dem System zugeschaltet werden, wenn nicht garantiert werden kann, dass nicht zu viele zu gro{\ss}e Partikel im Aerosol sind.
\end{itemize}

\subsection{Schaltvorrichtung}
Die Schaltvorrichtung ist das entscheidende Bauteil beim Erzeugen der Sprungfunktion. Dabei sind ein schneller Schaltprozess aber auch ein str\"{o}mungsg\"{u}nstiges Design für die G\"{u}te der Schaltvorrichtung die ausschlaggebenden Kriterien. Auch das Material kann hierbei eine Rolle spielen. Folgende Anforderungen lassen sich definieren:

\begin{itemize}
	\item 10. DIe Zeit, die die Vorrichtung ben\"{o}tigt um von einem Schaltzustand in den anderen zu wechseln, sollte so kurz wie m\"{o}glich sein. Diese Schaltzeit tr\"{a}gt zur Totzeit des Versuchsaufbaus bei. 
	
	\item 11. Hochwinklige Umlenkungen und abrupte Querschnitts\"{a}nderungen k\"{o}nnen zu Partikel- und Druckverlusten f\"{u}hren. Somit k\"{o}nnen sie die G\"{u}te der Partikelfront beeinflussen und sollten deswegen vermieden werden. Außerdem kann es bei ung\"{u}nstigem Design der Schaltvorrichtung zu Turbulenzen in ihrem Inneren kommen, was die Berechenbarkeit der Totzeit des Aufbaus erheblich erschwert. (HIwis vom Windkanal) Die vorhergehenden \"{U}berlegungen gelten auch für die Auslegung der Aerosolleitung.
\end{itemize}

\subsection{Aerosolleitungen}
Auch bei der Leitung des Aerosols, ist auf eine str\"{o}mungsg\"{u}nstige Auslegung zu achten. Dies betrifft die Ausma{\ss}e, das Material, den Zustand der Innenfl\"{a}chen und die r\"{a}umliche F\"{u}hrung der einzelnen Verbindungen. Die folgenden Anforderungen k\"{o}nnen also auch als Designkriterien angesehen werden:

\begin{itemize}
\item 12. Der Weg zwischen der Umschaltvorrichtung und dem Messger\"{a}teeinlass f\"{u}hrt bei gegebenen Volumenstr\"{o}men, abh\"{a}ngig von dem benutzten Messger\"{a}t, zu einer Verz\"{o}gerungszeit, die sich in der Totzeit des Versuchsaufbaus niederschl\"{a}gt. Da das Str\"{o}mungsverhalten in den Verbindungen nur unter Annahme von Vereinfachungen zutrifft, gilt: Umso l\"{a}nger der Weg ist, umso gr\"{o}{\ss}er ist der Rechenfehler bei der Berechnung der Totzeit. Außerdem f\"{u}hren Druck- und Partikelverluste auf diesem Weg zu einer Verzerrung der Partikelfront. (VDI 3491)

\item 13. Da eine turbulente Str\"{o}mung zu einem analytisch kaum berechenbaren Verhalten der Str\"{o}mung f\"{u}hrt, ist der Versuchsaufbau so zu gestalten, dass sich die Str\"{o}mung m\"{o}glichst \"{u}berall laminar verh\"{a}lt.(VDI 3491) Vor allem am Messger\"{a}te- Einlass, wo es zu einem Querschnitts\"{u}bergang kommen kann, sollten entsprechend gestaltete Adapter daf\"{u}r Sorge tragen.

\item 14. Die Messger\"{a}te besitzen einen Aerosoleinlass in Form eines R\"{o}hrchens mit unterschiedlichen Durchmessern, welches in die Umgebung ragt (Datenbl\"{a}tter/Handb\"{u}cher). Je nach Versuchsaufbau k\"{o}nnen Adapter von N\"{o}ten sein, um eine direkte Verbindung zwischen Schaltvorrichtung und Messger\"{a}t zu gew\"{a}hrleisten. Bei der Gestaltung dieser Adapter muss ein Kompromiss zwischen Anforderung 12 und 13 eingegangen werden.
\end{itemize}

\subsection{Versuchsaufbau}
Folgende Anforderungen betreffen den Versuchsaufbau als ganzes:

\begin{itemize}
\item 15. Die minimale Versuchszeit ist wichtig in der Versuchsplanung, da durch die Messaufl\"{o}sung der Messger\"{a}te von einer Sekunde, die Anzahl der Messwerte je nach Versuchsaufbau begrenzt ist. Nun kann eine Versuchsdurchf\"{u}hrung aus mehreren Abl\"{a}ufen (abwechselnd: gereinigte Luft (zb. 5 s) -- Aerosol(zb. 30s)) und die Abl\"{a}ufe wiederum aus mehreren Messungen (der Partikelanzahl) bestehen. Je nach Varianz der Partikelanzahl, sind eine bestimmte Anzahl von Messungen und Messabl\"{a}ufen von N\"{o}ten um ein valides Ergebnis für die Zeitkonstante zu erhalten. So sollte der Aufbau ca. 100 Abl\"{a}ufe mit je 30 Partikelanzahlmessungen (+ 5 Sekunden gereinigte Luft pro Ablauf) mit reproduzierbaren Ergebnissen liefern k\"{o}nnen.

\item 16. Aus wirtschaftlicher Sicht, sind die Kosten ein ausschlaggebender Faktor für die G\"{u}te des Aufbaus. Sie setzen sich aus den Materialkosten selbstgestalteter und eventuell selbstgebauter Teile, den Kosten für Zukaufteile, wie Filter, Verdichter, Aerosolgeneratoren etc. und den Kosten f\"{u}r Ressourcen, wie Strom, Aerosol etc., zusammen. Die maximalen Kosten d\"{u}rfen die vorgegebenen Mittel des Fachbereiches nicht \"{u}berschreiten. (Fachbereich Fahrzeugdynamik)

\item 17. Ziel des Aufbaus ist die Evaluation der Dynamik von Messger\"{a}ten, welche f\"{u}r Messungen von Bremspartikeln am Schwungmassenpr\"{u}fstand ben\"{o}tigt werden. Von den f\"{u}nf f\"{u}r diesen Zweck ausgesuchten Messger\"{a}ten, wurden vor allem der FMPS 3091 und der OPS 3330 f\"{u}r geeignet befunden(Fachbereich Fahrzeugdynamik). Nun sollen die, durch den hier konstruierten Versuchsaufbau, ermittelten Werte zumindest f\"{u}r diese beiden Ger\"{a}te ein valides Ergebnis darstellen. Jedes weitere Messger\"{a}t, das durch den Aufbau erfolgreich getestet werden kann, erh\"{o}ht dessen wirtschaftlichen und wissenschaftlichen Wert.
\end{itemize}

%*****************************************
\chapter{Technische Grundlagen}\label{ch:foundations}
%*****************************************
F\"{u}r die Arbeit wurden verschiedene technische Grundlagen verwendet, welche im Folgenden kurz erl\"{a}utert werden sollen. Hierf\"{u}r wird auf allgemeine Elemente der Str\"{o}mungslehre, sowie auf Einwirken der Reynoldszahl auf aktive Str\"{o}mungen hingewiesen. Da die Arbeit sich um die Messung von Bremsemissionspartikeln dreht, werden auch die Eigenschaften von Partikeln und Aerosolen, die f\"{u}r die Arbeit verwendet wurden kurz erkl\"{a}rt. Letztendlich werden die mechanischen Grundlagen f\"{u}r verschiedene verwendete Bauteile erl\"{a}utert und der aktuelle Stand der Technik im Bereich Aerosolstr\"{o}mung und Messtechnik aufgegriffen.

\section{Str\"{o}mungslehre}
Da es Ziel der Arbeit ist, einen konstanten Strom von Partikeln durch ein bestimmtes Medium zu generieren, um letztendlich ein Partikeltestsignal zu bekommen, ist es wichtig grundlegende physikalische Eigenschaften von Str\"{o}mungen zu kennen. Deshalb ist es unumg\"{a}nglich sich im Vorfeld mit den Grundlagen der Str\"{o}mungslehre auseinanderzusetzen.
\\\\
Die Str\"{o}mungslehre ist die Wissenschaft vom physikalischen Verhalten von Fluiden und Gasen. Die in dieser Lehre gewonnenen Kenntnisse sind Gesetzm\"{a}{\ss}igkeiten in Str\"{o}mungsvorg\"{a}ngen und dienen der L\"{o}sung von Str\"{o}mungsproblemen. Dabei liegt der Fokus auf den Problemen bei umstr\"{o}mten oder durchstr\"{o}mten Bauteilen. Gegenstand der Str\"{o}mungslehre sind die Bewegungen von Fluiden, Gasen und ruhenden, flie{\ss}enden oder str\"{o}menden Substanzen. Die Str\"{o}mungslehre l\"{a}sst sich in verschiedene Teilgebiete unterteilen, von denen allerdings f\"{u}r diese Arbeit nur die Fluiddynamik relevant ist und die Auslegungen sich daher auf dieses Teilgebiet beschr\"{a}nken werden\cite{stream}.

\subsection{Str\"{o}mungseigenschaften}
Grob betrachtet gibt es zwei wichtige Eigenschaften von Str\"{o}mungen, die f\"{u}r diese Arbeit wichtig sind. Auf der einen Seite gibt es die laminaren Str\"{o}mungen, die man sich als ein gleichm\"{a}{\ss}iges Flie{\ss}en vorstellen kann, auf der anderen Seite als Gegensatz die turbulenten Str\"{o}mungen.
\\\\
In der Fluiddynamik ist die laminare Str\"{o}mung eine Bewegung des Fluids ohne sichtbare Turbulenzen. Dabei str\"{o}mt das Fluid in Schichten, die sich nicht miteinander vermischen. Verwirbelungen treten erst mit h\"{o}heren Str\"{o}mungsgeschwindigkeiten auf, ab diesem Zeitpunkt ist die Str\"{o}mung turbulent.
\begin{figure}[H]
        \myfloatalign
        {\includegraphics[width=.9\linewidth]{gfx/fundamentals/laminar.jpg}} \quad
        \caption[Laminare Str\"{o}mung]
        {Laminare Str\"{o}mung}
        \label{fig:laminar}
\end{figure}
\newpage
Im Vergleich dazu ist die turbulente Str\"{o}mung die Bewegung von Fluiden, bei der Verwirbelungen auftreten. Diese Str\"{o}mungsform ist gekennzeichnet durch eine scheinbar zuf\"{a}llig variierende Komponente. Turbulenzen f\"{u}hren zu einer verst\"{a}rkten Durchmischung des Fluids mit umgebenden Gasen. Dabei hat die turbulente Str\"{o}mung eine ungeordnete und schwer vorhersagbare Struktur
\begin{figure}[H]
        \myfloatalign
        {\includegraphics[width=.9\linewidth]{gfx/fundamentals/turbulent.jpg}} \quad
        \caption[Turbulente Str\"{o}mung]
        {Turbulente Str\"{o}mung}
        \label{fig:turbulent}
\end{figure}
F\"{u}r diese Arbeit wurden verschiedene Ideen mit verschiedenen Anforderungen an die Str\"{o}mung entwickelt. Gemeinsam haben alle Konzepte, dass ein laminarer Strom des Fluids am Eingang des Messger\"{a}tes verlangt war. Bei einigen kam noch hinzu, dass das Fluid vor dem Einstr\"{o}men in das Messger\"{a}t mit Luft vermischt werden sollte. Dort ist eine gezielte turbulente Str\"{o}mung gewollt, um eine Vermischung zu generieren, die im sp\"{a}teren Aufbau allerdings wieder eine laminare Form annehmen sollte\cite{stream}.
\subsection{Reynoldszahl}
Die Reynoldszahl ist eine dimensionslose Kennzahl. Sie wird in der Str\"{o}mungslehre als Beurteilungskriterium f\"{u}r laminare Str\"{o}mungen verwendet und ist deshalb relevant f\"{u}r diese Arbeit. Die Reynoldszahl ist definiert durch:
\begin{align*}
{Re} = \frac{\rho vd}{\eta} = \frac{vd}{\nu}
\end{align*}
Dabei ist \(\rho\) die Dichte des Fluids, \(v\) die Str\"{o}mungsgeschwindigkeit des Fluids gegen\"{u}ber dem durchstr\"{o}mten K\"{o}rper und \(d\) die L\"{a}nge des K\"{o}rpers. Die kinematische Viskosit\"{a}t \(\nu\) des Fluids unterscheidet sich von der dynamischen Viskosit\"{a}t \(\eta = \nu \rho\) durch den Faktor \(\rho\). \"{U}berschreitet die Reynoldszahl einen kritischen Wert \({Re}_{krit}\), wird eine bis dahin laminare Str\"{o}mung anf\"{a}llig gegen kleinste St\"{o}rungen und aus der laminaren Str\"{o}mung wird eine turbulente Str\"{o}mung. In Bezug auf die Arbeit war es also wichtig eine Reynoldszahl zu erreichen, die dem verlangten Str\"{o}mungsbild gen\"{u}gt\cite{reynolds}.
\section{Partikeleigenschaften}
Die in dieser Arbeit verwendeten Partikel sind mineralische und organische Feststoffe, die nicht in L\"{o}sung gehen und wegen ihrer geringen Gr\"{o}{\ss}e und damit geringen Gewichts in der Schwebe gehalten oder durch geringe Bewegungen des Mediums verteilt werden. Im speziellen sind das hier Aerosolpartikel, St\"{a}ube und auch Nebel- und Rauchpartikel.\\
Die Eigenschaft von Partikeln, \"{u}ber l\"{a}ngere Zeit in Gasen transportiert werden zu k\"{o}nnen, liegt darin, dass sie sich mit abnehmendem Durchmesser immer mehr wie Gas Molek\"{u}le verhalten.\\
Partikel entstehen durch verschiedene Vorg\"{a}nge, welche ausschlaggebend f\"{u}r die Gr\"{o}{\ss}e der Partikel sind. Die kleinsten Partikel entstehen durch Verbrennungen und sind selten gr\"{o}{\ss}er als \(1 nm\). Mineralstaubpartikel hingegen gehen auf Abrieb mineralischer Stoffe zur\"{u}ck und sind meist von gr\"{o}{\ss}erer Form im \(\mu m\) Bereich.\\
Partikel k\"{o}nnen vom Menschen eingeatmet werden, dabei bleiben bis zu 10\% der Partikel im Atemtrakt h\"{a}ngen und f\"{u}hren so zu einer gro{\ss}en Belastung in Lungen und Bronchien. Da es abh\"{a}ngig von der Partikelgr\"{o}{\ss}e ist, ob und wie sch\"{a}dlich die Partikel sind, ist es wichtig durch Messungen zu erfahren welche Eigenschaften und Aufbau die Partikel haben, die durch den Bremsvorgang entstehen\cite{reinraumtechnik}.
\newpage

\subsection{Bremsemissionspartikel}
Die Emissionspartikel von Bremsen  sind Bestandteil des Bremsstaubes, welcher beim Bremsvorgang entsteht. Sie stellen ein Gemisch aus Bremsbelagabrieb und Abrieb der Bremsscheiben dar, wobei der gr\"{o}{\ss}ere Anteil auf die Bremsbel\"{a}ge f\"{a}llt, da diese wesentlich weicher sind. Kritisch beim Bremsstaub sind die Partikel von Eisen, Kupfer und Mangan, die beim scharfen Bremsen freigesetzt werden. Die toxische Wirkung der einzelnen Partikel k\"{o}nnen Entz\"{u}ndungen in den Lungenzellen verursachen.\\
Da es auf das Bremsverhalten ankommt, welche Partikel in welchen Massen entstehen, ist es wichtig die entstehenden Partikel bei verschiedenen Bremsvorg\"{a}ngen zu messen und zu analysieren. Nur dann kann man gezielt Partikel bestimmter Art und Sorte reduzieren oder abfangen. So ist zum Beispiel das krebserregende Asbest auch aus der Entwicklung von Bremsscheiben herausgenommen worden\cite{envyl}.

\subsection{Partikelmessverfahren}
Unter dem Begriff Partikelmessung wird eine Gruppe von Messverfahren zur Qualifizierung und Quantifizierung von Partikeln unterschiedlicher Natur zusammengefasst. Dabei wird meistens das Prinzip des Streulichtpartikelmessung angewendet. Bei der Streulichtpartikelmessung wird eine definierte Menge einer Probe durch einen Laserstrahl gef\"{u}hrt. Das Licht des Laserstrahls bricht sich an den Partikeln oder wird von ihnen absorbiert. Photodioden k\"{o}nnen diese Effekte messtechnisch erfassen und in ein elektrisches Signal umwandeln. Dieses Signal wird von einem Computer mit einem zuvor aufgenommenen Signal verglichen, bei dem Latexkugeln definierter Gr\"{o}{\ss}e vermessen worden sind, um Referenzwerte zu schaffen. Mit diesen Daten ist man in der Lage aufgrund von statistischen Methoden die Anzahl der Partikel in einem Kubikmeter Luft oder Gas zu ermitteln. Dieses Verfahren funktioniert allerdings nur f\"{u}r Partikel mit einer Gr\"{o}{\ss}e von unter \(25 \mu m\)\cite{reinraumtechnik}.

\subsection{Aerosole}
Ein Aerosol ist ein heterogenes Gemisch aus festen oder fl\"{u}ssigen Partikeln in einem Gas. Ein Aerosol ist ein dynamisches System und unterliegt st\"{a}ndigen \"{A}nderungen durch Kondensation von D\"{a}mpfen an bereits vorhandenen Partikeln, Verdampfen fl\"{u}ssiger Bestandteile der Partikel oder Abscheidung von Teilchen an umgebenden Gegenst\"{a}nden.\\
Aerosole k\"{o}nnen durch mechanische Zerkleinerung von Material oder durch Kondensation von Material entstehen. Mechanische Prozesse umfassen Zerreiben, Zerstoßen oder andere Zerkleinerungsprozesse von Feststoffen. Kondensation hingegen ist ein Prozess, in dem sich festes oder fl\"{u}ssiges Material aus \"{u}bers\"{a}ttigten Gasen bildet.
\\\\
Der Durchmesser von Aerosolpartikeln liegt in der Gr\"{o}{\ss}enordnung zwischen \(0,1 nm\) und \(10 \mu m\) und sie haben unterschiedliche Zusammensetzungen. Kleinste Partikel sind einzelne Molek\"{u}le, die bei Verbrennungen entstehen.
\\\\
Angewendet werden Aerosole gro{\ss}fl\"{a}chig in der Industrie als Spraybasis, zum Auftragen von Lacken, medizinische Sprays und als Industriek\"{u}hlmittel. Um f\"{u}r die Zwecke dieser Arbeit einen Partikelsprung zu erzeugen hat sich ein Aerosol als Partikeltr\"{a}ger angeboten, da die Partikel in Aerosolen gleich verteilt sind und zu laminaren Str\"{o}mungen neigen\cite{aerosole}.
\section{Mechanische Grundlagen}
In den vorgestellten Konzepten werden verschiedene mechanische Bauteile verwendet. Dabei werden einfache Teile wie Rohre, Klemmen und Schläuche als trivial angenommen und es wird nicht weiter auf die Eigenschaften dieser Teile eingegangen. Kompliziertere Bauteile sollen hier allerdings kurz erläutert werden um das Verständnis für deren Funktion zu vereinfachen und die Anwendung zu verstehen.
\\\\
Da der aufgebaute Aerosolstrom durch ein System von mehreren Rohren und Schläuchen geleitet werden muss und gegebenenfalls auch umgeleitet werden muss, werden Ventile benötigt. Ein weiteres Bauteil sind Luftfilter um die zum Mischen verwendete Umgebungsluft zu reinigen. Letztendlich werden noch Ventilatoren für einige Konzepte benötigt um dem Strom eine konstante Geschwindigkeit zu geben.

\subsection{Ventile}
Ein Ventil ist ein Bauteil zur Absperrung oder Regelung des Durchflusses von Fluiden. Innerhalb des Bauteils wird ein Verschlussteil nahezu parallel zur Strömungsrichtung des Fluids bewegt. Die Strömung kann reduziert oder unterbrochen, indem das gesamte Verschlussteil an eine passend geformte Öffnung gepresst wird. Ventile haben die Eigenschaft über den gesamten Stellbereich ein gleichmäßiges Strömungsbild zu besitzen, weshalb sie sich gut für Regelaufgaben eignen.
\\\\
Ventile lassen sich in vier Kategorien einteilen. Durchgangsventile reduzieren Strömungen und der Eintritt liegt in der selben Richtung wie der Austritt. Eckventile leiten den Strom um, indem Ein- und Austritt im rechten Winkel zueinander liegen, während Schrägventile die Strömungsrichtung um 45° ändern. Die für diese Arbeit wichtigen Ventile sind Drei-Wege-Ventile, die für das kontrollierte Mischen von Fluidströmen verwendet werden. Bei allen Kategorien kann man noch Unterschiede in den Betätigungsarten machen, um jedoch ein Signal für den Messbereich zu haben, werden hier elektronische Betätigungen bevorzugt.

\subsection{Luftfiltersysteme}
Als Luftfilter werden alle Abscheider bezeichnet, die Aerosole oder andere unerwünschte Schwebstoffe aus der Luft herausfiltern. Es gibt viele verschiedene Bauarten von Luftfiltersystemen, doch es werden nur Trockenfilter, Zyklonabscheider und elektrostatische Luftfilter verwendet, weshalb die flüssigkeitsbasierten Filter hier nicht wieder erwähnt werden. \\
Trockenfilter bestehen aus ringförmig oder rechteckig angeordneten Gebilde, die zickzackförmiges Gewebe als Filterelement haben. Die Faltung vergrößert die Filterfläche und verringert den Strömungswiderstand. Die kontaminierte und ungefilterte Luft wird ausserhalb des Zylinders angesaugt und die gefilterte saubere Luft wird innerhalb der Lamellen weitergeleitet. Die Partikel und Schwebeteilchen bleiben in den Gewebefalten hängen.\\
Elektrofilter finden hauptsächlich Anwendung bei der Abscheidung von Aerosolen, sind also für die Arbeit sehr gut geeignet. Im Filter werden die Staub- oder Aerosolpartikel elektrostatisch aufgeladen und an den Elektrodenflächen abgeschieden. In getakteten Abfolgen werden die aufgefangenen Verschmutzungen maschinell entfernt und in einem Trichter gesammelt. Der Vorteil dieser Filter im Vergleich zu den Trockenfiltern ist, dass es keine Filtereinsätze gibt, die ausgetauscht werden müssen, allerdings sind die elektrostatischen Filter wesentlich teurer als die Wegwerffilter mit Gewebeeinsatz.\\
Zyklonabscheider werden dazu genutzt um Partikel einer bestimmten Größe aus dem Strom herauszufiltern. Diese finden ihre Anwendung um nur Partikelgrößen durchzulassen, die für die verwendeten Messgeräte erkennbar sind. Sie sind allerdings zu ineffizient um die Luft von allen Partikeln zu reinigen, weshalb sie nur zur Aufbereitung des Partikelstroms in Frage kommen.

\subsection{Ventilatoren}
Ventilatoren werden in den Konzepten verwendet um einen konstanten Strom zu erzeugen. Mit ihnen können Fluide und Gase gefördert werden. Dafür wird die Antriebsart des Ventilators in die Bewegungsenergie des Mediums umgewandelt ohne dass es in der Strömung zu einer Drucksteigerung kommt. Ein Ventilator ist eine fremd angetriebene Strömungsmaschine, die mittels eines in einem Gehäuse rotierenden Laufrads ein gasförmiges Medium fördert. 
\section{Signaltechnik}
Da es in dieser Arbeit darum geht einen Signalsprung zu erzeugen, wird noch einmal die grundlegende Theorie zu Sprungfunktionen und Signalzeiten erl\"{a}utert. Nach der Definition ist ein Signal ein Zeitpunkt zu dem ein wichtiges Ereignis eintritt. In diesem Fall ist das Ereignis ein Partikelsprung. Als Messstelle wird der optische Z\"{a}hler im Messger\"{a}t verwendet, an dem die eigentliche Messung der Partikel stattfindet. Zu Beginn des Systems enth\"{a}lt das Gas, welches den Z\"{a}hler passiert in der Theorie keine Partikel. Der Partikelanteil liegt also bei \(0\) Partikel \(/\) \(1m^3\) Luft. Ist der Partikelstrom komplett aufgebaut und wird in das Messger\"{a}t geleitet, beschreibt der Partikelanteil im Strom einen Sprung auf einen angestrebten Wert.
\\
Bei diesem Vorgang sind allerdings einige Zeiten zu beachten, welche das System und die entstehende Sprungfunktion beeinflussen. Zun\"{a}chst einmal hat das Messger\"{a}t eine Totzeit, welche die Zeit beschreibt, die ein Partikel vom Eingang des Messger\"{a}ts bis zum Zeitpunkt seiner Messung ben\"{o}tigt. Also bis zu dem Zeitpunkt in dem das Auftreten des Partikels in der Sprungfunktion sichtbar wird. Eine weitere Zeitspanne, die ber\"{u}cksichtigt werden muss, ist die Schaltzeit eines Ventils. Nachdem der Strom voll aufgebaut ist, muss der Luftstrom, der in das Messger\"{a}t einstr\"{o}mt umgeschaltet werden auf den aufgebauten Partikelstrom. Dieses Umschalten kostet auch Zeit, je nachdem welcher Schaltmechanismus verwendet wird. Zusammen genommen ergeben die beiden Zeiten die Totzeit des gesamten Systems. Beschrieben wird die Zeitspanne vom Umschalten des Schaltmechanismus bis zum ersten Anstieg der gemessenen Sprungfunktion.
\begin{figure}[H]
        \myfloatalign
        {\includegraphics[width=.7\linewidth]{gfx/fundamentals/jump.jpg}} \quad
        \caption[Perfekter Signalsprung]
        {Perfekter Signalsprung}
        \label{fig:sprung}
\end{figure}
\section{Stand der Technik}
W\"{a}hrend der Einarbeitungszeit sind wir durch Literaturrecherche auf einige Dissertationen, Abschlussarbeiten und Richtlinien aufmerksam geworden, welche vergleichbare Versuchsaufbauten beinhalten, weche haupts\"{a}chlich zur Kalibrierung von Partikelmessger\"{a}ten dienen.

Die Masterarbeit \cite{auswertemethodik} hatte das Ziel das Ansprechverhalten einer Constant Volume Sampling (CVS) Anlage zu optimieren. Der daf\"{u}r verwendete Versuchsaufbau ist in Abbildung \ref{fig:aufbau_graz} abgebildet. Das hierf\"{u}r verwendete Aerosol wurde aus den Abgasen eines Versuchsfahzeugs entnommen und zusammen mit Verd\"{u}nnungsluft in einen CVS Tunnel geleitet, in welchem sich ein Partikelmesssystem befindet. Die Partikeltestsignale wurden hierbei mit Hilfe eines Umschalters erzeugt, bei dem zwischen einem Aerosolstrom und Luftstrom geschaltet werden kann (Abbildung \ref{fig:umschalter_graz}). Diese M\"{o}glichkeit der Generierung eines Partikeltestsignals wurde in die Erarbeitung von unseren Konzepten miteinbezogen. Zudem k\"{o}nnte der CVS Tunnel f\"{u}r nachfolgende Arbeiten, welche sich ausschlie{\ss}lich mit den Str\"{o}mungen der Aerosole besch\"{a}ftigen w\"{u}rde, von Interesse sein.

\begin{figure}[H]
	\myfloatalign
	{\includegraphics[width=.7\linewidth]{gfx/related/graz_versuch.pdf}} \quad
	\caption[Versuchsaufbau der TU Graz (Quelle: \cite{auswertemethodik}, S.2)]
	{Versuchsaufbau der TU Graz (Quelle: \cite{auswertemethodik}, S.2)}
	\label{fig:aufbau_graz}
\end{figure}

In \cite{candle} wird die chemische Zusammensetzung von Kerzenrauch untersucht. Diese Arbeit beinhaltet zwei Punkte, welche f\"{u}r unser Projekt von Interess waren. Zum einen zeigt die Arbeit, dass sich Kerzenrauch unter bestimmten Bedinungen als Pr\"{u}faeorosol eignet. Zum anderen stellt der verwendete Versuchsaufbau (Abbildung \ref{fig:kerze_exp}) eine M\"{o}glichkeit dar, wie eine solche Aerosolprobe zu einem Partikelmessger\"{a}t gef\"{u}hrt werden kann. Im Verlauf des Projektes hat sich allerdings gezeigt, dass eine Kerze als Aerosolquelle f\"{u}r den von uns zu erarbeitenden Pr\"{u}fstand nicht geeignet ist.

\begin{figure}[H]
	\myfloatalign
	{\includegraphics[width=.9\linewidth]{gfx/related/kerze_versuch.pdf}} \quad
	\caption[Versuchsaufbau zur Ermittlung der chemischen Zusammensetzung von Kerzenrauch (Quelle: \cite{candle}, S.195)]
	{Versuchsaufbau zur Ermittlung der chemischen Zusammensetzung von Kerzenrauch (Quelle: \cite{candle}, S.195)}
	\label{fig:kerze_exp}
\end{figure} 

Auf direkte Nachfrage hin beim Hersteller der Partikelmessger\"{a}te, wie die angegebenen Werte in den technischen Dokumentationen verifiziert wurden, wiesen diese lediglich darauf hin, dass die Partikelmessger\"{a}te mit Hilfe von Aerosolgeneratoren kalibriert werden. Dies entspricht der Standardvorgehensweise zur Kalibrierung solcher Messger\"{a}te, wie sie in den VDI Richtlinien zu finden sind. Aus diesen liesen sich Kenngr\"{o}{\ss}en und Richtlinien an die zu verwendeten Aerosolgeneratoren ableiten, welche in der Reinraumtechnik Verwendung finden.
\\\\
Uns sind letzendlich keine Versuchsaufbauten bekannt, welche ausschlie{\ss}lich zur Identifizierung der Zeitkonstante eines Partikelmessger\"{a}ts dienen. Zudem lies sich die Zeitkonstante aus keinem der uns vorliegenden Handb\"{u}cher entnehmen die darauf hindeuten w\"{u}rden, dass es ein existierendes Verfahren gibt diese zu messen. 

%************************************************
\chapter{Vergleich von Pr\"{u}f-Aerosole}\label{ch:aerosol}
%************************************************
Ein Pr\"{u}faerosol ist ein Aerosol f\"{u}r das je nach Verwendungszweck relevante Eigenschaften bekannt sind. Es kann auf verschiedene Arten und Weisen hergestellt werden und dient unter anderem zur Kalibrierung von Partikelmessger\"{a}ten und f\"{u}r Filtertests.\\
Die Herstellung eines Pr\"{u}faerosols erfolgt in mehreren Schritten. Es beginnt mit der konstanten Zuf\"{u}hrung des Partikelmaterials, gefolgt von der eigentlichen Partikelproduktion. War das Partikelmaterial hochkonzentriert, findet am Ende meist noch eine Verd\"{u}nnung statt. Herstellungsmethoden sind Zerst\"{a}ubung von Feststoffen, Vernebeln von Fl\"{u}ssigkeiten, Kondensationsverfahren oder Verbrennungsprozesse.\\
Da f\"{u}r einige Konzepte Partikelgeneratoren zum Einsatz kommen, ist es wichtig die g\"{a}ngstigen Pr\"{u}faerosole kurz zu analysieren um ein passendes Aerosol auszuw\"{a}hlen.

\section{Di-Ethyl-Hexyl-Sebacat (DEHS)}
%TODO
\section{Di-N-Octylphtalat (DOP)}
Di-N-Octylphtalat (DOP) ist eine organische Verbindung aus der Gruppe Phtalate. Es eine farblose, geruchlose und \"{o}lige Fl\"{u}ssigkeit. DOP wird durch die Reaktion von Phtalats\"{a}ureanhydrid und Octanol in Gegenwart eines Katalysators gewonnen. Wie DEHS ist DOP in Wasser kaum l\"{o}slich. Auch in der Anwendung deckt DOP das selbe Gebiet ab, dar\"{u}ber hinaus findet es allerdings noch Anwendung im medizinischen Bereich und in der Sprengstoffindustrie. DOP z\"{a}hlt als stark krebserregend, weshalb es nicht mehr als Weichmacher f\"{u}r Verbrauchsgegenst\"{a}nde verwendet werden darf. Auch hier wird eine gro{\ss}e Spannbreite an Partikelgr\"{o}{\ss}en abgedeckt, je nachdem welcher Partikelgenerator verwendet wird.
\begin{itemize}
\item Aggregatzustand: fl\"{u}ssig
\item Dichte: \(0,98 g/cm^3\)
\item Schmelzpunkt: \(-49^\circ\text{C}\)
\item Siedepunkt: \(385^\circ\text{C}\)
\item Partikelgr\"{o}{\ss}e: Abh\"{a}ngig vom Generator
\end{itemize}
\section{Emery 3004 (PAO-4)}
%TODO
\section{Poly Styrene Latex Spheres (PSL)}
%TODO
\section{Auswertung der Analyse}
%TODO

\subsection{Anforderungsvergleich der Aerosole}
%TODO
\section{Unkonventionelle Aerosolquellen}
Neben industriellen Partikelgeneratoren, wurden f\"{u}r die Arbeit noch andere, unkonventionelle Aerosolquellen in Betracht gezogen und f\"{u}r die Konzeptentwicklung verwendet. Grund daf\"{u}r ist einmal die Einfachheit solcher Quellen und die geringen Anschaffungskosten. Weiterhin ist es f\"{u}r die reine Erzeugung eines Partikelsprungs gar nicht n\"{o}tig eine hoch aufgel\"{o}ste und genaue Partikeldichte zu erreichen, sondern lediglich einen klaren Partikelanstieg zu generieren. In diesem Bereich k\"{o}nnten solche unkonventionellen Mittel bereits ausreichen, weshalb sie in die Konzeptentwicklung mit eingeschlossen wurden.\\
In Betracht gezogen wurden Kerzen, kleine Liquidverdampfer aus dem Modellbau oder einer elektronischen Zigarette und partikelbelastete Gro{\ss}r\"{a}ume. Eine Kerze erzeugt bei Windstille einen konstanten Ru{\ss}partikelstrom \"{u}ber einen langen Zeitraum und eignet sich so f\"{u}r einen Versuch mit mehreren Wiederholungen. Ein Nachteil ist, dass die Kerze nicht gut steuerbar ist, anders als bei einem elektronischen Modellverdampfer. Bei einem elektronischen Verdampfer muss der Strom allerdings jedes mal neu aufgebaut werden, w\"{a}hrend die Kerze direkt einen Konstanten Strom liefert und haben somit eine h\"{u}here Totzeit. Ein anderes Konzept war den Versuchsaufbau in einem konstant mit Partikeln belasteten Gro{\ss}raum einzusetzen, wie zum Beispiel in Werkst\"{a}tten oder in Sporthallen. Bei solch einem Einsatz muss darauf geachtet werden, dass der Frischluftstrom in der Nullphase des Systems mit guten Filtern erzeugt wird, da sonst der Partikelsprung nicht eindeutig genug zu erkennen ist.\\
Alles in allem ist es m\"{o}glich auch mit unkonventionellen Mitteln einen Partikelsprung zu generieren, wenn man den Versuchsaufbau entsprechend anpasst. In den Konzeptideen sind die m\"{o}glichen Anwendungen dargestellt.

%************************************************
\chapter{Konzepte f\"{u}r den Versuchsaufbau}\label{ch:concepts}
%************************************************
Im folgenden Kapitel werden die entwickelten Konzepte f\"{u}r den Versuchsaufbau vorgestellt. Dabei werden zuerst die f\"{u}r einige Konzepte verwendeten Partikelgeneratoren n\"{a}her erl\"{a}utert. Danach wird der morphologische Kasten erkl\"{a}rt, in dem verschiedene Bauteile f\"{u}r die Konzepte ermittelt wurden, um abschlie{\ss}end Aufbau und Funktionsweise der verschiedenen Konzepte darzulegen.

\section{Partikelgeneratoren}
F\"{u}r die Kalibrierung und Pr\"{u}fung von Partikelmessger\"{a}ten, werden oft industrielle Partikelgeneratoren verwendet. Diese Generatoren arbeiten auf Basis von Vernebelung von Fl\"{u}ssigkeiten. Die Fl\"{u}ssigkeiten werden unter hohem Druck in einer D\"{u}se vernebelt und verspr\"{u}ht. Dabei entsteht ein Aerosol welches Partikel der verwendeten Fl\"{u}ssigkeit enth\"{a}lt. Um die Messger\"{a}te gezielt kalibrieren zu k\"{o}nnen, kann man bei den Generatoren die gew\"{u}nschte Dichte sowie die Gr\"{o}{\ss}e der Partikel innerhalb des Aerosols regulieren. So hat man zu jedem Zeitpunkt Kenntnis \"{u}ber den Aufbau des Partikelstroms. Da sich in dieser Arbeit auf die Verwendung von DEHS Aerosol beschr\"{a}nkt wird, werden nur die beiden Generatoren in Erw\"{a}gung gezogen, welche DEHS unterst\"{u}tzen. Dies sind auch die beiden Ger\"{a}te, welche in der Industrie am meisten vorkommen und verwendetet werden.

\subsection{Topas ATM 220}
Die Ger\"{a}tereihe ATM von Topas wird schon seit langer Zeit speziell im Bereich der Reinraum- und Filtertestanwendung verwendet. Das Kernst\"{u}ck des Generators ist ein Edelstahlatomizer, eine nach dem Injektorprinzip arbeitende Zweistoffd\"{u}se. Dieser sorgt f\"{u}r die R\"{u}ckf\"{u}hrung der bei der Verd\"{u}sung entstehenden gro{\ss}en Tropfen, welches die Partikelkonzentration nur unwesentlich beeinflusst. Die f\"{u}r die Verd\"{u}sung ben\"{o}tigte Druckluft wird durch einen HEPA-Filter gereinigt. Die D\"{u}se ist direkt in die Fl\"{u}ssigkeit getaucht, um auch sehr geringe Massestr\"{o}me reproduzierbar einstellen zu k\"{o}nnen.\\
Der Topas ATM hat eine Partikelmengenspannbreite  von \(10^5\) bis \(10^8\) Partikel pro \(cm^3\), bei einer Gr\"{o}{\ss}enverteilung von \(0,1 - 0,5 \mu m\).\cite{topas}

\subsection{Palas 2000H}
Der Palas 2000H wird nicht nur f\"{u}r die Kalibrierung von Partikelmessger\"{a}ten sondern auch zum Testen von Filtersystemen verwendet. Im Gegensatz zum ATM arbeitet der Palas mit Hilfe von Kondensation und Verdampfung. So kann durch Druck- und Temperatur\"{a}nderungen die Partikelgr\"{o}{\ss}enverteilung reguliert werden. Der gro{\ss}e Vorteil des Palas 2000H ist, dass er \"{u}ber Ethernet von einem Computer aus gesteuert werden kann und somit direkt in ein Evaluationsprogramm f\"{u}r den Versuchsaufbau eingebunden werden kann. Er ist wesentlich nutzerfreundlicher als der f\"{u}r den Fachmann entwickelte ATM, erzeugt aber wesentlich gr\"{u}{\ss}ere Partikel, die man f\"{u}r den Versuchsaufbau eventuell gar nicht ben\"{o}tigt. Au{\ss}erdem arbeitet der Generator mit einer Temperaturregelung und erhitzt das erzeugte Aerosol auf diesem Weg, welches sich nachteilig auf das Verhalten der Str\"{o}mung auswirken kann. Die Partikelmenge kann nicht eingestellt werden und liegt bei \(10^6\) Partikeln\(/cm^3\) bei einer regulierbaren Gr\"{o}{\ss}enverteilung von \(0,2 \mu m - 100\mu m\).\cite{palas}
\section{Morphologischer Kasten f\"{u}r die Konzepte}
Im Folgenden werden die Teill\"{o}sungen des morphologischen Kastens erl\"{a}utert und mit Hilfe der Anforderungsliste bewertet. Grunds\"{a}tzlich unterscheiden sich die Teilfunktionen der Aerosolbereitstellung, dem Aerosol und der Sprungerzeugung.
\\\\
Die Aerosolbereitstellung erfolgt entweder durch einen Aerosolgenerator, welcher das Aerosol generiert oder einem Druckbeh\"{a}lter, gef\"{u}llt mit Aerosol. M\"{o}glich ist es auch einen Rauchentwickler oder einen bereits partikelbelasteten Raum, beispielsweise eine Sporthalle als Aerosolquelle zu nutzen.
\\\\
Ein Aerosolgenerator erzeugt ein langzeitstabiles Pr\"{u}faerosol mit einer Partikelgr\"{o}{\ss}enverteilung\cite{candle}. Der Massenstrom und die Partikelanzahlkonzentration sind regulierbar. Diese Konditionierung des Aerosols ist bei einer vorhandenen Aerosolquelle wie einem gef\"{u}llten Druckbeh\"{a}lter oder einem Rauchentwickler, wie zum Beispiel einer Kerze nicht m\"{o}glich. Da der FMPS und der APS einen h\"{o}heren Volumenstrom ben\"{o}tigen als der Aerosolgenerator erzeugen kann, sind konstruktive Ma{\ss}nahmen erforderlich, die das Aerosol verd\"{u}nnen und den Volumenstrom erh\"{o}hen. Hierbei wird eine homogene Durchmischung der gereinigten Umgebungsluft und des Aerosols mit Hilfe von lokalen Turbulenzen erreicht. Ein weiteres m\"{o}gliches Problem ist der Druckgradient, der durch den Druckbeh\"{a}lter oder den Aerosolgenerator entstehen kann.
\\\\
Da die Geschwindigkeit der Luftstr\"{o}mung bei gleichen Umgebungsbedingungen sehr viel geringer als die Schallgeschwindigkeit ist, kann die Str\"{o}mung als inkompressibel angenommen werden. Der Totaldruck einer reibungsfreien, inkompressiblen Rohrstr\"{o}mung, welcher an einer beliebigen Stelle auf den Rohrquerschnitt wirkt, errechnet sich nach Bernoulli durch
\begin{align*}
	P_\text{Total} = P_\text{Statisch} + P_\text{Gewicht} + P_\text{Dynamisch}
\end{align*}
Dabei ist \(P_\text{Gewicht}\) der Anteil des Totaldrucks, welcher durch das Eigengewicht des Fluides erzeugt wird. In dieser Betrachtung wird von einem horizontal verlaufenden Rohr ausgegangen, somit ist der Schweredruck \(P_\text{Gewicht}\) \"{u}ber die gesamte Rohrl\"{a}nge konstant.
\\\\
Der statische Druck \(P_\text{Statisch}\) ist der Druck, der senkrecht zur Str\"{o}mung gemessen werden kann. Am Aerosolauslass des Aerosolgenerators entspricht der statische Druck dem Betriebsdruck des Generators, also dem Druck welcher in seinem Inneren herrscht.
\\\\
Der Staudruck \(P_\text{Dynamisch}\) errechnet sich mit der Str\"{o}mungsgeschwindigkeit \(V\) und der Dichte \(\rho\) zu
\begin{align*}
	P_\text{Dynamisch} = \frac{\rho}{2} * V^2
\end{align*}
Die Betrachtung der Dr\"{u}cke \(P_\text{Total, Messger\"{a}teeinlass}\), welche am Einlass der Messger\"{a}te herrschen d\"{u}rfen, f\"{u}hren nun zu folgender Gleichung zur Bestimmung der erlaubten Betriebsdr\"{u}cke am Aerosolgeneratorauslass:
\begin{align*}
	P_\text{Betrieb} = P_\text{Total, Messger\"{a}teeinlass} - \frac{\rho}{2} * V_			\text{Aerosolauslass}^2
\end{align*}
Daraus ergeben sich folgende Betriebsdr\"{u}cke auf Grundlage der verschiedenen Messger\"{a}teanforderungen, dem Durchmesser des Aerosolgeneratorauslasses und der Dichte der Luft bei \(20^\circ C\) Umgebungstemperatur\cite{fmps_3091}\cite{ops_3330}\cite{ucpc_3776}\cite{aps_3321}\cite{topas}.
\begin{enumerate}
	\item \textbf{FMPS}\\
	\(P_\text{Total, Messger\"{a}teeinlass} = 70 - 103 kPa\)\\
	\(V_\text{Aerosolauslass} = 3,34 \frac{m}{s}\)\\
	\(P_\text{Betrieb} = 70 - 103 kPa\)\\
	\item \textbf{UCPC}\\
	\(P_\text{Total, Messger\"{a}teeinlass} = 70 - 103 kPa\)\\
	\(V_\text{Aerosolauslass} = 3,34 \frac{m}{s}\)\\
	\(P_\text{Betrieb} = 70 - 103 kPa\)\\
	\item \textbf{APS}\\
	\(P_\text{Total, Messger\"{a}teeinlass} = 70 - 103 kPa\)\\
	\(V_\text{Aerosolauslass} = 3,34 \frac{m}{s}\)\\
	\(P_\text{Betrieb} = 70 - 103 kPa\)
\end{enumerate} 
Aus den Ergebnissen l\"{a}sst sich schlie{\ss}en, dass der Aerosolgenerator im Bereich des Umgebungsdruckes arbeiten muss, damit er mit den Messger\"{a}ten kompatibel ist. Im n\"{a}chsten Schritt muss im Rahmen eines Ist-Soll-Vergleiches der tats\"{a}chliche Betriebsdruck des Aerosolgenerators ermittelt werden, dies w\"{u}rde aber den Rahmen dieses ADP's sprengen.
\begin{figure}[H]
        \myfloatalign
        {\includegraphics[width=.9\linewidth]{gfx/concepts/Morph.jpg}} \quad
        \caption[Morphologischer Kasten]
        {Morphologischer Kasten}
        \label{fig:Morph}
\end{figure}
Im Falle des OPS 3330 ist nur eine Druckdifferenz zwischen Aerosoleinlass und Messger\"{a}teauslass gegeben, welche den Grenzwert von \(P < 0,756 kPa\) nicht \"{u}berschreiten darf. Somit kann durch den konstruktiven Ma{\ss}nahmen, wie eine dem Auslass in Reihe geschaltete Pumpe, der Aerosolgenerator unabh\"{a}ngig von seinem Betriebsdruck genutzt werden.
\\\\
Bei den Berechnungen des Betriebsdruckes ist zu beachten, dass Druck\"{a}nderungen aufgrund von Leitungsreibung, eventuell horizontal verlaufenden Leitungsabschnitten, Querschnitts\"{u}berg\"{a}ngen, Umlenkungen und zugeschalteten Ger\"{a}ten vorerst vernachl\"{a}ssigt wurden. Au{\ss}erdem wurde angenommen, dass der Aerosolauslass den erforderlichen Volumenstrom bereitstellt, zum Beispiel durch eine Mischstufe mit Umgebungsluft. Genauere Ergebnisse k\"{o}nnen nur durch Betrachtung des fertigen Pr\"{u}faufbaus erfolgen.
\\\\
Zur Nutzung des Aerosolgenerators werden als m\"{o}gliche Aerosole eine DEHS-Luft Dispersion und PAO-4-Luft Dispersion genutzt, da diese den Anforderungen entsprechen und zudem gesundheitlich unbedenklich sind. Als m\"{o}gliche Aerosole f\"{u}r einen Rauchentwickler wurden der Rauch einer Kerzen, Tabakrauch und der Nassdampf einer E-Zigarette in Betracht gezogen. Der Rauch solcher Entwickler besteht aus unterschiedlichen organischen und anorganischen Substanzen, die unterschiedliche Eigenschaften aufweisen. Es besteht somit keine konstante Partikelgr\"{o}{\ss}enverteilung und keine einstellbare Partikelanzahl. Messungen mit Rauchentwicklern sind daher nicht reproduzierbar, da die Gleichheit der Partikeleigenschaften bei erneuten Messungen nicht garantiert werden kann(Q). Die Rauchentwickler erf\"{u}llen die Anforderungen zun\"{a}chst teilweise. Die Partikelgr\"{o}{\ss}en reichen von \(10 nm\) bis zu \(1000 nm\). Jedoch kann nicht sichergestellt werden, dass die Mindespartikelanzahlkonzentration der Partikel, die gr\"{o}{\ss}er als \(500 nm\) sind, f\"{u}r den APS erreicht wird. Hohe Partikelanzahlkonzentrationen ergeben sich f\"{u}r die Partikel der Gr\"{o}{\ss}en \(100 nm\) bis \(450 nm\). Diese Gr\"{o}{\ss}en k\"{o}nnen die anderen Messger\"{a}te erkennen, jedoch ist f\"{u}r diese Ger\"{a}te die Partikelanzahlkonzentration zu hoch.
\begin{align*}
C_\text{Kerze, brennend} < 0,51 - 1,14 * 10^6 \frac{\text{Partikel}}{cm^3}
\end{align*}
\begin{align*}
C_\text{Kerze, ru{\ss}end} < 0,27 - 0,89 * 10^6 \frac{\text{Partikel}}{cm^3}
\end{align*}
\begin{align*}
C_\text{Tabak} > 10^8 \frac{\text{Partikel}}{cm^3}
\end{align*}
\begin{align*}
C_\text{E-Zigarette} > 10^9 \frac{\text{Partikel}}{cm^3}
\end{align*}
Abhilfe kann jedoch durch einen Verd\"{u}nner geschaffen werden. Die Temperatur, die durch den Rauch entsteht, kann mit Hilfe eines Thermokonditionierer geregelt werden.
\\\\
Es besteht zus\"{a}tzlich die M\"{o}glichkeit einen feinstaubbelasteten Raum als Aerosolquelle zu nutzen. Da der ben\"{o}tigte Volumenstrom direkt aus der Umgebung entnommen wird, ist die erforderliche Menge gew\"{a}hrleistet. Auch Luft als Tr\"{a}gergas und die Tr\"{a}gergastemperatur k\"{o}nnen eingehalten werden. Die \"{U}berlegung die Umgebungsluft in Limburg als Aerosol zu nutzen, vereinfacht den Versuchsaufbau. Jedoch liegen Messwerte des hessischen Landesamtes f\"{u}r Naturschutz, Umwelt und Geologie des Jahres 2017 nur f\"{u}r \(PM_{10}\)-Werte vor. Diese liegt bei \(17,3 \frac{\mu g}{m^2}\). Dies entspricht einer Partikelanzahlkonzentration von \(0,033 \frac{\text{Partikel}}{cm^3}\)\cite{hlnug}. Damit ist jedoch nicht sichergestellt, dass die gegebenen Messger\"{a}te alle Partikel erkennen, da die Partikel bis zu \(10 \mu m\) gro{\ss} sein k\"{o}nnen. Durch die \(PM_{10}\)-Wertschwankungen, die tages- und monatsbedingt auftreten, kann die Partikelanzahlkonzentration auch geringer ausfallen. Daher ist die Umgebungsluft als Aerosol unzuverl\"{a}ssig und wird daher ausgeschlossen.\\
In einer Kletterhalle liegt ein hoher Verbrauch von Magnesiumcarbonat vor. In verschiedenen Kletterhallen in Deutschland wurden \(PM_{10}\), \(PM_{2,5}\) und \(PM_1\) Werte gemessen\cite{exp}. Um sicherzustellen, dass die Minimalforderungen eingehalten werden k\"{o}nnen, wurden die Partikelanzahlkonzentration zu den Sto{\ss}zeiten, welche haupts\"{a}chlich abends liegen ausgewertet:
\begin{align*}
C_{PM 1} = 76,39 \frac{\text{Partikel}}{cm^3}
\end{align*}
\begin{align*}
C_{PM 2,5} = 48,89 \frac{\text{Partikel}}{cm^3}
\end{align*}
\begin{align*}
C_{PM 10} = 4,77 \frac{\text{Partikel}}{cm^3}
\end{align*}
Die Mindestpartikelanzahlkonzentration f\"{u}r den FMPS wird nicht erf\"{u}llt. Es wird davon ausgegangen, dass durch den Trendsport die Partikelanzahlkonzentration im Jahre 2018 besonders zu den Sto{\ss}zeiten gestiegen ist, und eine neue Studie n\"{o}tig ist, um die aktuelle Partikeldichte zu messen. Die Studie zeigt, dass die geforderten Partikelgr\"{o}{\ss}en gegeben sind und ein Versuchspr\"{u}fstand mit handels\"{u}blichen Magnesiumcarbonat als Aerosol m\"{o}glich ist. Aufgrund der Nichteignung f\"{u}r den FMPS entf\"{a}llt die Magnesium-Luft Dispersion in einer Kletterhalle als Aerosol.
\\\\
Die Sprungerzeugung ist ein zentraler Mechanismus der Versuchseinrichtung. Daf\"{u}r werden Mehrwegschieber in Betracht gezogen, welche in sp\"{a}teren Kapitel erl\"{a}utert werden. Unkonventionelle Sprungerzeuger in Form eines Luftballons oder mit Hilfe des Mundes entfallen, da sowohl die ben\"{o}tigte menge an Volumenstrom und die erforderliche Mindestmesszeit mit den Partikelsprungerzeugern nicht eingehalten werden kann. Zudem sind die Messungen nicht reproduzierbar.
\section{Konzept 1}
Das erste Konzept basiert auf der Verwendung von industriellen Partikelgeneratoren und mechanischen Ventilen. Hierbei gibt es zwei M\"{o}glichkeiten den Strom zu erzeugen, die im Folgenden n\"{a}her erl\"{a}utert werden sollen. Der Fokus liegt bei diesen Konzepten auf dem Rohrsystem, da die eigentlichen Bauteile bereits funktionsf\"{a}hig gekauft werden und nur verbaut werden m\"{u}ssen.

\subsection{Aufbau}
Der Versuchsaufbau besteht aus einer Parallelschaltung eines Aerosolgenerators mit einem Ventilator. Der Ventilator erzeugt einen reinen Luftstrom der mit dem Aerosolstrom in das selbe Rohr geleitet wird. In diesem Rohr ist eine Mischvorrichtung installiert, welche aus Luft- und Aerosolstrom eine homogene Mischung erzeugt. Das Rohr endet am Anschluss(\textit{Anschluss 1}) eines 4/2 - Wegeventils (\ref{fig:ventil}). Ein Ausgang des Ventils (\textit{Anschluss 2}) f\"{u}hrt raus in die Umgebungsluft, \textit{Anschluss 4} f\"{u}hrt \"{u}ber einen Adapter direkt an den Einlass des Messger\"{a}tes. Dabei ist es wichtig, darauf zu achten, dass die Verbindung zwischen Ventil und Messger\"{a}t so kurz wie m\"{o}glich gehalten wird um die Totzeit des Systems klein zu halten. Dennoch muss es mindestens so lang sein um einen moderaten Querschnitts\"{u}bergang zu gew\"{a}hrleisten, um Druck- und Partikelverluste zu minimieren. \textit{Anschluss 3} des Ventils ist mit einem weiteren Ventilator verbunden, welcher auch einen reinen Luftstrom erzeugt. Dieser Luftstrom ist f\"{u}r die Nullphase des Messger\"{a}tes gedacht.
\begin{figure}[H]
        \myfloatalign
        {\includegraphics[width=.9\linewidth]{gfx/concepts/ventil_feder.jpg}} \quad
        \caption[Schaltm\"{o}glichkeiten von 4/2-Wegeventilen]
        {Schaltm\"{o}glichkeiten von 4/2-Wegeventilen}
        \label{fig:ventil}
\end{figure}

\newpage

\subsection{Funktionsweise}
Der Aerosolgenerator generiert einen konstanten Partikelmassenstrom mit einstellbaren Partikeleigenschaften wie die Konzentration oder die Gr\"{o}{\ss}enverteilung der Partikel. Zusammen mit der Luft aus dem externen Ventilator kann die Partikelkonzentration zus\"{a}tzlich nochmal reguliert werden und der Aerosolstrom dem angeschlossenen Messger\"{a}t angepasst werden. Die anschlie{\ss}ende Mischvorrichtung sorgt f\"{u}r eine gleichm\"{a}{\ss}ige \"{u}ber den Rohrquerschnitt verteilte Partikelkonzentration sowie eine laminare Str\"{o}mung beim Eintritt in das Ventil. Der Vorteil davon liegt darin, dass manche Messger\"{a}te dem Volumenstrom noch vor dem Passieren der Messkammer einen Schleierstrom (\textit{Sheath-Air}) entnehmen, da dieser abgezweigte Strom bei hohem Konzentrationsgradienten in radialer Richtung zu starken Schwankungen in der Konstanz des Partikelstroms f\"{u}hren kann. Das 4/2 - Wegeventil zeichnet sich durch vier Anschl\"{u}sse und zwei Schaltstellungen aus. Dabei ist in der stabilen Position (\textit{Stellung 0}) \textit{Anschluss 1} direkt mit \textit{Anschluss 2} verbunden. Dementsprechend f\"{u}hrt \textit{Anschluss 3} zu \textit{Anschluss 4}, welcher in das Messger\"{a}t m\"{u}ndet. In der monostabilen Stellung (\textit{Stellung 1}) f\"{u}hrt der Aerosolstrom (\textit{Anschluss 1}) direkt zum Messger\"{a}t (\textit{Anschluss 4}) und der Anschluss des Ventilators (\textit{Anschluss 3}) f\"{u}hrt direkt \"{u}ber \textit{Anschluss 2} in die Umgebungsluft. Das Ventil kann dabei elektronisch oder pneumatisch geschaltet werden. Der extern angetriebene Ventilator treibt Umgebunsluft durch einen Nullfilter zu dem jeweils geschalteten Anschluss. Dabei ist wichtig, dass der Volumenstrom der reinen Luft dem ben\"{o}tigten Volumenstrom des Messger\"{a}tes entspricht. Der Nullfilter sorgt f\"{u}r einen hohen Partikelkonzentrationsunterschied des Messger\"{a}teeinlassstroms zwischen Stellung 0 und Stellung 1 des Ventils.

\subsection{Durchf\"{u}hrung}
Als erstes stellt man den externen Ventilator an \textit{Anschluss 3} auf den Volumenstrom des Messger\"{a}tes ein und schlie{\ss}t das Messger\"{a}t dann \"{u}ber den Adapter an \textit{Anschluss 4} an.\\
Die Vorlaufzeit, also die Zeit bevor das Ventil geschaltet wird, h\"{a}ngt von mehreren Faktoren ab. So hat das Messger\"{a}t eine Einlaufzeit, nach der es erste zuverl\"{a}ssige Ergebnisse liefert. Nach Ablauf dieser Nullzeit wird das Messger\"{a}t weiter mit dem durch den Ventilator angetriebenen Luftvolumenstrom durchstr\"{o}mt. Parallel dazu laufen der Aerosolgenerator und der andere Ventilator, bis sie die gew\"{u}nschten Stromeigenschaften liefern. Diese Laufzeit ist vor allem von der Einstellzeit des Aerosolgenerators abh\"{a}ngig, welche die Anpassung eines Ausgangsparameters auf eine \"{A}nderung eines Eingangsparameters beschreibt. Nachdem die Voraussetzungen eines eingelaufenen Messger\"{a}tes und eines konstanten Aerosolstroms erf\"{u}llt sind, kann das Ventil geschaltet werden und der korrigierte Verlauf der Sprungantwort mit dem Eingangssignal verglichen werden. Die Korrektur betrifft hierbei die Berechnung der systematischen Totzeit des Versuchsaufbaus, welche sich aus der Schaltzeit des Ventils und der Zeit, welche die Partikelfront braucht um den Messger\"{a}teeinlass zu erreichen, zusammensetzt.

\subsection{Variation: Konzept 2}
Der Aufbau des ersten Konzeptes l\"{a}sst sich stark vereinfachen, wenn Messger\"{a}te benutzt werden, welche mit geringen Volumenstr\"{o}men arbeiten. Der geringe Volumenstrom ist alleine durch den Aerosolgenerator ohne einen externen Ventilator erreichbar. Somit ist auch die Mischvorrichtung zur Homogenisierung \"{u}berfl\"{u}ssig, was zu einem einfacher zu berechnenden Aufbau f\"{u}hrt. Voraussetzung ist, wie schon im oberen Konzept, dass das benutzte Messger\"{a}t durch externe Pumpen betrieben werden kann. Dies ist zum Beispiel beim \textit{OPS 3330} der Fall, somit eignet sich diese Variation des ersten Konzepts vor allem f\"{u}r dieses Messger\"{a}t. Die anderen Teill\"{o}sungen und die Durchf\"{u}hrung dieser Variante sind analog zu erstem Konzept.
\begin{figure}[H]
        \myfloatalign
        \subfloat[Konzept 1 mit Luftmischer]
        {\includegraphics[width=.45\linewidth]{gfx/concepts/Konzept_1.jpg}} \quad
        \subfloat[Variation ohne Luftbeimischung]
        {\includegraphics[width=.45\linewidth]{gfx/concepts/Konzept_2.jpg}} 
        \caption[Konzepte 1 und 2 (Vereinfachung)]
        {Konzepte 1 und 2 (Vereinfachung)}
        \label{fig:concepts_1_2}
\end{figure}


\section{Konzept 3}
%TODO

\subsection{Aufbau}
%TODO

\subsection{Funktionsweise}
%TODO

\subsection{Vorteile}
%TODO

\subsection{Nachteile}
%TODO
\section{Konzept 4}
%TODO

\subsection{Aufbau}
%TODO

\subsection{Funktionsweise}
%TODO

\subsection{Vorteile}
%TODO

\subsection{Nachteile}
%TODO
\section{Konzept 5}
Konzept f\"{u}nf ist eine Art Variation vom vierten Konzept, da der Aufbau grundlegend analog ist. Unterschiede liegen im Partikelmaterial und der Durchf\"{u}hrung des Messungen. Die Partikel werden hier von einem elektronischen Verdampfer einer E-Zigarette erzeugt. Da auch hier hohe Temperaturen entstehen, ist es n\"{o}tig diese vor dem Messen zu regulieren.

\subsection{Aufbau}
Genau wie bei Konzept vier besteht die Versuchseinrichtung aus einem Verd\"{u}nner, einem Thermokonditionierer, einer Schaltvorrichtung mit Ventilator und dem Messger\"{a}t. Zur Schaltvorrichtung f\"{u}hren zum einen der Schlauch von der E-Zigarette und zum anderen ein Rohr zur Umgebungsluft. Ein Kanal am Schaltsystem f\"{u}hrt zu einem extern angeschlossenen Ventilator, ein weiterer f\"{u}hrt zum Messger\"{a}t. Zwischen dem Schaltsystem und der E-Zigarette ist ein Verd\"{u}nner kombiniert mit einem Thermokonditionierer geschaltet. Anders als zu Konzept 4 wird hier anstatt Kerzen, der Rauch der E-Zigarette verwendet. Die Versuchseinrichtung wird direkt am Mundst\"{u}ck der E-Zigarette angeschlossen.
\begin{figure}[H]
        \myfloatalign
        {\includegraphics[width=.9\linewidth]{gfx/concepts/Konzept_5.jpg}} \quad
        \caption[Skizze Konzept 5]
        {Skizze Konzept 5}
        \label{fig:concept_5}
\end{figure}

\subsection{Funktionsweise}
Wie bei Konzept vier saugt das Messger\"{a}t in der ersten Phase gereinigte Umgebungsluft. So kann sich das Messger\"{a}t zun\"{a}chst einstellen und eine Nullz\"{a}hlrate durchf\"{u}hren. Die Schaltvorrichtung befindet sich zu Beginn in \textit{Stellung 0}. W\"{a}hrend der Nullz\"{a}hlrate des Messger\"{a}tes wird der Aerosolstrom durch den am Schaltsystem angeschlossenen Ventilator aufgebaut. Somit ist der Strom bereits geschwindigkeitsbehaftet, wenn dieser zu Messger\"{a}t umgeleitet wird. Da die Partikelanzahlkonzentration zu gro{\ss} f\"{u}r die Messger\"{a}te ist, durchstr\"{o}mt der Volumenstrom zun\"{a}chst einen Rotationsverd\"{u}nner, wo dem Aerosol gereinigte Luft beigemischt wird. Die Kombination aus Verd\"{u}nner und Thermokonditionierer arbeitet analog zu der Anwendung in Konzept vier. Anschlie{\ss}end gelangt der Volumenstrom zum Schaltsystem der Versuchseinrichtung.
\\\\
Zu einer fest eingestellten Zeit wird der Schaltmechanismus in \textit{Stellung 1} geschaltet. Dadurch str\"{o}mt das Aerosol zum Einlass des Messger\"{a}tes und erzeugt dort den Partikelsprung. Die Luftzufuhr zum Messger\"{a}t ist nun unterbrochen. Das abwechselnde Beaufschlagen von gefilterter Umgebungsluft und Aerosol gew\"{a}hrleistet einen ausreichenden Konzentrationsunterschied, um das generierte Sprungsignal zu verst\"{a}rken\cite{auswertemethodik}. Um bei Konzept vier die Totzeit des Systems so gering wie m\"{o}glich zu halten, wird der Verbindungsschlauch von Schaltsystem zu Messger\"{a}teeingang sehr kurz gew\"{a}hlt.

%************************************************
\chapter{Fazit}\label{ch:conclusion}
%************************************************
In diesem letzten Kapitel werden die Konzepte ausgewertet und nach einem vorgestellten System bewertet. Dabei werden VDI-Richtlinien und die Anforderungsliste verwendet, um ein Bewertungssystem zu erstellen. Anschlie{\ss}end wird das am besten bewertete Konzept ausgew\"{a}hlt und eine exakte Konstruktion wird erarbeitet.\\
Abschlie{\ss}end wird es ein Fazit der Arbeit geben und einen Ausblick auf etwaige zuk\"{u}nftige Arbeiten zu diesem Thema.
\section{Auswertung der Konzepte und Auswahl eines Konzepts}
Die entwickelten Konzepte wurden in nachfolgender Tabelle \ref{fig:TabEvalOne} nach den VDI-Richtlinien \textit{2222} und \textit{2225} bewertet. Hierf\"{u}r wurden die Konzeptvarianten mit den Mindest-, Ziel- und Wunschanforderungen aus der Anforderungsliste verglichen und den Punkte von eins bis vier zugeordnet. Unterschieden wurde zwischen technischen und wirtschaftlichen Bewertungsmerkmalen. Da jede Anforderung eine unterschiedliche Relevanz besitzt, wurden zus\"{a}tzlich die Gewichtungsfaktoren ebenfalls von eins bis vier eingef\"{u}hrt.
\begin{figure}[H]
        \myfloatalign
        {\includegraphics[width=.8\linewidth]{gfx/TabEvalOne.jpg}} \quad
        \caption[Auswertung nach Anforderungsliste]
        {Auswertung nach Anforderungsliste}
        \label{fig:TabEvalOne}
\end{figure}
Die Bewertungspunktzahl von jeder Anforderung wurde mit der zugeh\"{o}rigen Gewichtung summiert und anschlie{\ss}end wurden alle Bewertungen aufsummiert. Dieses Verfahren wird an jedem Konzept angewandt, sodass am Ende die Gesamtpunktzahl von jedem Konzept errechnet ist. In diesem Fall gab es f\"{u}nf Konzeptvarianten und somit auch f\"{u}nf Gesamtpunktzahlen auszuwerten.
\\\\
Anders als in der VDI-Richtlinie wurde die wirtschaftliche Wertigkeit nach dem gleichen Verfahren wie die technische Wertigkeit bewertet. Die wirtschaftliche Wertigkeit in den VDI-Richtlinien wird durch die Herstellungskosten bestimmt, welche zus\"{a}tzlich die Ermittlung der Materialkosten nach VDI \textit{2225} erfordert. In dieser Arbeit wird auf dieses Vorgehen verzichtet, da das Hauptaugenmerk auf der technischen Bewertung liegt. Die Kosten wurden als wirtschaftliche Anforderung nicht vernachl\"{a}ssigt, jedoch nur anhand der Herstellerinformationen \"{u}berschlagen. Die Wertigkeit wurde mit folgender Gleichung berechnet
\begin{align*}
	x_g = \frac{g_1p_1 + g_2p_2 + ... + g_np_n}{(g_1 + g_2 + ... + g_n)p_\textit{max}}
\end{align*}
Die Ergebnisse der Wertigkeiten f\"{u}r die Konzeptvarianten eins bist f\"{u}nf sind der Tabelle x zu entnehmen. F\"{u}r eine graphische Darstellung der Ergebnisse wurden die technische Wertigkeit als \(x\)-Koordinate und die wirtschaftliche Wertigkeit als \(y\)-Koordinate in ein sogenanntes \textit{s-Diagramm} eingetragen. Der Punkt mit den Koordinaten \(x\) und \(y\) wird als St\"{a}rke der Konzeptvariante bezeichnet. Die Ideall\"{o}sung \(s_i\) mit den Werte \(x = 1\) und \(y = 1\) wurde auch in das Diagramm eingetragen und mit einer Geraden mit dem Ursprung verbunden. Die Wertigkeit, mit der h\"{o}chsten St\"{a}rke, hat auch die h\"{o}chste technisch-wirtschaftliche Bewertung erhalten, welche in dieser Bewertung die Konzeptvariante vier ist.
\begin{figure}[H]
        \myfloatalign
        {\includegraphics[width=.8\linewidth]{gfx/TabEvalTwo.jpg}} \quad
        \caption[Technische Bewertung der Konzepte]
        {Technische Bewertung der Konzepte}
        \label{fig:TabEvalOne}
\end{figure}
\begin{figure}[H]
        \myfloatalign
        {\includegraphics[width=.8\linewidth]{gfx/TabEvalThree.jpg}} \quad
        \caption[Wirtschaftliche Bewertung der Konzepte]
        {Wirtschaftliche Bewertung der Konzepte}
        \label{fig:TabEvalOne}
\end{figure}
\section{Konstruktion der Konzeptauswahl}
Wie in der Beschreibung des Konzeptes vier bereits erw\"{a}hnt, soll der Partikelsprung mit Hilfe eines Schiebers, der translatorisch oder rotatorisch ausgelegt ist, erzeugt werden. W\"{a}hrend der Konstruktionsphase haben wir uns schlie{\ss}lich f\"{u}r eine rotatorische Auslegung entschieden, auf diese Entscheidung wird im Abschnitt \textit{Schaltvorrichtung} n\"{a}her eingegangen. Der Schieber besteht aus zwei Bauteilen, einem fixen Stator und einer rotierenden Scheibe, dem Rotor. Die restlichen Bauteile lassen sich in die Gruppen Zukaufger\"{a}te und Verbindungselemente kategorisieren. In den folgenden Abschnitten werden alle verwendeten Bauteile f\"{u}r die Umsetzung des Konzeptes detailliert beschrieben.

\subsection{Konstruktion der Bauteile}
Ausgehend von der Konzeptauswertung werden f\"{u}r die Aerosolquelle drei handels\"{u}bliche Kerzen verwendet. Nach der Studie von Pagels et al.\cite{candle} liegt der vollst\"{a}ndige Partikelgr\"{o}{\ss}enbereich von Kerzen zwischen \(16nm - 1000nm\). Partikelgr\"{o}{\ss}en von \"{u}ber \(100nm\) Durchmesser entstehen bei stabil brennenden Kerzen \"{u}ber Partikelwachstum. Ursache ist eine Koagulation \"{u}ber die Zeit. Flackernde Kerzen geben zus\"{a}tzlich Ru{\ss}partikel ab, sodass die Partikelgr\"{o}{\ss}e auf \"{u}ber \(270nm\) steigt.
\\\\
Da das Aerosol in der Versuchseinrichtung direkt nach der Entstehung entnommen wird und der Partikelgr\"{o}{\ss}enbereich zur Abdeckung aller Messger\"{a}te \"{u}ber \(300nm\) liegen muss, wird eine flackernde Kerzenflamme angestrebt. Um einen Luftzug nachzuahmen, wird ein handels\"{u}blicher Ventilator auf der niedrigsten Stufe hinter den Kerzen aufgestellt. Die Kerzen befinden sich in einem hohen offenen Beh\"{a}lter, in dem sich auf der H\"{o}he des Aerosols zwei Durchl\"{a}sse befinden. Ein Durchlass wird vom Eingangsschlauch des Rotationsverd\"{u}nners besetzt, w\"{a}hrend der zweite Durchlass f\"{u}r den erzeugten Luftstrom des Ventilators vorhanden ist.
\\\\
Um die zu hohe Partikelanzahlkonzentration des Aerosols zu regulieren, wird ein Rotationsverd\"{u}nner ben\"{o}tigt. F\"{u}r die Versuchseinrichtung wird der Rotationsverd\"{u}nner und Thermokonditionierer 379020A-30 von TSI verwendet. Das Verd\"{u}nnungsverh\"{a}ltnis ist einstellbar und somit auch kompatibel f\"{u}r alle Messger\"{a}te. Da der Rotationsverd\"{u}nner nur einen maximalen Volumenstrom von \(5L/min\) ausgibt, erh\"{o}ht der zugeschaltete Thermokonditionierer den abgegebenen Volumenstrom. Weiterhin reguliert dieser die Temperatur des Tr\"{a}gergases, da bei fortgeschrittener Versuchszeit eine kontinuierliche Temperatursteigerung aufgrund der Kerze entsteht.
\\\\
Um w\"{a}hrend der Totzeit den Aerosolstrom zu konditionieren, muss dieser in Bewegung versetzt werden, indem er mit Hilfe eines externen Verdichters mit dem gleichen Volumenstrom wie die des verwendeten Messger\"{a}tes angesaugt wird. Da der Versuchsaufbau mehrere Partikelmessger\"{a}te abdecken soll, muss der Volumenstrom des Verdichters einstellbar sein, sodass er den Aerosolstrom f\"{u}r jedes Messger\"{a}t konditionieren kann. Aus der Anforderungsliste l\"{a}sst sich entnehmen, welcher Bereich daf\"{u}r abgedeckt werden muss. Ein Verdichter, der unsere Anforderungen erf\"{u}llt, ist die \textit{Capex V2 Pumpe} der Firma Charles Austen, welche mit Gleichstrom betrieben wird. F\"{u}r den Betrieb wird der Verdichter an ein stufenloses Labornetzger\"{a}t angeschlossen, was es einem erm\"{o}glicht durch Variation der Spannung auch den Volumenstrom so anzupassen, dass er dem des verwendeten Messger\"{a}tes entspricht. Der maximale Volumenstrom den der Verdichter leisten kann, liegt bei \(17 L/min\). Die Ein- und Ausl\"{a}sse haben jeweils einen Au{\ss}endurchmesser von \(8mm\), was in der Gr\"{o}{\ss}enordnung der von uns verwendeten Schl\"{a}uche liegt.
\\\\
Die Luft, welche das Messger\"{a}t w\"{a}hrend der Nullzeit ansaugt, muss frei von Partikeln sein, um beim Wechsel zum Aerosolstrom einen Partikelsprung zu erm\"{o}glichen. Daf\"{u}r muss die Luft mit Hilfe eines HEPA Filters gereinigt werden, bevor sie vom Messger\"{a}t angesaugt wird. F\"{u}r diesen Zweck eignet sich der \textit{HS-Mikroseal JG-S Patronenfilter} der Firma HS-Luftfilterbau, da dieser zum Einbau in Rohrleitungssysteme geeignet ist, welche \"{u}ber kleine Baugr\"{o}{\ss}en mit moderatem Volumenstrom verf\"{u}gen. Der Nennvolumenstrom liegt bei bis zu \(22m^3 / h\) und verursacht einen maximalen Druckunterschied von 
\(1 kPa\). Der Filter beeintr\"{a}chtigt dadurch nicht die Funktionalit\"{a}t des Versuchsaufbaus. Am Austritt des Filters befindet sich ein \textit{G1 Au{\ss}engewinde}, an dem eine Schlaucht\"{u}lle mit einem \textit{G1 Innengewinde} f\"{u}r Schl\"{a}uche mit \(\frac{3}{8} inch\) Innendurchmesser angeschraubt werden muss, um die ben\"{o}tigte Luftleitung zum Stator und zum Rotor legen zu k\"{o}nnen.
\\\\
Alle verwendeten Verbindungen stammen von der Firma \textit{Swagelok}. Nach den Datenbl\"{a}ttern des Messger\"{a}teherstellers \textit{TSI} sind Verbindungen der Firma \textit{Swagelok} mit den Messger\"{a}ten kompatibel. Die Verbindungen sind aus Messung und haben daher eine geringe Rohrreibzahl. In folgender Tabelle \ref{fig:material} werden alle Verbindungen mit ihren Daten vorgestellt.
\begin{figure}[H]
        \myfloatalign
        {\includegraphics[width=.9\linewidth]{gfx/construction/material.pdf}} \quad
        \caption[Verbindungen und Adapter]
        {Verbindungen und Adapter}
        \label{fig:material}
\end{figure}
Der Rotationsverd\"{u}nner ben\"{o}tigt zwei Einschraubverschraubungen mit einem \(\textit{NPT} \frac{1}{4} inch\) Au{\ss}engewinde. Zum Verbinden der Schl\"{a}uche dienen somit die Schlauchverbindungen mit einem Innengewinde \(\textit{NPT} \frac{1}{4} inch\) und einem Zapfen passend f\"{u}r Schl\"{a}uche mit einem Innendurchmesser von \(\frac{3}{8} inch\). Auf Grundlage dieser Anschl\"{u}sse haben die weiteren Anschl\"{u}sse der Verbindungsplatte und der Drehscheibe die gleichen Innengewinde. Somit werden jeweils zwei Schlauchverbindungen an der Verbindungsplatte und an der Drehscheibe ben\"{o}tigt.
\\\\
Die Eing\"{a}nge der Messger\"{a}te sind verschieden positioniert und haben unterschiedliche Anschlussdurchmesser. Daher sind f\"{u}r jedes Messger\"{a}t Adapter notwendig, die an die Versuchseinrichtung angeschlossen werden k\"{o}nnen. Um Verschlei{\ss}erscheinungen, die durch Knicken des Schlauches entstehen, werden f\"{u}r den \textit{FMPS} und \textit{OPS} \(90^\circ\) Winkelverschraubungen genutzt. F\"{u}r den \textit{APS} und \textit{UCPC} werden Einschraubverbindungen in gerader Ausf\"{u}hrung verwendet. Die Verbindung zum Schlauch wird durch eine Schlauchverbindung mit Innengewinde hergestellt. Diese wird auf den Adapter f\"{u}r das jeweilige Messger\"{a}t geschraubt.
\\\\
Als Verbindung zu den Elementen der Versuchseinrichtung werden Vinylschl\"{a}uche verwendet. Der Vinylschlauch ist in Rollen erh\"{a}ltlich und kann zu den gew\"{u}nschten L\"{a}ngen zerschnitten werden. Neue Schl\"{a}uche gelten als technisch glatt\citep{grote} und eignen sich daher f\"{u}r die Versuchseinrichtung. Zur Verteilung der Bewegungen und Biegungen, die unter dem Mindesbiegeradius des Schlauches liegen und aufgrund des Messger\"{a}tebewegtoleranz ist der Schlauch mit ausreichender L\"{a}nge zu w\"{a}hlen. Damit werden fr\"{u}he Verschlei{\ss}erscheinungen vermieden. Unter den Bedingungen der Schlauchverbindungen und der Schlauchf\"{u}hrung ergeben sich folgende Schlauchl\"{a}ngen:
\begin{enumerate}
	\item Kerze zum Verd\"{u}nner: \(480mm\)
	\item Verd\"{u}nntes Aerosol zur Verbindungsplatte: \(160mm\)
	\item HEPA-Filter zur Verbindungsplatte: \(280mm\)
	\item Drehscheibe zum Verdichter: \(240mm\)
	\item Drehscheibe zum Messger\"{a}t: \(240mm\)
\end{enumerate}
Je nach Variation des Aufbaus k\"{o}nnen die Schlauchl\"{a}ngen variieren. Da Vinylschl\"{a}uche einfach in ihrer Beschaffenheit und kosteng\"{u}nstig sind, k\"{o}nnen die Schl\"{a}uche nach Belieben verl\"{a}ngert oder verk\"{u}rzt werden.
\\\\
Die Entscheidung zugunsten eines rotatorischen Schiebers fiel aus mehreren Gr\"{u}nden. Rotatorische Schritt- oder Servomotoren werden gro{\ss}fl\"{a}chig angeboten und sind dazu sehr preisg\"{u}nstig. Die Auslegung des Rotors und des Stators sind konstruktiv einfacher zu realisieren, da diese aus einfachen Grundgeometrien gefertigt werden k\"{o}nnen, w\"{a}hrend der translatorische Schieber die Konstruktion einer komplexen Schiene und einer \"{U}bertragungswelle ben\"{o}tigt. Die Ansteuerung eines rotatorischen Aktuators erweist sich au{\ss}erdem als weniger aufw\"{a}ndig und g\"{u}nstiger, da die Rotationsbewegungen aus dem Motor direkt \"{u}ber Software auslesbar ist, w\"{a}hrend bei einem translatorischen System komplizierte Umrechnungen notwendig sind. Das Ansaugen des Aerosols durch einen Verdichter, sowie der horizontale Verlauf der Str\"{o}me durch den Rotor, tragen zu einem str\"{o}mungsbeg\"{u}nstigendem Schaltvorgang bei.
\\\\
Als bewegliche Komponente ist der Rotor f\"{u}r den Wechsel zwischen Luftstrom und konditioniertem Aerosolstrom zust\"{a}ndig. Die daf\"{u}r notwendige mechanische Energie wird durch einen Schrittmotor zur Verf\"{u}gung gestellt. Der Rotor besteht aus einer 3-D gedruckten, runden Scheibe aus Polyamid mit folgenden Ma{\ss}en:
\begin{enumerate}
	\item Durchmesser: \(120 mm \pm 5 mm\)
	\item Dicke: \(11,5 mm \pm 1 mm\)
\end{enumerate}
Auf der dem Stator abgewandten Seite sind auf einem zum Rotor konzentrischen Kreis mit dem Radius \(4 mm\) zwei durchg\"{a}ngige konische Bohrungen. Die Bohrungen verj\"{u}ngen sich zur R\"{u}ckseite hin und besitzen jeweils ein \(\frac{1}{4} inch\) Innengewinde. Der Winkel zwischen den beiden Bohrungen betr\"{a}gt \(43,2^\circ\). Des Weiteren besitzt die Scheibe eine durchg\"{a}ngige Bohrung mit einem Radius von \(2,5 mm \pm 0,01 mm\) in ihrem Zentrum. Auf der R\"{u}ckseite des Rotors befindet sich eine Matte aus technischem Filz in Form eines symmetrischen Kreissegmentes mit folgenden Ma{\ss}en:
\newpage
\begin{enumerate}
	\item Dicke: \(4 mm \pm 1mm\)
	\item Radius: \(60 mm \pm 2 mm\)
	\item H\"{o}he in der Mitte: \(50 mm \pm 2 mm\)
\end{enumerate}
Die Matte ist auf der R\"{u}ckseite des Rotors mit diesem verklebt. Dabei liegt ihre Symmetrieachse mittig zwischen den beiden Bohrungen und die abgerundete Seite ist konzentrisch zur Scheibe angeordnet. An den Stellen der Bohrung ist die Matte mit kreisf\"{o}rmigen Ausschnitten versehen, welche konzentrisch zu den Bohrungen liegen. Der Durchmesser der Ausschnitte ist je \(1 mm\) gr\"{o}{\ss}er als der der Bohrungen. Der Rotor ist auf die Antriebswelle des Aktuators geschoben, sodass die Filzmatte durch den Stator leicht eingedr\"{u}ckt ist.
\\\\
Der Stator dient der Aufnahme der Kr\"{a}fte und Momente, welche durch den Schaltvorgang entstehen. Bei der Auswahl der Geometrie spielt auch die Funktion des Stators als Schlauchhalter eine Rolle. Au{\ss}erdem ist ein Material gew\"{a}hlt worden, mit welchem eine Dichtung trotz Relativbewegung realisierbar ist. Der Stator besteht aus einer Platte geh\"{a}rtetem Stahl, mit folgenden Ma{\ss}en:
\begin{enumerate}
	\item Tiefe: \(11,5 mm \pm 1 mm\)
	\item H\"{o}he: \(140 mm \pm 5mm\)
	\item Breite: \(110 mm \pm 5 mm\)
\end{enumerate}
Auf der dem Rotor abgewandten Seite befindet sich in H\"{o}he von \(70 mm\) und in der halben Breite \(55 mm\) eine durchgehende Bohrung mit dem Radius von \(3,5 mm \pm 0,25 mm\). In diese Bohrung ist die Antriebswelle des Motors eingef\"{u}hrt, sodass sie circa \(1 mm\) Spiel hat. Der Motor ist in dieser Position durch vier \textit{M3 x 12 Senkkopfschrauben} am Stator fixiert. Au{\ss}erdem ist zur Stabilisierung des Motors ein \( 40mm * 40 mm * 40 mm\) Edelstahlwinkel so am Startor angebracht, dass die Unterseite des Motors auf dem Winkel aufliegt. Der Stator selbst ist mit Hilfe von zwei \(40mm * 40mm * 30mm\) Edelstahlwinkeln an einer Grundplatte montiert. Auf einem zum Umfang der Antriebswelle konzentrischen Kreis mit \(40mm\) Radius befinden sich zwei um \(\pm 21,6^\circ\) von der 12-Uhr-Position versetzten, konisch zulaufende Durchgangsbohrungen, welche sich zum Rotor hin verj\"{u}ngen. In diesen Bohrungen sind jeweils \(\frac{1}{4} inch\) \textit{NPT} Innengewinde geschnitten. zus\"{a}tzlich befindet sich auf der R\"{u}ckseite eine, um \(64,8^\circ\) von der 12-Uhr-Position aus, gegen den Uhrzeigersinn versetzte, durchgehende Bohrung, mit einem Radius von \(5mm\). Die Bohrungen sind auf der Vorderseite mit je einer \(45^\circ\)-Fase versehen, welche der Abrasion der Filzmatte durch scharfe Bohrkanten vorbeugt.
\\\\
Als Aktuator f\"{u}r die Schaltvorrichtung wird der Schrittmotor \textit{QSH4218-41-10-035} der Firma \textit{Trinamic} benutzt. Der Motor hat ein Halte-Drehmoment von \(0,35 NM\) und l\"{a}sst sich \"{u}ber eine Schnittstelle mit einem \textit{Trinamic Board} \"{u}ber passende Software ansteuern. Die Antriebswelle hat eine L\"{a}nge von \(20 mm\) und einen Durchmesser von \(5 mm\). Der Motor hat sich aufgrund des g\"{u}nstigen Preises und der einfachen Ansteuerung durchgesetzt. Die Firma baut eigens f\"{u}r den Motor ein passendes Board zur digitalen Ansteuerung. Dabei wird der Motor mit einem \textit{RS-485 Kabel} mit dem Board verbunden, welches direkt per \textit{USB} mit einem Computer verbunden werden kann. Das Board l\"{a}sst sich frei programmieren und so kann man verschiedene Schaltvorg\"{a}nge implementieren.
\\\\
Abschlie{\ss}end wurde noch eine schematische Darstellung gefertigt (Abbildung \ref{fig:scheme}), welche zeigt wie die einzelnen Elemente des Aufbaus zusammen geh\"{o}ren.
\begin{figure}[H]
        \myfloatalign
        {\includegraphics[width=.9\linewidth]{gfx/conclusion/aufbau.pdf}} \quad
        \caption[Schematische Darstellung des Versuchaufbaus]
        {Schematische Darstellung des Versuchaufbaus}
        \label{fig:scheme}
\end{figure}

\subsection{Durchf\"{u}hrung des Versuchs}
F\"{u}r eine Analyse des dynamischen Verhaltens der Messger\"{a}te sind vor allem deren Totzeiten und die Genauigkeit der ersten Messwerte im Vergleich zu den folgenden Messwerten interessant. Um eine Aussauge \"{u}ber die Totzeit des Messger\"{a}tes treffe zu k\"{o}nnen, muss aus der gesamten Totzeit des Systems, die Totzeit des Versuchsaufbaus abgezogen werden. Diese wird beschrieben durch die Dauer zwischen Generieren des Schaltsignals und dem Ankommen der Partikelfront am Messger\"{a}teeinlass. Ein Anteil dieser Zeitspanne ist die Schaltzeit des Rotors welche bei \(t_s = 0,4 s\) liegt. Der restliche Anteil ist die Zeit, die der Volumenstrom ben\"{o}tigt um von der Schaltvorrichtung bis zum Messger\"{a}teeinlass zu gelangen und errechnet sich aus:
\begin{align*}
t_w = \frac{s * \pi * r^2}{V'}
\end{align*} 
mit \(s = \textit{Wegstrecke zwischen Schaltvorrichtung und Messger\"{a}teeinlass} = 250 mm\), \(r = \textit{Schlauchradius} = 4,7625 mm\) und \(V' = \textit{Volumenstrom des Messger\"{a}tes}\).\\
Damit diese Beziehung gilt muss das Verhalten der Str\"{o}mung im Schlauch zum Messger\"{a}t laminar sein. Um zu zeigen, dass dies erf\"{u}llt ist, wird unter Vernachl\"{a}ssigung der Querschnitts\"{u}berg\"{a}nge am Messger\"{a}teeinlass und am Rotorauslass, im Folgenden die Reynoldszahl f\"{u}r diesen Weg berechnet. Dabei ist zu erw\"{a}hnen, dass nur die Durchf\"{u}hrung am \textit{FMPS} betrachtet wird, da dieser den h\"{o}chsten Volumendurchsatz besitzt und die anderen Parameter unabh\"{a}ngig vom Messger\"{a}t sind.
\begin{align*}
Re = \frac{2V'}{r * \pi * \nu_\textit{Luft}} = 137 < 2300
\end{align*}
wobei \(V' = 1,67*10^{-5}\frac{m^3}{s}\), \(r = 4,7625 * 10^{-3}m\) und \(\nu_\textit{Luft}(30^\circ C) = 163 * 10^{-7} \frac{m^2}{s}\).
Daraus folgt, dass sich der Strom bei isolierter Betrachtung dieses Weges laminar verh\"{a}lt. Weiterhin ergeben sich nun die Totzeiten unseres Versuchsaufbaus abh\"{a}ngig von den verschiedenen Messger\"{a}ten durch\\
\(t_\textit{Versuch, Messger\"{a}t} = t_w + t_s\):
\begin{itemize}
	\item \(t_\textit{Versuch, FMPS} = 0,507s\)
	\item \(t_\textit{Versuch, APS} = 0,614s\)
	\item \(t_\textit{Versuch, OPS} = 1,47s\)
	\item \(t_\textit{Versuch, UCPC} = 1,11s\)
\end{itemize}
Da bei der Kerze als Aerosolquelle eine relativ hohe Streuung der Partikelanzahlkonzentrationswerte zu erwarten ist, muss der Pr\"{u}fvorgang mehrere Pr\"{u}fdurchg\"{a}nge enthalten. Die hohe Filterwirkung des \textit{HEPA-Filters} garantiert hierbei trotzdem, dass die minimale Partikelanzahlkonzentrationsdifferenz eingehalten wird.\\
Der Pr\"{u}faufbau sieht folgenderma{\ss}en aus:
\begin{enumerate}
	\item Alle Schl\"{a}uche werden auf die f\"{u}r sie vorgesehene Steckpl\"{a}tze gesteckt. 
	\item Das Messger\"{a}t wird mit dem Versuchsaufbau verbunden.
	\item Die Kerzen werden angez\"{u}ndet.
	\item Der Motor wird angeschaltet und In \textit{Stellung 0} gebracht.
	\item Das Messger\"{a}t und die Pumpe werden angeschaltet.
\end{enumerate}
Nachdem sich die Pumpe und das Messger\"{a}t eingelaufen haben und die erw\"{u}nschten Volumenstr\"{o}me erreicht sind kann die eigentliche Pr\"{u}fdurchf\"{u}hrung beginnen. Je nach Messger\"{a}t wird zuerst der Zeitpunkt des Eintreffens der Partikelfront auf den Messger\"{a}teeinlass nach der Schaltung bestimmt. Dies passiert in Abh\"{a}ngigkeit der vorher berechneten Totzeit \(t_\textit{Versuch, Messger\"{a}t}\). Der Rotor wird jetzt in \textit{Stellung 1} gebracht und das gemessene Partikelsprungsignal wird aufgezeichnet. Der Partikelmesswert nach dem Sprung wird abgespeichert und der Rotor wird wieder in \textit{Stellung 0} geschaltet. Ist der Nullwert wieder erreicht kann die n\"{a}chste Messung durchgef\"{u}hrt werden.\\
Um das Ergebnis der Versuchseinrichtung als valide annehmen zu k\"{o}nnen muss eine statistische Messreihe erstellt werden. Daf\"{u}r werden nun so viele Versuche durchgef\"{u}hrt, bis das arithmetische Mittel der Messreihe sich nicht mehr ausschlaggebend \"{a}ndert. Eine Messreihe wird von uns als valide angenommen wenn die Mittelwert\"{a}nderung unter \(0,001\) ist, der Wert also eine Genauigkeit von \(99,9\%\) angenommen hat.

\subsection{Kosten}
F\"{u}r die Kosten des Versuchsaufbaus wurden die Kosten von jedem Bauteil ermittelt und aufsummiert. Dabei ist zu beachten, dass nicht alle Preise fest zu sehen sind, da sie immer nur auf Nachfrage bei den Herstellern herausgegeben werden und sich eventuell \"{a}ndern k\"{o}nnen. Die Kosten f\"{u}r die 3-D gedruckten Teile wurden aus Erfahrungswerten und dem Preis f\"{u}r 3-D Druck Material ermittelt.\\
Die Gesamtkosten sind in folgender Tabelle \ref{fig:kosten} einzusehen:
\begin{figure}[H]
        \myfloatalign
        {\includegraphics[width=.9\linewidth]{gfx/construction/kosten.pdf}} \quad
        \caption[Kosten aller Teile]
        {Kosten aller Teile}
        \label{fig:kosten}
\end{figure}

\subsection{Evaluation}
Die Genauigkeit und Korrektheit des Versuchs h\"{a}ngt ma{\ss}geblich von zwei Faktoren ab. Einmal von den Druckverh\"{a}ltnissen in den Schl\"{a}uchen und Verbindungen und andererseits von der Schaltzeit des Rotors.\\
Die folgende Druckberechnung verdeutlicht die Druckdifferenz, die aufgrund der Schlauch und Schlauchverbindungsreibungen  zwischen dem
Rotationsverd\"{u}nner und dem Messger\"{a}teingang anf\"{a}llt. Die Schlauchverbindungen bestehen aus Messing und die Schl\"{a}uche aus Polyvinylchlorid (weiches PVC). Da das Aerosol des Kerzenrauchs im Rotationsverd\"{u}nner verd\"{u}nnt wird, ist f\"{u}r den Druckgradienten am Messger\"{a}teinlass der Weg vom Verd\"{u}nner aus relevant. Dieser betr\"{a}gt ungef\"{a}hr 
\begin{align*}
l_{gesamt} = l_{Messing} + l_{PVC} = 170mm + 320mm = 530mm
\end{align*}
Der Druckgradient berechnet sich zu
\begin{align*}
\Delta p_\textit{Messger\"{a}t} = \zeta_\textit{Messing} \frac{l_\textit{Messing}}{d_\textit{Messing}}\frac{\rho * u^{2}_\textit{Messger\"{a}t, Messing}}{2} + \zeta_\textit{PVC} \frac{l_\textit{PVC}}{d_\textit{PVC}} \frac{\rho * u^{2}_\textit{Messger\"{a}t, PVC}}{2}
\end{align*} 
\begin{center}
\begin{tabular}{c | c}
$\zeta_{i}$ & Widerstandsbeiwert\\
\hline
$d_{i}$ & Innendurchmesser der Verbindung\\
\hline
$\rho$ & Dichte des Volumenstroms\\
\hline
$u_{j,i}$ & Geschwindigkeit des Aerosolstroms,\\
& die sich aus dem Volumenstrom ergibt,\\
& welchen die Messger\"{a}te erzeugen
\end{tabular}
\end{center}
F\"{u}r laminare Str\"{o}mungen l\"{a}sst sich der Rohrreibungskoeffizient nach dem Gesetz von Hagen-Poiseuille bestimmen:
\[\lambda_{i} = \frac{64}{Re_{i}}=\zeta_{i}\]
\[\zeta_{Messing_\textit{max}} = 0,0015\]
\[\zeta_{PVC_\textit{max}} = 0,0015\]
Es kann die Dichte des Tr\"{a}gergases angenommen werden. Diese betr\"{a}gt bei $25^{\circ}C$ \[\rho = 1,18 \frac{kg}{m^{3}}\]\\
Des Weiteren betragen die Durchmesser
\[d_{Messing} = 4,8 mm\]
\[d_{PVC} = 9,5 mm\]
Im Folgenden wird nur die Druckdifferenz, die bei Nutzung des  FMPS entsteht, berechnet. Da dieser den h\"{o}chsten Volumenstrom aller verwendeten Messger\"{a}te erzeugt, ist folglich die Geschwindigkeit des Aerosolstroms hier am gr\"{o}{\ss}ten. Relevant ist diese Druckdifferenz, da die Druckgradienten bei der Nutzung der anderen Messger\"{a}te geringer ausfallen.
\\\\
Mit den Geschwindigkeiten des Aerosolstroms \[u_{FMPS, Messing} = 9,21 \frac{m}{s}\] \[u_{FMPS, PVC} = 2,35 \frac{m}{s}\] ergibt sich der Druckunterschied f\"{u}r den FMPS zu \[\Delta p_{FMPS} = 2,8 Pa\]
\\\\
Aufgrund der Wahl von technisch glatten Rohren und der kurzen Versuchsstrecke sind die entstehenden Druckgradienten f\"{u}r die Messger\"{a}te vernachl\"{a}ssigbar und k\"{o}nnen ohne weitere Probleme ausgeglichen werden.
\\\\
Der zweite Faktor ist die Schaltzeit die unter anderem vom Motor abh\"{a}ngt. Dazu wurde die Schaltzeit mit dem verwendeten Motor mit allen wichtigen Komponenten berechnet. F\"{u}r die folgende Rechnung wurde zur Vereinfachung die Reibung der Filzdichtung ignoriert.
Momentengleichung um Rotorachse:
\[M_\textit{Rotor}: \Theta * \phi'' = M_M\]
\[\Rightarrow M_M = \frac{2*\phi*\Theta}{t^2}\]
\[\Rightarrow t = \sqrt{\frac{2*\phi*\Theta}{M_M}}\]
\[M_M = 0,35Nm\]
\[\phi = 43,2^\circ = 0,754 \textit{rad}\]
Das Fl\"{a}chentr\"{a}gheitsmoment ergibt sich zu:
\[\Theta = \Theta_\textit{Scheibe} + \Theta_{Adapter} + 2m_\textit{Adapter} * d^2\]
\[d = 4 * 10^{-2}m\]
\[n_\textit{Adapter} = 0,018kg\]
\[\Theta_\textit{Scheibe} = \frac{1}{2}*m_\textit{Scheibe} * r^2 = \frac{1}{2} * 1140\frac{kg}{m^3}*\pi *(6*10^{-2}m)^2 * 1,15 * 10^{-2}m = 0,0236kgm^2\]
\[\Theta_\textit{Adapter} = \frac{m*(r_1^2+r_2^2)}{2} = \frac{0,018kg * ((7,6*10^{-2}m)^2+(16,6*10^{-2}m)^2)}{2} = 3*10^{-4}kgm^2\]
\[\Rightarrow \Theta = 0,02426kgm^2\]
\\\\
\[\Rightarrow t = \sqrt{\frac{2*0,754\textit{rad}*0,02426 kgm^2}{0,35Nm}} = 0,3233s\]
\\\\
Mit gew\"{a}hltem Motor l\"{a}sst sich also eine Schaltzeit von \(0,3233s\) erreichen, was die angestrebte Schaltzeit von \(0,4s\) unterschreitet und somit f\"{u}r den Versuchsaufbau ausreichend ist.\\


\section{Fazit und Ausblick}
F\"{u}r zuk\"{u}nftige Arbeiten gilt es auf jeden Fall, den Versuchsaufbau mit Hilfe des Konstruktionsplans baulich umzusetzen.
\\\\
Unsere Annahme \"{u}ber das laminare Verhalten des Aerosolstroms beruht auf unserer Wahl, technisch glatte Materialien f\"{u}r die Aerosolleitungen zu verwenden und die Verbindungen und Leitungen m\"{o}glichst str\"{o}mungsg\"{u}nstig auszulegen. Eine Str\"{o}mungssimulation kann hierbei mehr Aufschl\"{u}sse \"{u}ber das Str\"{o}mungsverhalten geben.
\\\\
Zus\"{a}tzlich lie{\ss}e sich die Schaltzeit durch den Einsatz einer Federauslegung statt eines Elektromotors verbessern, da so sehr viel h\"{o}here Drehmomente erzeugt werden k\"{o}nnen als mit einem Elektromotor. Dies konnte von uns aufgrund der technischen Komplexit\"{a}t nicht umgesetzt werden.
\\\\
Als Fazit l\"{a}sst sich sagen, dass eine robuste Versuchseinrichtung entworfen wurde um das zeitliche dynamische Verhalten von Partikelmessger\"{a}ten zu bestimmen. Daf\"{u}r wurden mit Hilfe der VDI Richtlinien mehrere Konzepte entwickelt und an Hand von bestimmten Anforderungen ausgewertet. Daraufhin wurde das beste Konzept ausgew\"{a}hlt und eine detaillierte technische Konstruktion entworfen. Es wurden Probleme erl\"{a}utert, welche f\"{u}r einen Versuchsaufbau dieser Art gel\"{o}st werden mussten und es wurde ein Ausblick gegeben, wie die vorgestellte Einrichtung noch verbessert und erweitert werden k\"{o}nnte.



