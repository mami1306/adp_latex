%*****************************************
\chapter{Einf\"{u}hrung}
%*****************************************
\section{Motivation}
Im Bereich der Fahrzeugentwicklung spielen seit Jahren Leistung und Energieverbrauch die gr\"{o}{\ss}te Rolle als Fortschrittskriterium. Durch steigende Belastung der Umwelt und der Gesundheit von Mensch und Tier, sind Umweltschutzfaktoren bei der Entwicklung von Kraftfahrzeugen immer weiter in den Vordergrund ger\"{u}ckt. Immer mehr gesetzliche Richtlinien schr\"{a}nken die Automobilhersteller in ihren Freiheiten ein und zwingen die Entwicklung in eine umweltfreundlichere Richtung. Anfangs zielte man nur auf Verbrennungsprodukte ab, die bei der Verwendung von Otto- und Dieselmotoren entstehen. Sp\"{a}ter kamen Altschrottentsorgung und Reifenabrieb hinzu. 
\\\\
Neben all diesen Faktoren ist auch die Bremse ein Bauteil im Kraftfahrzeug, welches durch Benutzung Emissionen abgibt. Da auch in diesem Teilgebiet gesetzliche Richtlinien zu erwarten sind, konzentriert sich die Forschung aktuell auf die Entstehung solcher Emissionen bei Kraftfahrzeugbremsen. Da Emissionsart und Partikelzusammensetzung der Bremsemissionen aktuell nicht gut erforscht und deshalb unbekannt sind, kann man nicht direkt Ma{\ss}nahmen gegen die Belastung durch Bremsemissionen ergreifen. Um Aufschluss \"{u}ber die Art der entstehenden Partikel zu bekommen, werden Partikelmesssysteme eingesetzt. An diesem Punkt steigt diese Arbeit ein um das zeitliche Messverhalten solcher Systeme zu \"{u}berpr\"{u}fen, damit dieser sp\"{a}ter an der Kraftfahrzeugbremse eingesetzt werden k\"{o}nnen.
\section{Voraussetzungen}


%*****************************************
\chapter{Projektdefinition und Zeitmanagement}\label{ch:project}
%*****************************************
\section{Projektziele}
Ziel unseres Projektes ist es Konzepte f\"{u}r einen Versuchsaufbau zu erarbeiten und evaluieren. Mit Hilfe des Versuchsaufbau soll die Zeitkonstante f\"{u}r verschiedene Partikelmessger\"{a}te ermittelt werden.
\section{Aufgaben}
Zur Entwicklung eines Konzepts f\"{u}r die Generierung eines Partikeltestsignals sind folgende Aufgaben zu erf\"{u}llen:
\begin{enumerate}
	\item Einarbeitung in die Themen Bremspartikelemission, Partikelmesstechnik und Systemidentifikation mit Hilfe von Testsignalen
	\item Ausarbeitung einer Anforderungsliste f\"{u}r den Versuchsaufbau
	\item Erarbeitung von Versuchseinrichtungskonzepten
	\item Vergleich der erarbeiteten Konzepte an Hand von Kriterien aus der Anforderungsliste
	\item Auswahl eines geeigneten Konzeptes
	\item Konstruktion des ausgew\"{a}hlten Konzeptes
	\item Dokumentation der Ergebnisse 
\end{enumerate}
\section{Anforderungen}
\subsection{Aerosole}
\subsection{Versuchsaufbau}

%*****************************************
\chapter{Technische Grundlagen}\label{ch:foundations}
%*****************************************
\section{Str\"{o}mungsmechanik}
\subsection{Str\"{o}mungseigenschaften}
\subsubsection{Reynolds- und Prandtlzahl}

\section{Reinraumtechnik}
\subsection{Eigenschaften von Partikeln}
\subsubsection{Bremsemissionspartikel}
\subsection{Partikelmessverfahren}
\subsection{Aerosole}

\section{Mechanische Grundlagen}
\subsection{Ventile (Noch nicht fest)}
\subsection{Luftfiltersysteme (Noch nicht fest)}


%************************************************
\chapter{Versuchsplattform}\label{ch:platform}
%************************************************
\section{Partikelmessger\"{a}te}
\subsection{OPS-3330}
\subsection{FMPS-3091}

\section{Simulation (Unsicher)}
\subsection{SpaceClaim (Unsicher)}
\subsection{Fluent (Unsicher)}

\section{Partikelgeneratoren}
\subsection{Topas ATM 220}
\subsection{Palas 2000H}


%************************************************
\chapter{Analyse von Pr\"{u}f-Aerosole}\label{ch:aerosol}
%************************************************
\section{Di-Ethyl-Hexyl-Sebacat (DEHS)}
\section{Di-N-Octylphtalat (DOP)}
\section{Emery 3004 (PAO-4)}
\section{Poly Styrene Latex Spheres (PSL)}
\section{Auswertung der Analyse}
\subsection{Anforderungsvergleich der Aerosole}

\addtocontents{toc}{\protect\newpage}

%************************************************
\chapter{Konzepte f\"{u}r den Versuchsaufbau}\label{ch:concepts}
%************************************************
\section{Konzept 1}
\subsection{Aufbau}
\section{Konzept 2}
\section{Konzept 3}
\section{Konzept 4}
\section{Konzept 5}


%************************************************
\chapter{Evaluation}\label{ch:evaluation}
%************************************************

\section{Analyse der Konzepte}
\section{Evaluation der Ergebnisse}

%************************************************
\chapter{Fazit}\label{ch:conclusion}
%************************************************

\addcontentsline{toc}{chapter}{\listfigurename}

