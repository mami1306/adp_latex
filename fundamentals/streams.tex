\section{Str\"{o}mungslehre}
Die Str\"{o}mungslehre ist die Wissenschaft vom physikalischen Verhalten von Fluiden und Gasen. Die in dieser Lehre gewonnenen Kenntnisse sind Gesetzm\"{a}{\ss}igkeiten in Str\"{o}mungsvorg\"{a}ngen und dienen der L\"{o}sung von Str\"{o}mungsproblemen. Dabei liegt der Fokus auf den Problemen bei umstr\"{o}mten oder durchstr\"{o}mten Bauteilen. Gegenstand der Str\"{o}mungslehre sind die Bewegungen von Fluiden, Gasen und ruhenden, fließenden oder str\"{o}menden Substanzen. Die Str\"{o}mungslehre l\"{a}st sich in verschiedene Teilgebiete unterteilen von denen allerdings f\"{u}r diese Arbeit nur die Fluiddynamik relevant ist und die Auslegungen sich daher auf dieses Teilgebiet beschr\"{a}nken werden.

\subsection{Str\"{o}mungseigenschaften}
%TODO

\subsection{Reynolds- und Prandtlzahl}
%TODO