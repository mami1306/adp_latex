\section{Partikeleigenschaften}
Die in dieser Arbeit verwendeten Partikel sind mineralische und organische Feststoffe, die nicht in L\"{o}sung gehen und wegen ihrer geringen Gr\"{o}{\ss}e und damit geringen Gewichts in der Schwebe gehalten oder durch geringe Bewegungen des Mediums verteilt werden. Im speziellen sind das hier Aerosolpartikel, St\"{a}ube und auch Nebel- und Rauchpartikel.\\
Die Eigenschaft von Partikeln \"{u}ber l\"{a}ngere Zeit in Gasen transportiert werden zu k\"{o}nnen, liegt darin, dass sie sich mit abnehmendem Durchmesser immer mehr wie Gas Molek\"{u}le verhalten.\\
Partikel entstehen durch verschiedene Vorg\"{a}nge, welche ausschlaggebend f\"{u}r die Gr\"{o}{\ss}e der Partikel sind. Die kleinsten Partikel entstehen durch Verbrennungen und sind selten gr\"{o}{\ss}er als \(1 nm\). Mineralstaubpartikel hingegen gehen auf Abrieb mineralischer Stoffe zur\"{u}ck und sind meist von gr\"{o}{\ss}erer Form im \(\mu m\) Bereich.\\
Partikel k\"{o}nnen vom Menschen eingeatmet werden, dabei bleiben bis zu 10\% der Partikel im Atemtrakt h\"{a}ngen und f\"{u}hren so zu einer gro{\ss}en Belastung in Lungen und Bronchien. Da es abh\"{a}ngig von der Partikelgr\"{o}{\ss}e ist, ob und wie sch\"{a}dlich die Partikel sind, ist es wichtig durch Messungen zu erfahren, welche Eigenschaften und Aufbau die Partikel haben, die durch den Bremsvorgang entstehen\cite{reinraumtechnik}.
\newpage

\subsection{Bremsemissionspartikel}
Die Emissionspartikel von Bremsen  sind Bestandteil des Bremsstaubes, welcher beim Bremsvorgang entsteht. Sie stellen ein Gemisch aus Bremsbelagabrieb und Abrieb der Bremsscheiben dar, wobei der gr\"{o}{\ss}ere Anteil auf die Bremsbel\"{a}ge f\"{a}llt, da diese wesentlich weicher sind. Kritisch beim Bremsstaub sind die Partikel von Eisen, Kupfer und Mangan, die beim scharfen Bremsen freigesetzt werden. Die toxische Wirkung der einzelnen Partikel k\"{o}nnen Entz\"{u}ndungen in den Lungenzellen verursachen.\\
Da es auf das Bremsverhalten ankommt, welche Partikel in welchen Massen entstehen, ist es wichtig die entstehenden Partikel bei verschiedenen Bremsvorg\"{a}ngen zu messen und zu analysieren. Nur dann k\"{o}nnen gezielt Partikel bestimmter Art und Sorte reduziert oder abgefangen werden. So ist zum Beispiel das krebserregende Asbest auch aus der Entwicklung von Bremsscheiben herausgenommen worden\cite{envyl}.

\subsection{Partikelmessverfahren}
Unter dem Begriff Partikelmessung wird eine Gruppe von Messverfahren zur Qualifizierung und Quantifizierung von Partikeln unterschiedlicher Natur zusammengefasst. Dabei wird meistens das Prinzip der Streulichtpartikelmessung angewendet. Bei der Streulichtpartikelmessung wird eine definierte Menge einer Probe durch einen Laserstrahl gef\"{u}hrt. Das Licht des Laserstrahls bricht sich an den Partikeln oder wird von ihnen absorbiert. Photodioden k\"{o}nnen diese Effekte messtechnisch erfassen und in ein elektrisches Signal umwandeln. Dieses Signal wird von einem Computer mit einem zuvor aufgenommenen Signal verglichen, bei dem Latexkugeln definierter Gr\"{o}{\ss}e vermessen worden sind, um Referenzwerte zu schaffen. Mit diesen Daten ist es m\"{o}glich aufgrund von statistischen Methoden die Anzahl der Partikel in einem Kubikmeter Luft oder Gas zu ermitteln. Dieses Verfahren funktioniert allerdings nur f\"{u}r Partikel mit einer Gr\"{o}{\ss}e von unter \(25 \mu m\)\cite{reinraumtechnik}.

\subsection{Aerosole}
Ein Aerosol ist ein heterogenes Gemisch aus festen oder fl\"{u}ssigen Partikeln in einem Gas. Es handelt sich um ein dynamisches System und unterliegt st\"{a}ndigen \"{A}nderungen durch Kondensation von D\"{a}mpfen an bereits vorhandenen Partikeln, Verdampfen fl\"{u}ssiger Bestandteile der Partikel oder Abscheidung von Teilchen an umgebenden Gegenst\"{a}nden.\\
Aerosole k\"{o}nnen durch mechanische Zerkleinerung von Material oder durch Kondensation von Material entstehen. Mechanische Prozesse umfassen Zerreiben, Zerst\"{a}uben oder andere Zerkleinerungsprozesse von Feststoffen. Kondensation hingegen ist ein Prozess, in dem sich festes oder fl\"{u}ssiges Material aus \"{u}bers\"{a}ttigten Gasen bildet. Der Durchmesser von Aerosolpartikeln liegt in der Gr\"{o}{\ss}enordnung zwischen \(0,1 nm\) und \(10 \mu m\) und sie haben unterschiedliche Zusammensetzungen.
\\\\
Angewendet werden Aerosole gro{\ss}fl\"{a}chig in der Industrie als Spraybasis, zum Auftragen von Lacken, medizinische Sprays und als Industriek\"{u}hlmittel. Um f\"{u}r die Zwecke dieser Arbeit einen Partikelsprung zu erzeugen hat sich ein Aerosol als Partikeltr\"{a}ger angeboten, da die Partikel in Aerosolen gleich verteilt sind und zu laminaren Str\"{o}mungen neigen\cite{aerosole}.