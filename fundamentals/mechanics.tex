\section{Mechanische Grundlagen}
In den vorgestellten Konzepten werden verschiedene mechanische Bauteile verwendet. Dabei werden einfache Teile wie Rohre, Klemmen und Schläuche als trivial angenommen und es wird nicht weiter auf die Eigenschaften dieser Teile eingegangen. Kompliziertere Bauteile sollen hier allerdings kurz erläutert werden um das Verständnis für deren Funktion zu vereinfachen und die Anwendung zu verstehen.
\\\\
Da der aufgebaute Aerosolstrom durch ein System von mehreren Rohren und Schläuchen geleitet werden muss und gegebenenfalls auch umgeleitet werden muss, werden Ventile benötigt. Ein weiteres Bauteil sind Luftfilter um die zum Mischen verwendete Umgebungsluft zu reinigen. Letztendlich werden noch Ventilatoren für einige Konzepte benötigt um dem Strom eine konstante Geschwindigkeit zu geben.

\subsection{Ventile}
Ein Ventil ist ein Bauteil zur Absperrung oder Regelung des Durchflusses von Fluiden. Innerhalb des Bauteils wird ein Verschlussteil nahezu parallel zur Strömungsrichtung des Fluids bewegt. Die Strömung kann reduziert oder unterbrochen, indem das gesamte Verschlussteil an eine passend geformte Öffnung gepresst wird. Ventile haben die Eigenschaft über den gesamten Stellbereich ein gleichmäßiges Strömungsbild zu besitzen, weshalb sie sich gut für Regelaufgaben eignen.
\\\\
Ventile lassen sich in vier Kategorien einteilen. Durchgangsventile reduzieren Strömungen und der Eintritt liegt in der selben Richtung wie der Austritt. Eckventile leiten den Strom um, indem Ein- und Austritt im rechten Winkel zueinander liegen, während Schrägventile die Strömungsrichtung um 45° ändern. Die für diese Arbeit wichtigen Ventile sind Drei-Wege-Ventile, die für das kontrollierte Mischen von Fluidströmen verwendet werden. Bei allen Kategorien kann man noch Unterschiede in den Betätigungsarten machen, um jedoch ein Signal für den Messbereich zu haben, werden hier elektronische Betätigungen bevorzugt.

\subsection{Luftfiltersysteme}
Als Luftfilter werden alle Abscheider bezeichnet, die Aerosole oder andere unerwünschte Schwebstoffe aus der Luft herausfiltern. Es gibt viele verschiedene Bauarten von Luftfiltersystemen, doch es werden nur Trockenfilter, Zyklonabscheider und elektrostatische Luftfilter verwendet, weshalb die flüssigkeitsbasierten Filter hier nicht wieder erwähnt werden. \\
Trockenfilter bestehen aus ringförmig oder rechteckig angeordneten Gebilde, die zickzackförmiges Gewebe als Filterelement haben. Die Faltung vergrößert die Filterfläche und verringert den Strömungswiderstand. Die kontaminierte und ungefilterte Luft wird ausserhalb des Zylinders angesaugt und die gefilterte saubere Luft wird innerhalb der Lamellen weitergeleitet. Die Partikel und Schwebeteilchen bleiben in den Gewebefalten hängen.\\
Elektrofilter finden hauptsächlich Anwendung bei der Abscheidung von Aerosolen, sind also für die Arbeit sehr gut geeignet. Im Filter werden die Staub- oder Aerosolpartikel elektrostatisch aufgeladen und an den Elektrodenflächen abgeschieden. In getakteten Abfolgen werden die aufgefangenen Verschmutzungen maschinell entfernt und in einem Trichter gesammelt. Der Vorteil dieser Filter im Vergleich zu den Trockenfiltern ist, dass es keine Filtereinsätze gibt, die ausgetauscht werden müssen, allerdings sind die elektrostatischen Filter wesentlich teurer als die Wegwerffilter mit Gewebeeinsatz.\\
Zyklonabscheider werden dazu genutzt um Partikel einer bestimmten Größe aus dem Strom herauszufiltern. Diese finden ihre Anwendung um nur Partikelgrößen durchzulassen, die für die verwendeten Messgeräte erkennbar sind. Sie sind allerdings zu ineffizient um die Luft von allen Partikeln zu reinigen, weshalb sie nur zur Aufbereitung des Partikelstroms in Frage kommen.

\subsection{Ventilatoren}
Ventilatoren werden in den Konzepten verwendet um einen konstanten Strom zu erzeugen. Mit ihnen können Fluide und Gase gefördert werden. Dafür wird die Antriebsart des Ventilators in die Bewegungsenergie des Mediums umgewandelt ohne dass es in der Strömung zu einer Drucksteigerung kommt. Ein Ventilator ist eine fremd angetriebene Strömungsmaschine, die mittels eines in einem Gehäuse rotierenden Laufrads ein gasförmiges Medium fördert. 